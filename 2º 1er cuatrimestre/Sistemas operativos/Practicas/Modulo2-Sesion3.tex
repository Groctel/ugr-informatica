\section{Llamadas al sistema para el control de procesos}

\subsection{Creación de procesos}

\subsubsection{Identificadores de procesos}

Cada proceso tiene un identificador unívoco o \code{PID}, que es un entero no negativo.
Se tiene que para el proceso principal del sistema, \code{init}, \code{PID=1}.
Además del \code{PID}, existen otros identificadores asociados a procesos:

\begin{lstlisting}[language=C]
#include <unistd.h>
#include <sys/types.h>
pid_t getpid(void);  // PID del proceso invocante
pid_t getppid(void); // PID del padre del proceso invocante
uid_t getuid(void);  // UID del usuario invocante real
uid_t geteuid(void); // UID del usuario invocante efectivo
gid_t getgid(void);  // GID del usuario invocante real
gid_t getegid(void); // GID del usuario invocante efectivo
\end{lstlisting}

Distinguimos entre usuario real y efectivo, siendo el primero el que ejecuta un proceso y el segundo aquél con el que se identifica el primero.
Por ejemplo, al ejecutar una orden precedida de \code{sudo}, el usuario real es quien la ejecuta y el usuario efectivo es \code{root}.

\subsubsection{\code{fork}}



\begin{lstlisting}[language=C]
\end{lstlisting}
\begin{lstlisting}[language=C]
\end{lstlisting}
\begin{lstlisting}[language=C]
\end{lstlisting}
\begin{lstlisting}[language=C]
\end{lstlisting}
\begin{lstlisting}[language=C]
\end{lstlisting}
\begin{lstlisting}[language=C]
\end{lstlisting}
