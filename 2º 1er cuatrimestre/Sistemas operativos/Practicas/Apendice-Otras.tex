\chapter{Otras bibliotecas}

\section{\texttt{exec}}\label{exec}

\begin{lstlisting}[language=C]
#include <unistd.h>

extern char** environ;

int execl   (const char* pathname, const char* arg, ...
             /*, (char*) NULL */);
int execlp  (const char* file, const char* arg, ...
             /*, (char*) NULL */);
int execle  (const char* pathname, const char* arg, ...
             /*, (char*) NULL, char* const envp[] */);
int execv   (const char* pathname, char* const argv[]);
int execvp  (const char* file, char* const argv[]);
int execvpe (const char* file, char* const argv[],
             char* const envp[]);
\end{lstlisting}

Todas las funciones de la familia \texttt{exec} reemplazan la imagen del proceso incovante con una del proceso invocado, actuando como interfaces de la llamada al sistema \texttt{execve}.
Todas estas funciones devuelven, únicamente si ocurre un error, \texttt{-1} acompañado por el \texttt{errno} apropiado.
Reciben como primer argumento el nombre del fichero que se va a ejecutar y se dividen en varias categorías según sus sufijos:

\subsection{\texttt{e}: \texttt{execle()} y \texttt{execvpe()}}

El entorno del invocante se especifica mediante el argumento \texttt{envp}, que es un array de punteros a cadenas de caracteres de finalización nula y debe acabar obligatoriamente en un puntero nulo.
El resto de funciones \texttt{exec} (que no incluyen la \texttt{e}) toman el entorno para la nueva imagen del proceso de la variable externa \texttt{environ} en la invocación.

\subsection{\texttt{l}: \texttt{execl()}, \texttt{execlp()}y \texttt{execle()}}

El argumento \texttt{arg} se interpreta como \texttt{arg0, arg1, ..., argn}, siendo todos cadenas de caracteres de finalización nula que representan los argumentos con los que se ejecuta el proceso.
El último argumento debe ser obligatoriamente un puntero nulo casteado a \texttt{(char*) NULL}.

\subsection{\texttt{p}: \texttt{execlp()}, \texttt{execvp()} y \texttt{execvpe()}}

Duplican las acciones de la shell para la búsqueda del ejecutable si éste no contiene un carácter \texttt{/}.
Éste se busca en \texttt{\$PATH}.
En caso de que no esté definido, se busca por defecto en el valor que devuelve \texttt{confstr(\_CS\_PATH)} (generalmente \texttt{/bin:/usr/bin}).
Si la ruta sí contiene \texttt{/}, se ignora \texttt{\$PATH} y se busca en la ruta especificada.

\subsection{\texttt{v}: \texttt{execv()}, \texttt{execvp()} y \texttt{execvpe()}}

El argumento \texttt{argv} es un array de punteros a cadenas de caracteres de finalización nula que representa los argumentos con los que se inejecuta el proceso y que debe acabar obligatoriamente en un puntero nulo.

\section{\texttt{ntfw}}\label{ntfw}

\begin{lstlisting}[language=C]
#include <ftw.h>

int nftw (const char *dirpath,
          int (*fn) (const char* fpath, const struct stat* sb,
                     int typeflag, struct FTW* ftwbuf),
          int nopenfd, int flags);
\end{lstlisting}

Recorre el árbol de directorios de raíz \texttt{fpath} y llama a la rutina \texttt{fn} una vez por cada entrada.
Por defecto, recorre el árbol en preorden transversal, es decir, primero lee recursivamente los directorios y luego los ficheros.

Para evitar agotar el número de descriptores de ficheros, se establece \texttt{nopenfd} como máximo de ficheros abiertos posibles simultáneamente.
Al rebasar este límite, \texttt{nftw} funciona más lento porque tiene que ir cerrando y abriendo directorios.
Se usa como máximo un descriptor por fichero del árbol.

Para cada entrada del árbol, \texttt{fn} recibe cuatro argumentos:

\begin{itemize}
	\item\texttt{fpath}\textbf{:} Ruta de la entrada expresada relativa o absolutamente de la misma forma que \texttt{dirpath}.
	\item\texttt{sb}\textbf{:} Puntero a la estructura \texttt{stat} devuelta por \texttt{stat(fpath)}.
	\item\texttt{typeflag}\textbf{:} Un entero que tiene uno de los siguientes valores:
	\begin{itemize}
		\item\texttt{FTW\_D}\textbf{:} \texttt{fpath} es un directorio.
		\item\texttt{FTW\_DNR}\textbf{:} \texttt{fpath} es un directorio que no puede leerse.
		\item\texttt{FTW\_DP}\textbf{:} Recorriendo en postorden (\texttt{FTW\_DEPTH}), el contenido de este directorio ya ha sido recorrido.
		\item\texttt{FTW\_F}\textbf{:} \texttt{fpath} es un fichero regular.
		\item\texttt{FTW\_NS}\textbf{:} La llamada a \texttt{stat} ha fallado con este fichero (probablemente por restricciones de permisos), por lo que \texttt{statbuf} tiene un valor indefinido.
		\item\texttt{FTW\_SL}\textbf{:} Desreferenciando enlaces simbólicos (\texttt{FTW\_PHYS}), el fichero es uno.
		\item\texttt{FTW\_SLN}\textbf{:} Sin desreferenciar enlaces simbólicos, el fichero es uno, por lo que se pierde su referencia.
	\end{itemize}
	\item\texttt{ftwbuf}\textbf{:} Puntero a una estructura de tipo \texttt{FTW}.
\end{itemize}

\begin{lstlisting}[language=C]
struct FTW {
	int base,  // Offset del fichero con respecto a pathname
	    level; // Profundidad del fichero en el árbol
};
\end{lstlisting}

Cada vez que se llama a \texttt{fn}, ésta devuelve un valor entero que valdrá \texttt{0} en caso de éxito, y cualquier otro número en caso de error.
Al devolverse un error, \texttt{nftw} deja de recorrer el árbol y devuelve dicho valor.
También devuelve \texttt{-1} si el error es de otro tipo.

Por último, \texttt{flags} indica banderas seleccionables múltiplemente con \texttt{OR}, que pueden ser una de las siguientes:

\begin{itemize}
	\item\texttt{FTW\_DIR}\textbf{:} Realizar un cambio de directorio (\texttt{chdir}) en cada uno antes de procesar su contenido.
	\item\texttt{FTW\_DEPTH}\textbf{:} Realizar un recorrido postorden del árbol, es decir, se llama a \texttt{fn} sobre todos los ficheros y subdirectorios de un directorio antes de llamarla sobre él mismo.
	\item\texttt{FTW\_MOUNT}\textbf{:} No salir del sistema de archivos, es decir, no cruzar puntos de montaje.
	\item\texttt{FTW\_PHYS}\textbf{:} No seguir enlaces simbólicos.
\end{itemize}

\section{\texttt{perror}}

\begin{lstlisting}[language=C]
#include <stdio.h>

void perror (const char* s);
\end{lstlisting}

Imprime un mensaje de error con la cadena de caracteres \texttt{s} por la salida de error estándar.
Por la salud mental de quienes lo lean, se recomienda (porque la logística de obligarlo a la fuerza es demasiado compleja) que este error acabe en \texttt{\textbackslash{}n}.

\begin{lstlisting}[language=C]
perror("Sa matao Paco!\n");
\end{lstlisting}

\section{\texttt{printf}}

\begin{lstlisting}[language=C]
#include <stdio.h>

int printf  (const char* format, ...);
int dprintf (int fd, const char* format, ...);
int sprintf (char *str, const char* format, ...);
\end{lstlisting}

Imprime por salida estándar una cadena de caracteres formada por un formato \texttt{format} y tantos datos como se deseen añadir.
El formato es una cadena de caracteres que puede contener una cadena normal y (entre muchos otros) los siguientes especificadores:

\begin{itemize}
	\item\texttt{\%d}\textbf{:} Se sustituye por un número escrito en base decimal especificado en los argumentos siguientes.
	\item\texttt{\%s}\textbf{:} Se sustituye por una cadena de caracteres especificada en los argumentos siguientes.
	\item\texttt{\textbackslash{}n}\textbf{:} Se sustituye por un salto de línea.
\end{itemize}

Los argumentos a imprimir añadidos tras el formato deben corresponderse en orden y tipo con los especificados en el formato.
En caso de éxito, devuelve el número de caracteres impresos.
En caso de fracaso, devuelve un valor negativo.

\begin{lstlisting}[language=C]
char* saludo = "Hola\0";
int   numero = 10;

// Hola
printf("Hola\n");
// Hola
printf("%s\n", saludo);
// Hola, muy buenas.
printf("%s, muy buenas\n", saludo);
// Hola, muy buenas.
printf("%s%s\n", saludo, "muy buenas");
// En este examen voy a sacar un 10!
printf("En este examen voy a sacar un %d!\n", numero);
// He sacado un 10. Oh! Hola, no te había visto.
printf("He sacado un %d. Oh! %s, no te había visto.\n", numero, saludo);
\end{lstlisting}

La variante \texttt{dprintf} introduce la salida en el fichero identificado por su descriptor \texttt{fd}\footnote{Estoy bastante seguro de que borra todos los datos incluso probando a abrir el fichero con \texttt{O\_APPEND}, pero es muy tarde y no me apetece ponerme a investigar movidas raras. Mucho que podemos escribir en un fichero con \texttt{printf}}.
La variante \texttt{sprintf} introduce la salida en la cadena de caracteres \texttt{str}, permitiendo traducir cualquier dato a cadenas de caracteres.

\section{\texttt{scanf}}

\begin{lstlisting}[language=C]
#include <stdio.h>

int scanf  (const char* format, ...);
\end{lstlisting}

Lee de entrada estándar una cadena de caracteres y la traduce al tipo de dato especificado en el formato \texttt{format}, introduciéndola traducida en las variables pasadas como argumentos.
El formato de lectura es idéntico al de \texttt{printf}.
Las variables a sobreescribir se pasan a los argumentos posteriores al formato por referencia, excepto las cadena de caracteres, que pueden pasarse por valor (ya que apuntan al primer elemento de la misma) y deben estar inicializadas en tiempo de compilación, es decir, declaradas como vectores de caracteres o con un valor.

\begin{lstlisting}[language=C]
int  numero;
char cadena[256];

scanf("%d", &numero);
scanf("%s", &cadena);
scanf("%s", cadena);
\end{lstlisting}

