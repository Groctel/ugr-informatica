\chapter{Rutinas comunes}

\subsection{Ajuste de la posición de flujos de directorios}

Variables utilizadas:

\begin{itemize}
	\item\code{DIR* dir}\textbf{:} Flujo de directorio.
	\item\code{long loc}\textbf{:} Posición sobre el directorio.
\end{itemize}

\begin{lstlisting}[language=C]
loc = telldir(dir);

// Trabajo sobre el directorio

seekdir(dir, loc);
\end{lstlisting}

\subsection{Apertura de directorios}

Variables utilizadas:

\begin{itemize}
	\item\code{DIR* dir}\textbf{:} Flujo de directorio.
\end{itemize}

\begin{lstlisting}[language=C]
if (dir = opendir("Directorio") == NULL) {
	printf("Error %d al intentar abrir el directorio\n", errno);
	perror("Error en opendir\n");
	exit(1);
}
\end{lstlisting}

\subsection{Apertura de ficheros}

Variables utilizadas:

\begin{itemize}
	\item\code{int fd}\textbf{:} Descriptor de fichero.
\end{itemize}

\begin{lstlisting}[language=C]
if ((fd = open("./fichero", O_CREAT|O_WRONLY, S_IRUSR|S_IWUSR)) < 0) {
	printf("Error %d al intentar abrir el fichero\n", errno);
	perror("Error en open\n");
	exit(1);
}
\end{lstlisting}

\pagebreak

\subsection{Borrado de ficheros}

\begin{lstlisting}[language=C]
if (unlink("./fichero") < 0) {
	printf("Error %d al intentar eliminar el fichero\n", errno);
	perror("Error en unlink\n");
	exit(1);
}
\end{lstlisting}

\subsection{Cambio de permisos de ficheros}

Variables utilizadas:

\begin{itemize}
	\item\code{int fd}\textbf{:} Descriptor de fichero.
\end{itemize}

\begin{lstlisting}[language=C]
if (chmod("./fichero", 0750) < 0) {
	printf("Error %d al intentar cambiar los permisos del fichero", errno);
	perror("Error en chmod\n");
	exit(1);
}
\end{lstlisting}

\begin{lstlisting}[language=C]
if (fchmod(fd, 0750) < 0) {
	printf("Error %d al intentar cambiar los permisos del fichero", errno);
	perror("Error en fchmod\n");
	exit(1);
}
\end{lstlisting}

\subsection{Cambio del directorio de trabajo}

Variables utilizadas:

\begin{itemize}
	\item\code{int fd}\textbf{:} Descriptor de fichero.
\end{itemize}

\begin{lstlisting}[language=C]
if (chdir("./directorio") < 0) {
	printf("Error %d al intentar cambiar el directorio de trabajo", errno);
	perror("Error en chdir\n");
	exit(1);
}

\end{lstlisting}
\begin{lstlisting}[language=C]
if (chdir(fd) < 0) {
	printf("Error %d al intentar cambiar el directorio de trabajo", errno);
	perror("Error en chdir\n");
	exit(1);
}
\end{lstlisting}

\pagebreak

\subsection{Cierre de directorios}

Variables utilizadas:

\begin{itemize}
	\item\code{DIR* dir}\textbf{:} Flujo de directorio.
\end{itemize}

\begin{lstlisting}[language=C]
if (closedir(dir) < 0) {
	printf("Error %d al intentar cerrar el directorio\n", errno);
	perror("Error en closedir\n");
	exit(1);
}
\end{lstlisting}

\subsection{Cierre de ficheros}

Variables utilizadas:

\begin{itemize}
	\item\code{int fd}\textbf{:} Descriptor de fichero.
\end{itemize}

\begin{lstlisting}[language=C]
if (close(fd) < 0) {
	printf("Error %d al intentar cerrar el fichero\n", errno);
	perror("Error en close\n");
	exit(1);
}
\end{lstlisting}

\subsection{Creación de cauces con nombre}

\begin{lstlisting}[language=C]
if (mkfifo("./fifo", 0750) < 0) {
	printf("Error %d al intentar crear un fichero fifo", errno);
	perror("Error en mkfifo\n");
	exit(1);
}
\end{lstlisting}

\pagebreak

\subsection{Creación de cauces sin nombre}

Variables utilizadas:

\begin{itemize}
	\item\code{int pid}\textbf{:} Identificador de los procesos padre e hijo.
	\item\code{int pipefd[2]}\textbf{:} Identificadores de los extremos del cauce.
	\item\code{char c}\textbf{:} Carácter que se transmite por el cauce.
\end{itemize}

\begin{lstlisting}[language=C]
pipe(pipefd);

pid = fork();

if (!pid) {
	close(pipefd[0]);
	write(pipefd[1], "a", 1);
}
else {
	close(pipefd[1]);
	read(pipefd[0], c, 1);
	printf("%s\n", c);
}
\end{lstlisting}

\subsection{Creación de proceso hijo}

Variables utilizadas:

\begin{itemize}
	\item\code{int pid}\textbf{:} Identificador de los procesos padre e hijo.
\end{itemize}

\begin{lstlisting}[language=C]
pid = fork();

if (pid == 0)
	// Tarejas ejecutadas por el hijo
else
	// Tareas ejecutadas por el padre
\end{lstlisting}

\subsection{Consulta de la posición de elementos en directorios}

Variables utilizadas:

\begin{itemize}
	\item\code{DIR* dir}\textbf{:} Flujo de directorio.
\end{itemize}

\begin{lstlisting}[language=C]
if (telldir(dir) < 0) {
	printf("Error %d al intentar obtener una posición sobre el directorio", errno);
	perror("Error en telldir\n");
	exit(1);
}
\end{lstlisting}

\pagebreak

\subsection{Ejecución de órdenes externas al proceso}

Variables utilizadas:

\begin{itemize}
	\item\code{char* arg}\textbf{:} Cadena de caracteres que se pasa como argumento.
\end{itemize}

\begin{lstlisting}[language=C]
execlp("yay", "yay", "-S", "minecraft-launcher", NULL);
execlp("./ejecutable", arg, NULL);
\end{lstlisting}

\subsection{Escritura de ficheros}

Variables utilizadas:

\begin{itemize}
	\item\code{char* buf}\textbf{:} Cadena de caracteres a escribir en el fichero.
	\item\code{int fd}\textbf{:} Descriptor de fichero.
	\item\code{int long\_buf}\textbf{:} Longitud de la cadena del  búfer (\code{buf}).
\end{itemize}

\begin{lstlisting}[language=C]
if (write(fd, buf, long_buf) != long_buf1) {
	printf("Error %d al intentar escribir sobre el fichero", errno);
	perror("Error en write\n");
	exit(1);
}
\end{lstlisting}

\subsection{Lectura de datos de entrada estándar}

Variables utilizadas:

\begin{itemize}
	\item\code{int numero}\textbf{:} Número entero a leer.
	\item\code{char cadena[256]}\textbf{:} Cadena de hasta 256 caracteres a leer.
\end{itemize}

\begin{lstlisting}[language=C]
scanf("%d", &numero);
scanf("%s, cadena);
\end{lstlisting}

\subsection{Lectura del contenido de directorios}

Variables utilizadas:

\begin{itemize}
	\item\code{DIR* dir}\textbf{:} Flujo de directorio.
	\item\code{struct dirent* dnt}\textbf{:} Entrada del contenido de un directorio.
\end{itemize}

\begin{lstlisting}[language=C]
while ((dnt = readdir(dir)) != NULL)
	printf("%s\n", dnt->d_name);
\end{lstlisting}

\pagebreak

\subsection{Muestra de variables por salida estándar}

Variables utilizadas:

\begin{itemize}
	\item\code{int numero}\textbf{:} Número entero a mostrar.
	\item\code{char* cadena}\textbf{:} Cadena de caracteres a mostrar.
\end{itemize}

\begin{lstlisting}[language=C]
printf("%s %d\n", cadena, numero);
\end{lstlisting}

\subsection{Posicionamiento sobre ficheros}

Variables utilizadas:

\begin{itemize}
	\item\code{int fd}\textbf{:} Descriptor de fichero.
\end{itemize}

\begin{lstlisting}[language=C]
if (lseek(fd, 40, SEEK_SET) < 0) {
	printf("Error %d al intentar posicionarse sobre sobre el fichero", errno);
	perror("Error en lseek\n");
	exit(1);
}
\end{lstlisting}

\subsection{Vuelta al inicio de directorios}

Variables utilizadas:

\begin{itemize}
	\item\code{DIR* dir}\textbf{:} Flujo de directorio.
\end{itemize}

\begin{lstlisting}[language=C]
rewinddir(dir);
\end{lstlisting}

