\section{Comunicación entre procesos utilizando cauces}

\subsection{Concepto y tipos de cauce}

Los cauces son mecanismos que permiten la comunicación y sincronización entre procesos.
En un mismo cauce puede haber varios procesos lectores y escritores, lo que implica un problema de \emph{productor/consuimdor} en su uso, que hay que resolver para su correcto funcionamiento.
Los datos de un cauce se tratan en orden FIFO y su lectura se traduce en la eliminación de los mismo del cauce, por lo que se consumen únicamente por el primer proceso que los lee.
Si un proceso intenta leer datos de un cauce vacío se bloquea y espera hasta poder leerlos, sincronizándose con el proceso escritor.
Un ejemplo muy común de uso de cauces es el uso del operador \code{|} en las órdenes que mandamos a la shell:

\begin{lstlisting}[language=Bash]
pacman -Q | grep xorg | sed 's/ .*//g'
\end{lstlisting}

En este caso, el resultado de ejecutar \code{pacman -Q}, que se muestra por \code{stdout}, se pasa como entrada a \code{grep xorg}, que muestra únicamente los paquetes que contienen \code{xorg} en el nombre, y esta salida se pasa a \code{sed 's/ .*//g'}, que elimina las versiones de los paquetes y deja únicamente los nombres.
Ambos son cauces sin nombres creados dinámicamente mediante la llamada al sistema \code{pipe} (\S\ref{pipe}).

\subsection{Cauces sin nombre}

Los cauces sin nombre son cauces que no tienen un fichero asociado en el sistema de archivos, sino que existen únicamente en memoria principal.
Al crearse usando la llamada \code{pipe}, se devuelven dos descriptores, uno de lectura y uno de escritura.
Estos cauces sólo pueden utilizarse por el proceso creador y sus hijos y se cierran y eliminan por el kernel cuando tienen exactamente cero productores y consumidores.

\subsubsection{Creación de cauces sin nombre}

Para crear un cauce sin nombre usamos la llamada \code{pipe}, a la que le pasamos un array de dos enteros.
El primer entero es el descriptor de fichero del extremo de lectura del cauce y el segundo, el de escritura.
Creado el cauce, creamos un proceso hijo para que herede los descriptores y establecemos el sentido del flujo en ambas partes cerrando los descriptores que no queremos usar en cada uno de los procesos con la llamada \code{close}.

Para redireccionar la entrada o salida estándar al descriptor de lectura o escritura podemos usar \code{close} para cerrar el cauce y \code{dup2} para realizar la redirección.

\subsection{Cauces con nombre}

Los cauces con nomnbre se crean mediante una llamada a \code{mknod} y \code{mkfifo} como ficheros especiales, por lo que están permanentemente asociados a una ruta en el sistema de archivos.
Al contrario que los cauces sin nombre, éstos sí se abren y cierran llamando a \code{open} y \code{close} y cualquier proceso puede usarlos llamando a \code{read} y \code{write}.

\subsubsection{Creación de ficheros FIFO}

Para crear ficheros fifo usamos la llamada \code{mknod} especificando \code{S\_IFIFO} en OR con los permisos de creación (en octal) en el argumento \code{mode} y \code{0} en el argumento \code{dev}.
Para esta tarea también podemos usar la llamada \code{mkfifo}, que simplemente necesita pasarle como argumentos la ruta de creación del fichero y los permisos de acceso en \code{mode}.

\subsubsection{Uso de cauces FIFO}

Los cauces FIFO se pueden leer con \code{read} de forma bloqueante para el proceso lector siempre en la cabeza de la cola, por lo que no se puede ejecutar \code{lseek} sobre ellos (igualmente, no tiene sentido acceder a otros elementos de la cola).
La llamada \code{read} desbloquea el proceso lector al leer el dato o cuando todos los procesos productores cierran el cauce o terminan, devolviendo \code{0}.
