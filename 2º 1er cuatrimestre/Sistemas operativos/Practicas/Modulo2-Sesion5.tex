\section{Llamadas al sistema para gestión y control de señales}

\subsection{Señales}

Las señales son un mecanismo básico de sincronización que hace uso del kernel de Linux para indicar a los procesos que han ocurrido eventos (síncrono o asíncronos) relacionados con su ejecución.
Los procesos pueden tanto recibir señales del kernel como enviarse señales entre ellos para informar que los receptores deben reaccionar a un evento.

Un \textbf{manejador de señales} es una rutina definida en el programa que es invocada cuando el proceso recibe una señal.
Los manejadores pueden interrumpir el flujo en cualquier momento de forma que, tras acabar su trabajo, el programa continúa ejecutándose por donde iba.
Decimos que una señal es \emph{depositada} cuando el proceso receptor inicia una acción sobre ella y que está \emph{pendiente} cuando ha sido generada pero aún no ha sido depositada.

Un proceso puede bloquear la recepción de una o varias señales a la vez.
Las señales bloqueadas se almacenan en la \textbf{máscara de bloqueo de señales}, que es un conjunto de señales a la espera de depositarse o ignorarse.
Una señal recibida cuando ya está enmascarada es redundante y no se regista en ella.

\subsection{Tipos de señales}

POSIX define las siguientes señales:

\begin{itemize}
\item\texttt{SIGABRT}\textbf{:} Señal de aborto procedente de la llamada \texttt{abort}, que termina el proceso y realiza un volcado de memoria\footnote{\emph{Core dumped}.}.
\item\texttt{SIGALRM}\textbf{:} Señal procedente de la llamada al sistema \texttt{alarm}, que termina el proceso.
\item\texttt{SIGCONT}\textbf{:} Reanuda el proceso si estaba parado.
\item\texttt{SIGFPE}\textbf{:} Excepción de coma flotante, que termina el proceso y realiza un volcado de memoria.
\item\texttt{SIGHUP}\textbf{:} Desconexión de la terminal que termina el proceso. También reanuda los daemons \texttt{init}, \texttt{httpd} e \texttt{inetd}.
\item\texttt{SIGILL}\textbf{:} Excepción producida por la ejecución de una instrucción ilegal, que termina el proceso y realiza un volcado de memoria.
\item\texttt{SIGINT}\textbf{:} Interrupción de teclado \texttt{\^{}C}, que termina el proceso.
\item\texttt{SIGKILL}\textbf{:} Señal para terminar un proceso que no se puede ignorar ni manejar.
\item\texttt{SIGPIPE}\textbf{:} Señal de cauce roto (escritura sin lectores), que finaliza el proceso.
\item\texttt{SIGQUIT}\textbf{:} Terminación procedente del teclado, que termina el proceso y realiza un volcado de memoria.
\item\texttt{SIGSEGV}\textbf{:} Violación de segmento (referencia inválida a memoria), que termina el proceso y realiza un volcado de memoria.
\item\texttt{SIGSTOP}\textbf{:} Señal para parar un proceso que no se puede ignorar ni manejar.
\item\texttt{SIGTERM}\textbf{:} Señal para terminar un proceso que sí puede manejarse.
\item\texttt{SIGTSTP}\textbf{:} Señal para parar la escritura en la tty, que detiene el proceso.
\item\texttt{SIGTTIN}\textbf{:} Señal de entrada de la tty para un proceso de fondo, que detiene el proceso.
\item\texttt{SIGTTOU}\textbf{:} Señal de salida a la tty para un proceso de fondo, que detiene el proceso.
\item\texttt{SIGUSR1}\textbf{:} Señal definida por el usuario, que detiene el proceso.
\item\texttt{SIGUSR2}\textbf{:} Señal definida por el usuario, que detiene el proceso.
\end{itemize}

