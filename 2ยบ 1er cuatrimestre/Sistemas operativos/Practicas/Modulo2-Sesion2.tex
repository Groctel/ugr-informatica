\section{Llamadas al sistema para el sistema de archivos (parte 2)}

\subsection{Llamadas al sistema relacionadas con los permisos de los archivos}

\subsubsection{\texttt{umask}}

Podemos jugar la máscara de creación de ficheros creados por un procesos con la orden \texttt{umask} (\S\ref{umask}).
Este máscara es un valor octal de tres cifras precedidas de \texttt{0} en las que se especifican los permisos \texttt{rwx} para los tres grupos de usuarios.
Por defecto (y por desgracia) se asigna la máscara \texttt{0777}.
Por favor, no uséis \texttt{0777}.

\subsubsection{\texttt{chmod} y \texttt{fchmod}}

Estas llamadas nos permiten cambiar los permisos de acceso de un fichero ya existente.
La llamada \texttt{chmod} (\S\ref{chmod}) nos permite trabajar sobre un fichero especidicado por su ruta mientras que \texttt{fchmod} (\S\ref{fchmod}) lo hace sobre un fichero referenciado por su descriptor en un proceso.
Como máscara se le puede pasar un valor octal o los valores de permisos de \texttt{stat} aparte de \texttt{S\_ISUID}, \texttt{S\_ISGID} y \texttt{S\_IVTX}.

\subsection{Rutinas de manejo de directorios}

Para trabajar con directorios de forma portable (ya que su estructura puede variar entre sistemas) trabajamos con la biblioteca \texttt{dirent.h} (apéndice B).
Esta biblioteca incorpora rutinas que trabajan con las estructuras \texttt{DIR} para referirse a los directorios y \texttt{dirent} para referirse a sus entradas (la constante \texttt{NAME\_MAX} viene definida en el sistema):

\begin{lstlisting}[language=C]
typedef struct _dirdesc {
	int   dd_fd;
	long  dd_loc,
	      dd_size,
	      dd_bbase,
	      dd_entno,
	      dd_bsize;
	char* dd_buf;
} DIR;

struct dirent {
	long d_ino;            // Número de inodo
	char d_name[NAME_MAX]; // Nombre del fichero
};
\end{lstlisting}

El recorrido de sistemas de archivos también puede hacerse mediante \texttt{nftw} (\S\ref{ntfw}).

