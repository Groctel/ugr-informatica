\section{Llamadas al sistema para el sistema de archivos (parte 1)}

A lo largo de este módulo se presupone que se tiene acceso a las bibliotecas \code{libc} y \code{glibc} y conocimientos sobre C y las diferencias que presenta con C++.
Las llamadas al sistema que se discuten de aquí en adelante pueden referenciarse en el manual del sistema con el prefijo \code{2} (más información en \code{man man}).
También cabe destacar el uso de \code{errno} y \code{perror} para devolver errores y la posibilidad de usar \code{strace} para analizar la traza de los programas ejecutados a la hora de depurarlos.

\subsection{Entrada/Salida de archivos regulares}

A la hora de abrir un fichero lo hacemos con \code{open} (\S\ref{open}), que lo crea en caso de que no exista, y lo cerramos con \code{close} (\S\ref{close}).
Podemos leer su contenido con \code{read} (\S\ref{read}) y escribirlo con \code{write}.

Todos los ficheros abiertos por un proceso se identifican mediante un descriptor de fichero (\code{fd}), que es un entero no negativo que lo identifica unívocamente en el fichero en el que se encuentra.
De esta manera, podemos asignar un número a un fichero en función de la función (valga la redundancia) que va a realizar en el programa sin necesitar especificar su ruta cada vez que lo referenciamos.
Por convenio, POSIX incorpora tres descriptores propios de las shell definidos en \code{unistd.h}:

\begin{itemize}
	\item\textbf{0:} Entrada estándar (\code{STDIN\_FILENO}).
	\item\textbf{1:} Salida estándar (\code{STDOUT\_FILENO}).
	\item\textbf{2:} Salida de error estándar (\code{STDERR\_FILENO}).
\end{itemize}

Los ficheros abiertos incorporan una posición de lectura/escritura actual, que es el índice entero no negativo ($[0,n-1]$) del byte del mismo al que se está accediendo, al que llamamos \emph{current file offset}.
Esta posición se define como \code{0} por defecto a menos que especifiquemos \code{O\_APPEND} a la hora de abrir el fichero, en cuyo caso será \code{n-1}.
Podemos buscar una posición exacta del fichero con \code{lseek} (\S\ref{lseek}).

\subsubsection{Trabajo con llamadas de gestión y procesamiento sobre archivos regulares}

A la hora de crear un fichero utilizamos \code{open}, que permite, entre otras, dos flags:

\begin{itemize}
	\item\code{O\_APPEND}\textbf{:} Colocar el offset en el último byte del fichero al abrirlo.
	\item\code{O\_CREAT}\textbf{:} Crear el fichero en caso de que no exista.
\end{itemize}

Estas \code{flags} son realmente enteros que se pueden referenciar por sus nombres de constante del sistema y combinar en un sólo entero mediante el uso del \code{or} lógico.

El \code{mode} (también llamado \code{umask}) nos permite indicar los permisos del fichero a crear si \code{O\_CREAT} está activado.
Podemos definir dichos permisos en su valor octal (precedido de \code{0}) o mediante un \code{or} lógico de los diferentes permisos:

\begin{itemize}
	\item\code{S\_IRWXU}\textbf{:} El usuario tiene los tres permisos sobre el fichero.
	\item\code{S\_IRUSR}\textbf{:} El usuario puede leer el fichero.
	\item\code{S\_IWUSR}\textbf{:} El usuario puede escribir sobre el fichero.
	\item\code{S\_IXUSR}\textbf{:} El usuario puede ejecutar el fichero.
	\item\code{S\_IRWXG}\textbf{:} El grupo tiene los tres permisos sobre el fichero.
	\item\code{S\_IRGRP}\textbf{:} El grupo puede leer el fichero.
	\item\code{S\_IWGRP}\textbf{:} El grupo puede escribir sobre el fichero.
	\item\code{S\_IXGRP}\textbf{:} El grupo puede ejecutar el fichero.
	\item\code{S\_IRWXO}\textbf{:} El resto de usuarios tiene los tres permisos sobre el fichero.
	\item\code{S\_IROTH}\textbf{:} El resto de usuarios puede leer el fichero.
	\item\code{S\_IWOTH}\textbf{:} El resto de usuarios puede escribir sobre el fichero.
	\item\code{S\_IXOTH}\textbf{:} El resto de usuarios puede ejecutar el fichero.
\end{itemize}

\subsection{Metadatos de un fichero}

\subsubsection{Tipos de ficheros}

En Linux podemos diferenciar entre varios tipos de ficheros:

\begin{itemize}
	\item\textbf{Ficheros regulares:} Contiene datos de cualquier tipo sin que el kernel distinga entre éstos. Es responsabilidad del programador interpretarlos correctamente.
	\item\textbf{Ficheros de directorio:} Contiene inodos a otros ficheros. Cualquier proceso con permisos de lectura sobre ellos puede ver su contenido pero sólo el kernel puede modificar su contenido.
	\item\textbf{Ficheros especiales de dispositivo de caracteres:} Representan ciertos tipos de dispositivosen un sistema.
	\item\textbf{Ficheros especiales de disposivito de blogues:} Normalmente representan unidades de discos.
	\item\textbf{Ficheros FIFO (cauces):} Se usan para comunicación entre procesos (\emph{IPC}).
	\item\textbf{Enlaces simbólicos:} Ficheros cuyo inodo apunta a otro fichero en el sistema.
	\item\textbf{Sockets:} Utilizados para comunicación en red entre procesos o entre procesos en un único nodo.
\end{itemize}

El estándar POSIX define las siguientes macros para comprobar el tipo de fichero:

\begin{itemize}
	\item\code{S\_ISREG (st\_mode)}\textbf{:} El fichero es de tipo regular
	\item\code{S\_ISDIR (st\_mode)}\textbf{:} El fichero es un directorio
	\item\code{S\_ISCHR (st\_mode)}\textbf{:} El fichero es un dispositivo de caracteres
	\item\code{S\_ISBLK (st\_mode)}\textbf{:} El fichero es un dispositivo de bloques
	\item\code{S\_ISFIFO (st\_mode)}\textbf{:} El fichero es un cauce con nombre
	\item\code{S\_ISLNK (st\_mode)}\textbf{:} El fichero es un enlace simbólico
	\item\code{S\_ISSOCK (st\_mode)}\textbf{:} El fichero es un nombre
\end{itemize}

\subsubsection{Estructura \code{stat}}

Los metadatos de un fichero se pueden consultar con la orden \code{stat} (\S\ref{stat}), que realiza una llamada al sistema homónima y devuelve los datos de la siguiente estructura:

\begin{lstlisting}[language=C]
struct stat {
	dev_t st_dev;             // Número de dispositivo
	dev_t st_rdev;            // Número de dispositivo (ficheros especiales)
	ino_t st_ino;             // Número de inodo
	mode_t st_mode;           // Tipo de archivo y mode
	nlink_t st_nlink;         // Número de enlaces duros
	uid_t st_uid;             // UID del propietario
	gid_t st_gid;             // GID del propietario
	off_t st_size;            // Tamaño total en bytes (ficheros regulares)
	unsigned long st_blksize; // Tamaño del bloque de E/S para el filesystem
	unsigned long st_blocks;  // Número de bloques (512B) asignados
	time_t st_atime;          // Hora de último acceso
	time_t st_mtime;          // Hora de última modifiación
	time_t st_ctime;          // Hora de último cambio
};
\end{lstlisting}

El tamaño de bloque designado en \code{st\_blksize} puede diferir del tamaño de bloque del sistema de archivos, pues se asigna buscando la mayor eficiencia en las tareas de E/S.
Aunque en ext4 sí ocurre, no todos los sistemas de archivos implementan todos los campos de hora.
Generalmente, \code{st\_atime} se modifica mediante \code{mknod}, \code{utime}, \code{read}, \code{write} y \code{truncate}; \code{st\_mtime} por \code{mknod}, \code{utime} y \code{write}, pero no por las modificaciones en el propietario, grupo, cuenta de enlaces físicos o modo; y \code{st\_ctime} al escribir o modificar la información  del inodo.

Las siguientes flags permiten trabajar con \code{st\_mode}:

\begin{itemize}
	\item\code{S\_IFMT (0170000)} Máscara de bits para los campos de bit del tipo de archivo (no POSIX).
	\item\code{S\_IFSOCK (0140000)}\textbf{:} Socket (no POSIX).
	\item\code{S\_IFLNK (0120000)}\textbf{:} Enlace simbólico (no POSIX).
	\item\code{S\_IFREG (0100000)}\textbf{:} Archivo regular (no POSIX).
	\item\code{S\_IFBLK (0060000)}\textbf{:} Dispositivo de bloques (no POSIX).
	\item\code{S\_IFDIR (0040000)}\textbf{:} Directorio (no POSIX).
	\item\code{S\_IFCHR (0020000)}\textbf{:} Dispositivo de caracteres (no POSIX).
	\item\code{S\_IFIFO (0010000)}\textbf{:} Cauce con nombre (FIFO) (no POSIX).
	\item\code{S\_ISUID (0004000)}\textbf{:} Bit SUID\@.
	\item\code{S\_ISGID (0002000)}\textbf{:} Bit SGID\@.
	\item\code{S\_ISVTX (0001000)}\textbf{:} Sticky bit (no POSIX).
	\item\code{S\_IRWXU (0000700)}\textbf{:} User (propietario del archivo) tiene permisos de lectura, escritura y  ejecución.
	\item\code{S\_IRUSR (0000400)}\textbf{:} User tiene permiso de lectura (igual que S\_IREAD, no POSIX).
	\item\code{S\_IWUSR (0000200)}\textbf{:} User tiene permiso de escritura (igual que S\_IWRITE, no POSIX).
	\item\code{S\_IXUSR (0000100)}\textbf{:} user tiene permiso de ejecución (igual que S\_IEXEC, no POSIX).
	\item\code{S\_IRWXG (0000070)}\textbf{:} Group tiene permisos de lectura, escritura y ejecución.
	\item\code{S\_IRGRP (0000040)}\textbf{:} Group tiene permiso de lectura.
	\item\code{S\_IWGRP (0000020)}\textbf{:} Group tiene permiso de escritura.
	\item\code{S\_IXGRP (0000010)}\textbf{:} Group tiene permiso de ejecución.
	\item\code{S\_IRWXO (0000007)}\textbf{:} Other tienen permisos de lectura, escritura y ejecución.
	\item\code{S\_IROTH (0000004)}\textbf{:} Other tienen permiso de lectura.
	\item\code{S\_IWOTH (0000002)}\textbf{:} Other tienen permiso de escritura.
	\item\code{S\_IXOTH (0000001)}\textbf{:} Other tienen permiso de ejecución.
\end{itemize}

\subsubsection{Permisos de acceso a archivos}

Los permisos de lectura, escritura y ejecución se utilizan de forma diferente según la llamada al sistema que los interprete.
Las más relevantes son las siguientes:

\begin{itemize}
	\item\textbf{Lectura:} Determina si podemos abrir un fichero y leer su contenido. Está relacionado con los flags \code{O\_RDONLY} y \code{O\_RDWR} de la llamada \code{open}.
	\item\textbf{Escritura:} Determina si podemos abrir un fichero y modificar su contenido. Está relacionado con los flags \code{O\_WRONLY} y \code{O\_RDWR} de la llamada \code{open}.
	\begin{itemize}
		\item Necesitamos tener permisos de escritura en un fichero para poder especifiar el flag \code{O\_TRUNC} en la llamada \code{open}.
	\end{itemize}
	\item\textbf{Lectura y ejecución de directorios:} Los directorios con permiso de lectura permiten leer su contenido, pero sólo los ejecutables permiten pasar a través de ellos. Al acceder a un fichero regular necesitamos permiso de ejecución en cada directorio de su pathname. Por esto, llamamos bit de búsqueda al bit de permiso de ejecución para directorios.
	\begin{itemize}
		\item Para añadir un fichero a un directorio necesitamos permisos de escritura y ejecución sobre el mismo.
		\item Para eliminar un fichero necesitamos permisos de escritura y ejecución sobre el directorio, pero no de lectura o escribura sobre el fichero.
	\end{itemize}
\end{itemize}

