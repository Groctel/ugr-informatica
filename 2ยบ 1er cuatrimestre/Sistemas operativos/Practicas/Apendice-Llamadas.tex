\chapter{Llamadas al sistema}

\section{\texttt{chmod}}\label{chmod}

\begin{lstlisting}[language=C]
#include <sys/stat.h>

int chmod  (const char* pathname, mode_t mode);
int fchmod (int fd, mode_t mode);
\end{lstlisting}

Cambia los permisos de acceso del fichero especificado por \texttt{pathname} a los especificado en la máscara \texttt{mode}.
En caso de éxito, devuelve \texttt{0}.
En caso de fracaso, devuelve \texttt{-1} acompañado por el \texttt{errno} apropiado.

La variante \texttt{fchmod} identifica el fichero por su descriptor \texttt{fd}.

\section{\texttt{chdir}}\label{chdir}

\begin{lstlisting}[language=C]
#include <unistd.h>

int chdir  (const char* path);
int fchdir (int fd);
\end{lstlisting}

Cambia el directorio de trabajo al especificado por la ruta \texttt{path}.
En caso de éxito, devuelve \texttt{0}.
En caso de fracaso, devuelve \texttt{-1} acompañado por el \texttt{errno} apropiado.

La variante \texttt{fstat} identifica el directorio por su descriptor \texttt{fd}.

\section{\texttt{clone}}\label{clone}

\begin{lstlisting}[language=C]
#define _GNU_SOURCE
#include <sched.h>

int clone (int (*fn)(void*), void* stack, int flags, void* arg, ...
           /* pid_t* parent_tid, void* tls, pid_t* child_tid */);
\end{lstlisting}

Crea un proceso hijo de manera análoga a \texttt{fork}, con más control sobre la creación.
Este hijo ejecuta la rutina \texttt{arg} y finaliza cuando ésta lo hace o cuando se reciba un \texttt{\_exit}.
En caso de éxito, devuelve el \texttt{PID} del hijo.
En caso de fracaso, devuelve \texttt{-1} acompañado por el \texttt{errno} apropiado.

\pagebreak

\section{\texttt{close}}\label{close}

\begin{lstlisting}[language=C]
#include <unistd.h>

int close (int fd);
\end{lstlisting}

Cierra el fichero especificado por su descriptor \texttt{fd}, eliminando a su vez todos los cerrojos asociados.
En caso de éxito, devuelve \texttt{0}.
En caso de fracaso, devuelve \texttt{-1} acompañado por el \texttt{errno} apropiado.

\section{\texttt{dup}}\label{dup}

\begin{lstlisting}[language=C]
#include <unistd.h>

int dup  (int oldfd);
int dup2 (int oldfd, int newfd);
\end{lstlisting}

Crea una copia del descriptor de fichero \texttt{oldfd} usando el descriptor de fichero no usado de orden menor.
En caso de éxito, devuelve \texttt{0}.
En caso de fracaso, devuelve \texttt{-1} acompañado por el \texttt{errno} apropiado.

La variante dup2 permite seleccionar el nuevo descriptor de fichero como \texttt{newfd}.

\section{\texttt{fcntl}}\label{fcntl}

\begin{lstlisting}[language=C]
#include <unistd.h>
#include <fcntl.h>

int fcntl (int fd, int cmd, ... /* arg */);
\end{lstlisting}

Manipula el fichero identificado por su descriptor \texttt{fd} según la acción \texttt{cmd} determinada.
Estas acciones vienen detalladas en \S2.6.
La presencia del tercer argumento \texttt{arg} viene determinada por las especificaciones de cada \texttt{cmd}.
En caso de éxito, devuelve un valor distinto en función del \texttt{cmd}:

\begin{itemize}
	\item\texttt{F\_DUPFD}\textbf{:} El nuevo descriptor de fichero.
	\item\texttt{F\_GETFD}\textbf{:} Las banderas del descriptor de fichero.
	\item\texttt{F\_GETFL}\textbf{:} Las banderas de estado del fichero.
	\item\texttt{F\_GETOWN}\textbf{:} El propietario del descriptor de fichero.
	\item\texttt{F\_GETPIPE\_SZ} \textbf{y} \texttt{F\_SETPIPE\_SZ}\textbf{:} La capacidad del cauce.
	\item\textbf{Cualquier otro} \texttt{cmd}\textbf{:} \texttt{0}.
\end{itemize}

En caso de fracaso, devuelve \texttt{-1} acompañado por el \texttt{errno} apropiado.

\pagebreak

\section{\texttt{fork}}\label{fork}

\begin{lstlisting}[language=C]
#include <sys/types.h>
#include <unistd.h>

pid_t fork ();
\end{lstlisting}

Crea un proceso hijo duplicando el proceso invocante o proceso padre.
En caso de éxito, devuelve el \texttt{PID} del hijo en el padre y \texttt{0} en el hijo.
En caso de fracaso, devuelve \texttt{-1} en el padre acompañado por el \texttt{errno} apropiado y sin crear ningún hijo.

\section{\texttt{kill}}\label{kill}

\begin{lstlisting}[language=C]
#include <sys/types.h>
#include <signal.h>

int kill (pid_t pid, int sig);
\end{lstlisting}

Envía una señal \texttt{sig} a un proceso de identificador \texttt{pid}, que puede pertenecer a cuatro categorías:

\begin{itemize}
	\item\texttt{< -1}\textbf{:} Cualquier proceso cuyo \texttt{GID} sea igual al valor absoluto de \texttt{pid}.
	\item\texttt{-1}\textbf{:} Cualquier proceso sobre el que el invocante tiene permiso de envío de señales exceptuando \texttt{init}.
	\item\texttt{0}\textbf{:} Cualquier proceso cuyo \texttt{GID} sea igual al del invocante en el momento de llamar a \texttt{kill}.
	\item\texttt{> 0}\textbf{:} El proceso cuyo \texttt{PID} sea igual al valor de \texttt{pid}.
\end{itemize}

\section{\texttt{lskeek}}\label{lseek}

\begin{lstlisting}[language=C]
#include <sys/types.h>
#include <unistd.h>

off_t lseek (int fd, off_t offset, int whence);
\end{lstlisting}

Reposiciona el offset del fichero especificado por su descriptor \texttt{fd} a \texttt{offter} de acuerdo a la directiva \texttt{whence}, que puede ser:

\begin{itemize}
	\item\texttt{SEEK\_END}\textbf{:} Mover el offset a \texttt{offset} bytes posteriores al final del fichero.
	\item\texttt{SEEK\_CUR}\textbf{:} Mover el offset a \texttt{offset} bytes posteriores al actual.
	\item\texttt{SEEK\_SET}\textbf{:} Mover el offset a \texttt{offset} bytes.
\end{itemize}

\section{\texttt{mkfifo}}\label{mkfifo}

\begin{lstlisting}[language=C]
#include <sys/types.h>
#include <sys/stat.h>

int mkfifo(const char* pathname, mode_t mode);
\end{lstlisting}

Crea un cauce con nombre en \texttt{pathname} con permisos de acceso \texttt{mode}.
En caso de éxito, devuelve \texttt{0}.
En caso de fracaso, devuelve \texttt{-1} en el padre acompañado por el \texttt{errno}.

\section{\texttt{mknod}}\label{mknod}

\begin{lstlisting}[language=C]
#include <sys/types.h>
#include <sys/stat.h>
#include <fcntl.h>
#include <unistd.h>

int mknod(const char* pathname, mode_t mode, dev_t dev);
\end{lstlisting}

Crea un nodo del sistema de archivos en \texttt{pathname} de tipo \texttt{mode} y \emph{mayor} y \emph{minor} \texttt{dev} en caso de que el nodo sea un dispositivo especial de caracteres o bloques (por defecto es \texttt{0}).
El argumento \texttt{mode} puede ser uno de los siguientes:

\begin{itemize}
	\item\texttt{S\_IFBLK}\textbf{:} Fichero especial de bloques.
	\item\texttt{S\_IFCHR}\textbf{:} Fichero especial de caracteres.
	\item\texttt{S\_IFIFO}\textbf{:} Fichero FIFO (cauce).
	\item\texttt{S\_IFREG}\textbf{:} Fichero regular.
	\item\texttt{S\_IFSOC}\textbf{:} Fichero de tipo socket.
\end{itemize}

En caso de éxito, devuelve \texttt{0}.
En caso de fracaso, devuelve \texttt{-1} en el padre acompañado por el \texttt{errno} apropiado.

\section{\texttt{open}}\label{open}

\begin{lstlisting}[language=C]
#include <sys/types.h>
#include <sys/stat.h>
#include <fcntl.h>

int open  (const char* pathname, int flags);
int open  (const char* pathname, int flags, mode_t mode);
int creat (const char* pathname, mode_t mode);
\end{lstlisting}

Abre el fichero especificado por \texttt{pathname}.
Si no existe y se ha especificado \texttt{O\_CREAT} como \texttt{flag}, llama a \texttt{creat} para crearlo pasándole sus propios argumentos.
Devuelve el descriptor del fichero abierto.

\section{\texttt{pipe}}\label{pipe}

\begin{lstlisting}[language=C]
#include <unistd.h>

int pipe(int pipefd[2]);
\end{lstlisting}

Crea un cauce sin nombre.
El primer elemento de \texttt{pipefd} es el proceso lector y el segundo, el escritor.
En caso de éxito, devuelve \texttt{0}.
En caso de fracaso, devuelve \texttt{-1} acompañado por el \texttt{errno} apropiado y no modifica \texttt{pipefd}.

\pagebreak

\section{\texttt{read}}\label{read}

\begin{lstlisting}[language=C]
#include <unistd.h>

ssize_t read (int fd, void* buf, size_t count);
\end{lstlisting}

Lee los \texttt{count} primeros bytes de un fichero especificado por su descriptor \texttt{fs} y los almacena en el búfer \texttt{buf}.
En caso de éxito, devuelve el número de bytes leídos o \texttt{0} si se ha llegado el final del fichero, tras el cual nunca sigue leyendo.
En caso de fracaso, devuelve \texttt{-1} acompañado por el \texttt{errno} apropiado.

\section{\texttt{sigaction}}\label{sigaction}

\begin{lstlisting}[language=C]
#include <signal.h>

struct sigaction {
	void     (*sa_handler)(int);
	void     (*sa_sigaction)(int, siginfo_t*, void*);
	sigset_t   sa_mask;
	int        sa_flags;
	void     (*sa_restorer)(void);
};

int sigaction (int signum, const struct sigaction* act,
               struct sigaction* oldact);
\end{lstlisting}

Determina la acción \texttt{act} realizada por un proceso al recibir la señal \texttt{signum} (a excepción de \texttt{SIGKILL} y \texttt{SIGSTOP}).
Si \texttt{oldact} es una estructura no nula, se guarda la acción sobreescrita en ésta.
En caso de éxito, devuelve \texttt{0}.
En caso de fracaso, devuelve \texttt{-1} acompañado por el \texttt{errno} apropiado.

\section{\texttt{sigpending}}\label{sigpending}

\begin{lstlisting}[language=C]
#include <signal.h>

int sigpending(sigset_t* set);
\end{lstlisting}

Examina el número de señales pendientes de recepción por parte de la hebra invocante, escribiéndolas en \texttt{set}.
En caso de éxito, devuelve \texttt{0}.
En caso de fracaso, devuelve \texttt{-1} acompañado por el \texttt{errno} apropiado.

\section{\texttt{sigprocmask}}\label{sigprocmask}

\begin{lstlisting}[language=C]
#include <signal.h>

int sigprocmask (int how, const sigset_t* set, sigset_t* oldset);
\end{lstlisting}

Examina o cambia la máscara de señal de la hebra invocante en función del valor de \texttt{how}, que puede ser uno de los siguientes:

\begin{itemize}
	\item\texttt{SIG\_BLOCK}\textbf{:} Se bloquean las señales actuales más las de \texttt{set}.
	\item\texttt{SIG\_SETMASK}\textbf{:} Se desbloquean las señales de \texttt{set}.
	\item\texttt{SIG\_UNBLOCK}\textbf{:} Se cambian las señales bloqueadas por las de \texttt{set}.
\end{itemize}

Por su parte, \texttt{set} es un puntero a un conjunto de señales enmascaradas.
Si es un puntero nulo, \texttt{sigprocmask} se utiliza como consulta.
Por último, \texttt{oldset} representa el conjunto de señales enmascaradas actualmente.
Si no es un puntero nulo, el valor anterior de la máscara de señal se guarda en él.
En caso de éxito, devuelve \texttt{0}.
En caso de fracaso, devuelve \texttt{-1} acompañado por el \texttt{errno} apropiado.

\section{\texttt{sigsuspend}}\label{sigsuspend}

\begin{lstlisting}[language=C]
#include <signal.h>

int sigsuspend(const sigset_t* mask);
\end{lstlisting}

Reemplaza temporalmente la máscara de señal de la hebra invocante por la máscara \texttt{mask} y suspende la ejecución de la hebra hasta la recepción de una señal cuya acción sea invocar al receptor de señales o finalizar el proceso.
Siempre devuelve \texttt{-1} acompañado por el \texttt{errno} apropiado, a menos que la señal finalize el proceso, en cuyo caso no devuelve ningún valor.

\section{\texttt{stat}}\label{stat}

\begin{lstlisting}[language=C]
#include <sys/types.h>
#include <sys/stat.h>
#include <unistd.h>

int stat  (const char* pathname, struct stat* statbuf);
int fstat (int fd, struct stat* statbuf);
int lstat (const char* pathname, struct stat* statbuf);
\end{lstlisting}

Escribe los metadatos del fichero identificado por su \texttt{pathname} en la estructura \texttt{statbuf}.
No requiere permisos en el fichero de trabajo, pero sí en todos sus directorios ancestros.
En caso de éxito, devuelve \texttt{0}.
En caso de fracaso, devuelve \texttt{-1} acompañado por el \texttt{errno} apropiado.

La variante \texttt{fstat} identifica el fichero por su descriptor \texttt{fd}.
La variante \texttt{lstat} es idéntica a \texttt{stat}, pero si se le pasa un enlace simbólico devuelve su información y no la del fichero al que apunta.

\section{\texttt{umask}}\label{umask}

\begin{lstlisting}[language=C]
#include <sys/types.h>
#include <sys/stat.h>

mode_t umask (mode_t mask);
\end{lstlisting}

Cambia la máscara de creación de ficheros por defecto del proceso que ejecuta la llamada a \texttt{mask}.
Devuelve que había antes de ser cambiada.

\pagebreak

\section{\texttt{unlink}}\label{unlink}

\begin{lstlisting}[language=C]
#include <unistd.h>

int unlink(const char* pathname);
\end{lstlisting}

Elimina el fichero de ruta \texttt{pathname}.
En caso de éxito, devuelve \texttt{0}.
En caso de fracaso, devuelve \texttt{-1} acompañado por el \texttt{errno} apropiado.

\section{\texttt{wait}}\label{wait}

\begin{lstlisting}[language=C]
#include <sys/types.h>
#include <sys/wait.h>

pid_t wait    (int* wstatus);
pid_t waitpid (pid_t pid, int* wstatus, int options);
\end{lstlisting}

Suspende la ejecución del proceso invocante hasta que uno de sus hijos termine de ejecutarse y guarda información sobre el estado del mismo en el entero al que apunta \texttt{wstatus}.
En caso de éxito, devuelve el \texttt{PID} del hijo finalizado.
En caso de fracaso, devuelve \texttt{-1} acompañado por el \texttt{errno} apropiado.

La variante \texttt{waitpid} permite especificar el hijo en el argumento \texttt{pid}, que puede pertenecer a cuatro categorías:

\begin{itemize}
	\item\texttt{< -1}\textbf{:} Cualquier hijo cuyo \texttt{GID} sea igual al valor absoluto de \texttt{pid}.
	\item\texttt{-1}\textbf{:} Cualquier hijo.
	\item\texttt{0}\textbf{:} Cualquier hijo cuyo \texttt{GID} sea igual al del invocante en el momento de llamar a \texttt{waitpid}.
	\item\texttt{> 0}\textbf{:} El hijo cuyo \texttt{PID} sea igual al valor de \texttt{pid}.
\end{itemize}

Esta variante también acepta tres opciones seleccionables con \texttt{OR} en \texttt{options}:

\begin{itemize}
	\item\texttt{WCONTINUED}\textbf{:} Devolver valor adicionalmente si un hijo detenido ha sido continuado por \texttt{SIGCONT}.
	\item\texttt{WNOHANG}\textbf{:} Devolver valor inmediatamente si ningún hijo ha finalizado. Devuelve \texttt{0} si el hijo \texttt{pid} existe pero no se ha cambiado su estado.
	\item\texttt{WUNTRACED}\textbf{:} Devolver valor adicionalmente si un hijo ha sido detenido.
\end{itemize}

\section{\texttt{write}}\label{write}

\begin{lstlisting}[language=C]
#include <unistd.h>

ssize_t write (int fd, const void* buf, size_t count);
\end{lstlisting}

Escribe los \texttt{count} primeros bytes de un búfer \texttt{buf} en un fichero especificado por su descriptor \texttt{fd}.
En caso de éxito, se devuelve el número de bytes escritos.
En caso de fracaso, devuelve \texttt{-1} acompañado por el \texttt{errno} apropiado.
