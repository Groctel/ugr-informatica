\section{Herramientas de administración del sistema de archivos}

\subsection{Partición de dispositivos de almacenamiento secundario}

Para poder utilizar una unidad de memoria masiva es necesario dividirla en secciones (al menos una) llamadas \textbf{particiones}, que permiten alojar un sistema de archivos y constituyen un \textbf{disco lógico} (siendo el físico la propia unidad de memoria).
Las particiones están formadas por un sectores de 512, 1024, 2048 (u otros múltiplos de 2) bytes, que son las unidades mínimas de almacenamiento de un sistema de archivos.
Al crear una partición se le asocia una etiqueta en función del sistema de archivos que se quiere alojar en ellas.
Estas pueden consultarse mediante la orden \code{sfdisk -T}.

Para identificar las particiones de un disco, éste incorpora al inicio una tabla de particiones.
Antiguamente, ésta seguía el formato MBR (\emph{master boot record}), pero los sistemas modernos siguen el formato GPT (\emph{GUID partition table}).

\subsubsection{Particiones recomendadas en un sistema}

Un disco en el que se almacene un sistema debe contar, como mínimo, con una partición de arranque (\code{/boot}) y una partición primaria en la que se encuentre la raíz del sistema de archivos.
La partición primaria puede dividirse en varias particiones separadas, como una partición específica para \code{/home}, que permite reinstalar el sistema sin tener que crear una copia de seguridad de todos los ficheros de usuario o utilizar un disco SSD únicamente para la raíz del sistema.
También se pueden añadir particiones \code{swap}, que permite almacenar datos que no caben en la RAM o guardar toda la RAM al hibernar el sistema.

Los discos de almacenamiento secundario pueden contener tantas particiones como se desee sin más restricciones en su tipo o formato que el propósito de los mismos.

Las particiones de un sistema pueden consultarse mediante la orden \code{lsblk}:

\begin{lstlisting}[language=Bash]
lsblk
# NAME        MAJ:MIN RM   SIZE RO TYPE MOUNTPOINT
# sda           8:0    0 931.5G  0 disk
#   sda1        8:1    0  1023M  0 part /boot
#   sda2        8:2    0 922.5G  0 part /home
#   sda3        8:3    0     8G  0 part [SWAP]
# nvme0n1     259:0    0 238.5G  0 disk
#   nvme0n1p1 259:1    0 238.5G  0 part /
\end{lstlisting}

\subsubsection{Partición paso a paso de un dispositivo}

Los ficheros \code{/dev/loop?} nos permiten simular dispositivos de almacenamiento virtual del espacio de los mismos.
Aquí mostramos paso a paso cómo crear un disco \code{loop} y montarlo en el sistema.
Todos estos pasos deben realizarse desde \code{root}.

Primero vamos a crear dos ficheros \code{loop} mediante la orden \code{mknod}, que recibe los siguientes argumentos:

\begin{itemize}
	\item El nombre del fichero a crear
	\item El tipo de fichero a crear (\code{b} simboliza un fichero de bloques especial con búfer).
	\item Identificador del tipo de hardware de la partición (\emph{major}).
	\item Identificador de la instancia de del \code{major} (\code{minor}).
\end{itemize}

\begin{lstlisting}[language=Bash]
mknod /dev/loop0 b 7 0
mknod /dev/loop1 b 7 1
\end{lstlisting}

Hecho esto, vamos a crear dos ficheros de 20 y 30 MB, respectivamente, mediante la orden \code{dd}, que recibe los siguientes argumentos:

\begin{itemize}
	\item Fichero de entrada de datos (\code{if}).
	\item Fichero de salida de datos (\code{of}).
	\item Número de bytes a leer en bloque (\code{bs}).
	\item Número de bloques a copiar (\code{count}).
\end{itemize}

\begin{lstlisting}[language=Bash]
dd if=/dev/zero of=./fichero_SA20 bs=2k count=10000
dd if=/dev/zero of=./fichero_SA30 bs=3k count=10000
\end{lstlisting}

Creados los ficheros, vamos a asociar a cada dispositivo \code{loop} su propio fichero de sistema de archivos mediante la orden \code{losetup}:

\begin{lstlisting}[language=Bash]
losetup /dev/loop0 ./fichero_SA20
losetup /dev/loop1 ./fichero_SA30
\end{lstlisting}

Usando \code{fdisk} o \code{gdisk} podemos leer el estado de las particiones:

\begin{lstlisting}[language=Bash]
fdisk -l /dev/loop0
# Disk /dev/loop0: 19.54 MiB, 20480000 bytes, 40000 sectors
# Units: sectors of 1 * 512 = 512 bytes
# Sector size (logical/physical): 512 bytes / 512 bytes
# I/O size (minimum/optimal): 512 bytes / 512 bytes
gdisk /dev/loop0
# GPT fdisk (gdisk) version 1.0.4
#
# Partition table scan:
#   MBR: not present
#   BSD: not present
#   APM: not present
#   GPT: not present
#
# Creating new GPT entries in memory.
\end{lstlisting}

Para particionar un disco externo se recomienda usar \code{gdisk}, que crea una tabla de particiones GPT en caso de que no exista y permite modificar las particiones navegando por menús sencillos.

\subsection{Asignación de un sistema de archivos a una partición (formateo lógico)}

Ya creadas las particiones de un disco, se debe escoger el sistema de archivos que va a utilizar cada una.
El sistema utilizado por los sistemas Linux actuales es \code{ext4}, una evolución de \code{ext3} y \code{ext2} que incorpora un diario (\emph{journal}) con información de recuperación de los ficheros, extensiones que permiten describir conjuntos de bloques de discos contiguos para mejorar el rendimiento al trabajar con ficheros grandes y reducir la fragmentación del disco y asigación retardada de espacio en disco, que permite postergr la asignación de bloques de disco hasta el momento inminente de la escritura.

\pagebreak

Podemos crear particiones mediante la orden \code{mkfs} seguida de un punto y el tipo de partición que queremos asignar:

\begin{lstlisting}[language=Bash]
mkfs.ext3 /dev/loop0
mkfs.ext4 /dev/loop1
\end{lstlisting}

\subsection{Ajuste de parámetros configurables de un sistema de archivos y comprobación de errores}

Una vez disponibles los sistemas de archivos podemos usar la orden \code{tune2fs} para ajustar los parámetros de los sistemas de archivos \code{ext[234]}.
La opción \code{-l} nos permite obtener información sobre el sistema de archivos pasado como segundo argumento, \code{-c} establece el segundo argumento como el número máximo de montajes que se pueden realizar sin llevar a cabo una comprobación de consistencia del sistema de archivos pasado como tercer argumento y \code{-L} pone al sistema de archivos pasado como tercer argumento la etiqueta pasada como segundo argumento.

Con el tiempo, los sistemas de archivos pueden corromperse, empeorando su rendimiento y causando posibles pérdidas de información.
Para subsanar esto, Linux comprueba automáticamente la salud de los sistemas en función del número de montajes de los mismos.
Para realizar esta tarea manualmente usamos la orden \code{fsck}, que comprueba los metadatos del sistema de archivos (no su contenido explícito) para buscar inconsistencias en el mismo que pueden ser, por ejemplo, las siguientes:

\begin{itemize}
	\item Bloques asignados a varios ficheros simultáneamente.
	\item Bloques marcados como libres asignados a ficheros.
	\item Bloques marcados como asignados que realmente están libres.
	\item Inconsistencia del número de enlaces a un fichero.
	\item Inodos marcados como ocupados que no están asociados a ningún fichero.
\end{itemize}

\subsection{Montaje y desmontaje de sistemas de archivos}

Los sistemas de archivos deben montarse en el sistema para acceder a ellos.
En Linux, los sistemas de archivos pueden montarse en cualquier directorio, que denominamos \textbf{punto de montaje}.
Para montarlos, usamos la orden \code{mount} pasando el sistema de archivos como primer argumento y el punto de montaje como segundo argumento.
Para desmontarlos, usamos la orden \code{umount} pasando el punto de montaje como argumento.
Debemos tener en cuenta que no podemos desmontar un sistema de archivos si está siendo utilizado.

Para montar un sistema de archivos de forma automática debemos incluirlo en \code{/etc/fstab} con los argumentos indicados en \S1.1.3 o usando la orden \code{genfstab}.

\subsection{Administración de software}

En los sistemas Linux, el software no se instala descargando un ejecutable de una página web (que podría ser maliciosa) e instalando su contenido en el ordenador.
Éste es un proceso ortopédico y atrasado, ya que no garantiza la seguridad o autenticidad del software descargado más allá de la ilusión de confianza y obliga al desarrollador a incorporar su propio sistema de actualizaciones y gestión de dependencias.\footnote{Por supuesto, estamos hablando de Windows.}

Los sistemas Linux estructuran el software en paquetes que se instalan una única vez en el sistema y cuentan con un árbol de dependencias de otros paquetes.
Estos paquetes pueden contener binarios o código fuente que debe ser compilado para instalarlo.
Las labores de administración de software en Linux se realizan mediante \textbf{gestores de paquetes}, que son interfaces gráficas o textuales que permiten instalar, desinstalar y administrar programas.
En esta sección mostramos cómo usar algunos de ellos.

\subsubsection{\code{apt}}

Es el gestor de paquetes de Debian y sus distribuciones derivadas.
Permite instalar ficheros \code{.deb} y \code{.rpm} y acepta los siguientes argumentos:

\begin{itemize}
	\item\code{autoremove}\textbf{:} Elimina todas las dependencias huérfanas.
	\item\code{install}\textbf{:} Instala el paquete pasado como segundo argumento.
	\item\code{search}\textbf{:} Busca el paquete pasado como segundo argumento en los repositorios instalados.
	\item\code{update}\textbf{:} Actualiza la base de datos de los repositorios instalados.
	\item\code{upgrade}\textbf{:} Actualiza los paquete cuya versión es inferior a la disponible en los repositorios instalados.
	\item\code{remove}\textbf{:} Elimina el paquete pasado como segundo argumento.
\end{itemize}

\subsubsection{\code{pacman}}

Es el gestor de paquetes de ArchLinux y sus distribuciones derivadas.
Acepta los siguientes argumentos.

\begin{itemize}
	\item\code{-c}\textbf{:} Elimina todas las dependencias huérfanas.
	\item\code{-Q}\textbf{:} Muestra todos los paquetes instalados o el pasado como segundo argumento si lo está. Puede encauzarse a \code{grep} para buscar paquetes que coincidan con una expresión regular.
	\item\code{-R}\textbf{:} Elimina el paquete pasado como segundo argumento.
	\item\code{-S}\textbf{:} Instala el paquete pasado como argumento.
	\item\code{-Syu}\textbf{:} Actualiza los repositorios cuya versión local es inferior a la global y actualiza el sistema.
	\item\code{-Syyu}\textbf{:} Actualiza todos los repositorios y actualiza el sistema.
	\item\code{-U}\textbf{:} Instala un paquete desde la ruta pasada como segundo argumento.
\end{itemize}

\subsubsection{\code{rpm}}

Es un gestor de paquetes de bajo nivel propio de RedHat.
Acepta los siguientes argumentos.

\begin{itemize}
	\item\code{-e}\textbf{:} Elimina el paquete pasado como segundo argumento.
	\item\code{-F}\textbf{:} Actualiza paquetes descargándolos del servidor pasado como segundo argumento.
	\item\code{-i}\textbf{:} Instala el paquete pasado como segundo argumento.
	\item\code{-qa}\textbf{:} Muestra información sobre todos los paquetes instalados. Puede encauzarse con \code{grep} para buscar paquetes que coincidan con una expresión regular.
	\item\code{-qi}\textbf{:} Muestra información sobre el paquete instalado pasado como segundo argumento.
	\item\code{-U}\textbf{:} Actualiza el paquete pasado como segundo argumento.
	\item\code{-V}\textbf{:} Verifica la integridad de la instalación del paqeute pasado como segundo argumento.
\end{itemize}

\subsubsection{\code{yay}}

No es un gestor de paquetes como tal, sino un cliente para buscar paquetes en el AUR (\emph{Arch User Repository}) que incorpora todas las funcionalidades de \code{pacman}.
Como adición a \code{pacman}, permite buscar paquetes en los repositorios pasando una cadena de caracteres como argumento y ejecuta \code{yay -Syu} al ser llamado sin argumentos.

\subsubsection{\code{yum}}

Es un gestor de paquetes usado para instalar ficheros \code{.rpm}, propios de RedHat.
Acepta los siguientes argumentos:

\begin{itemize}
	\item\code{install}\textbf{:} Instala el paquete pasado como segundo argumento.
	\item\code{list}\textbf{:} Muestra los paquetes disponibles en los repositorios.
	\item\code{list installed}\textbf{:} Muestra los paquetes instalados en el sistema.
	\item\code{list updates}\textbf{:} Muestra los paquetes instalados que pueden ser actualizados.
	\item\code{remove}\textbf{:} Elimina el paquete pasado como segundo argumento y todos aquellos que dependen de él.
	\item\code{update}\textbf{:} Actualiza todos los paquetes instalados.
\end{itemize}

\subsection{Administración de cuotas}

Las cuotas de disco son restricciones del uso de los sistemas de archivos que pueden realizar los usuarios de un sistema que requieren del paquete \code{quota}.
Estas restricciones sobre el uso de bloques e inodos pueden establecerse mediante dos tipos de límites:

\begin{itemize}
	\item\textbf{Límite \emph{hard}:} El usuario no puede sobrepasarlo. Si llega al límite, el sistema no le permite usar más bloques ni inodos, impidiéndole aumentar el tamaño de sus ficheros o crear nuevos.
	\item\textbf{Límite \emph{soft}:} Debe configurarse como un valor inferior al límite \emph{hard} y puede sobrepasarse durante cierto tiempo (periodo de gracia), tras el cual el sistema actúa como si se hubiera superado el límite \emph{hard}.
\end{itemize}

Para trabajar con las cuotas de un sistema de archivos pueden usarse las órdenes \code{quota} para asignar cuotas al usuario pasado como argumento y \code{repquota} para mostrar estadísticas de las cuotas de todos los usuarios que hagan uso del sistema de archivos pasado como argumento.

