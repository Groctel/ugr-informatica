\chapter{Ejemplo de un cálculo no exitoso}

Como ejemplo simple de un cálculo no exitoso, supongamos los siguientes datos:

\begin{itemize}
	\item\textbf{Candidatos:} $C=\{1,1,2,3,4,5\}$.
	\item\textbf{Objetivo:} $O=500$.
\end{itemize}

Sabemos que el valor más cercano alcanzable con estos candidatos es $122=2\cdot3\cdot4\cdot5+1+1$.

Como primera y más cercana aproximación al objetivo, obtenemos $4\cdot5=20$.
Buscando aproximar en la primera iteración, el algoritmo llega a $3\cdot2\cdot aproximaci\acute on=120$.
Registrado este \texttt{valor} como el más alto conseguido hasta el momento, se intenta aproximar en una nueva iteración sumando $1$ a la aproximación, consiguiendo $121$.
Una última vez más, sumamos $1$ y obtenemos $122$, siendo éste el valor más cercano alcanzable por el algoritmo.

Dado que no tenemos un sistema de control del algoritmo, éste continuará hasta agotar el resto de valores sin éxito.
