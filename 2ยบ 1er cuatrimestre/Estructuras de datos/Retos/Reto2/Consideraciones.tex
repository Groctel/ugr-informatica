\chapter{Consideraciones preambulares}

Dados seis números aleatorios tomados de una reserva formada por nueve números de una cifra, cuatro de dos y uno de tres, buscamos encontrar, mediante las cuatro operaciones elementales, un número de 3 cifras.
Debido a que 9 de cada 14 números ($\~{}64.28\%$) tomados en cada caso van a ser de una cifra, parece sensato comenzar buscando los valores más aproximados a la solución.

Dado un valor obtenido como aproximación al valor objetivo de tres cifras, no importa si la diferencia entre ambos es positiva o negativa, por lo que la definiremos como $\lvert objetivo-valor\rvert$.

Dados dos valores iguales, es posible que para llegar a uno se hayan utilizado más cifras que para el otro ($4=2\cdot2$ y $4=2+1\cdot2$).
Consideraremos más cercano a la solución el valor que menos valores haya utilizado para ser calculado, ya que la mayor cantidad de valores restantes le ofrece una mayor probabilidad de llegar al número objetivo.

Tras la primera aproximación obtenemos una diferencia entre el valor objetivo y el valor aproximado.
Para simplificar los cálculos operaremos con esta diferencia y los valores restantes.
Es importante tener en cuenta que, tras la primera aproximación, sí se pueden obtener valores negativos.

