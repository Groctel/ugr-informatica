\section{}

\textbf{Una tribu de antropófagos comparte una olla en la que caben M misioneros.
Cuando algún salvaje quiere comer, se sirve directamente de la olla, a no ser que ésta esté vacía.
Si la olla está vacía, el salvaje despertará al cocinero y esperará a que éste haya rellenado la olla con otros M misioneros.}

\begin{lstlisting}[language=Pascal]
process Salvaje[ i : 0..2 ];
begin
	while true do begin
		{ esperar a servirse un misionero: }
		.....
		{ comer }
		Comer();
	end
end
\end{lstlisting}

\begin{lstlisting}[language=Pascal]
process Cocinero;
begin
	while true do begin
		{ dormir esperando solicitud para llenar: }
		......
		{ confirmar que se ha rellenado la olla }
		......
	end
end
\end{lstlisting}

\textbf{Implementar los procesos salvajes y cocinero usando paso de mensajes, usando un proceso olla que incluye una construcción de espera selectiva que sirve peticiones de los salvajes y el cocinero para mantener la sincronización requerida, teniendo en cuenta que:}

\begin{itemize}
	\item\textbf{La solución no debe producir interbloqueo.}
	\item\textbf{Los salvajes podrán comer siempre que haya comida en la olla.}
	\item\textbf{Solamente se despertará al cocinero cuando la olla esté vacía.}
\end{itemize}

\pagebreak

Asumimos las siguientes constantes globales:

\begin{itemize}
	\item\code{M}\textbf{:} Numero de misioneros.
	\item\code{S}\textbf{:} Número de salvajes.
\end{itemize}

El cocinero sólo debe preocuparse de recibir peticiones de relleno de la olla por parte de los salvajes:

\begin{lstlisting}[language=Pascal]
process Cocinero
	var peticion : integer := ...;
begin
	while true do begin
		select for salvaje := 0 to S
			when receive (peticion, Salvaje[salvaje]) do
				send (peticion, Olla);
				receive (peticion, Olla);
				send (peticion, Salvaje[salvaje]);
		end
	end
end
\end{lstlisting}

Los salvajes comienzam preguntándole a la olla cuántos misioneros quedan en ella.
Si quedan misioneros, le pide uno para comérselo.
En caso contrario, le pide al cocinero que la rellene y luego le pide a la olla un misionero.

\begin{lstlisting}[language=Pascal]
process Salvaje [ i : 0..S ];
	var comida    : integer := ...;
	    consulta  : integer := ...;
	    relleno   : integer := ...;
	    restantes : integer := 0;
begin
	while true do begin
		send (consulta, Olla);
		receive (restantes, Olla);

		if restantes == 0 then
			send (relleno, Cocinero);
			receive (relleno, Cocinero);
		end

		send (comida, Olla);
		receive (comida, Olla);

		Comer();
	end
end
\end{lstlisting}

La olla primero procesa la consulta de misioneros restantes.
Si no quedan misioneros, se prepara para procesar el relleno de la misma por parte del cocinero.
En cualquier caso, se prepara para recibir una petición de comida de cualquier salvaje una vez llena.

\begin{lstlisting}[language=Pascal]
process Olla
	var comida    : integer := ...;
	    consulta  : integer := ...;
	    relleno   : integer := ...;
	    restantes : integer := M;
begin
	while true do begin
		select for salvaje := 0 to S do
			when receive (consulta, Salvaje[salvaje]) do
				send (restantes, Salvaje[salvaje]);
		end

		if restantes = 0 then
			receive (relleno, Cocinero);
			restantes := M;
			send (relleno, Cocinero);
		end

		select for salvaje := 0 to S do
			when receive (comida, Salvaje[salvaje]) do
				send (comida, Salvaje[salvaje]);
				restantes := restantes - 1;
		end
	end
end
\end{lstlisting}
