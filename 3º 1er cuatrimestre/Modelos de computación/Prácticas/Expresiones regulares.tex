\chapter{}

\section{Enunciado}

Usar expresiones regulares y Lex para detectar cadenas dentro de un texto.

\section{Solución}

Trabajamos con el siguiente código de Lex:

\begin{lstlisting}[language=c]
minuscula [a-z]
mayuscula [A-Z]
numero    [0-9]

letra     ({minuscula}|{mayuscula})
normal    ({letra}|{numero})

usuario   (\@{letra}+)
dominio   (\.{letra}+)
byte      ((((""|1)(""|[1-9]))[0-9])|(2[0-4][0-9])|(25[0-5]))

dir_ip    (({byte}\.){3}{byte})
web       ({normal}+{dominio})
correo    (({normal}|"_")+\@{dominio})

%%

{usuario} {printf("Usuario válido: '%s'\n", yytext);}
{dominio} {printf("Dominio válido: '%s'\n", yytext);}
{byte}    {printf("Byte válido:    '%s'\n", yytext);}
{dir_ip}  {printf("Dir_ip válido:  '%s'\n", yytext);}
{web}     {printf("Web válido:     '%s'\n", yytext);}
{correo}  {printf("Correo válido:  '%s'\n", yytext);}
.+

%%

int yywrap ()
{
	return 1;
}

int main ()
{
	yylex();
	return 1;
}
\end{lstlisting}

\pagebreak

Lo compilamos con las siguiente órdenes:

\begin{lstlisting}[language=sh]
lex -o practica2.c practica2.l
gcc practica2.c -o practica2
\end{lstlisting}

Por último, lo ejecutamos y producimos la siguiente salida:

\begin{lstlisting}
@groctel
Usuario válido: '@groctel'

ugr.es
Web válido:     'ugr.es'

143
Byte válido:    '143'

0.0.0.0
Dir_ip válido:  '0.0.0.0'

255.123.12.96
Dir_ip válido:  '255.123.12.96'

.com
Dominio válido: '.com'

deiit@ugr.es
Correo válido:  'deiit@ugr.es'
\end{lstlisting}
