\chapter{}

\section{Enunciado}

Para un lenguaje libre contexto que vosotros elijáis, comprobar si una gramatica para el lenguaje es ambigua encontrando dos árboles de derivación distintos para una misma cadena del lenguaje.

\section{Solución}

Especificamos la siguiente gramática:

\begin{align*}
	G &= (V, T, P, S) : \\
	V &= \{E\} \\
	T &= \{a, b, c, d, e\} \\
	P &=
		\begin{cases}
		\begin{array}{lll}
			E \rightarrow /E/ &  & E \rightarrow /bE      \\
			E \rightarrow aEa &  & E \rightarrow /cE      \\
			E \rightarrow bEb &  & E \rightarrow a        \\
			E \rightarrow cEc &  & E \rightarrow b        \\
			E \rightarrow a/E &  & E \rightarrow c        \\
			E \rightarrow b/E &  & E \rightarrow /        \\
			E \rightarrow c/E &  & E \rightarrow \epsilon \\
			E \rightarrow /aE &  &
		\end{array}
 		\end{cases} \\
	S &= E
\end{align*}

Vamos a demostrar que la cadena \texttt{a/b/c/b/a} puede ser generada por una gramática ambigua generando dos árboles de derivación:

\begin{center}
\begin{tikzpicture}
\Tree
[.\texttt{E}
	[.\texttt{a} ]
	[.\texttt{E}
		[.\texttt{/} ]
		[.\texttt{E}
			[.\texttt{b} ]
			[.\texttt{E}
				[.\texttt{/} ]
				[.\texttt{E}
					[.\texttt{c} ]
				]
				[.\texttt{/} ]
			]
			[.\texttt{b} ]
		]
		[.\texttt{/} ]
	]
	[.\texttt{a} ]
]
\end{tikzpicture}
\ \ \ \ \ \ \ \ \ \ \ \ \ \ \ \
\begin{tikzpicture}
\Tree
[.\texttt{E}
	[.\texttt{a} ]
	[.\texttt{/} ]
	[.\texttt{E}
		[.\texttt{b} ]
		[.\texttt{/} ]
		[.\texttt{E}
			[.\texttt{c} ]
			[.\texttt{/} ]
			[.\texttt{E}
				[.\texttt{b} ]
				[.\texttt{/} ]
				[.\texttt{E}
					[.\texttt{a} ]
				]
			]
		]
	]
]
\end{tikzpicture}
\end{center}

De esta forma, demostramos que la gramática es ambigua.
