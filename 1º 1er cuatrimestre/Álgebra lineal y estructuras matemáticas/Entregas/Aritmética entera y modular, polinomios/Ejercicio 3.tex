\section{}\label{ej-1-3}

\subsection*{Enunciado (El problema del mono y los cocos)}

Cinco hombres y un mono naufragan en una isla desierta.
Durante el primer día los hombres se dedican a recoger cocos.
Al final del día deciden dejar el reparto para el día siguiente.
Por la noche, uno de ellos despierta y, desconfiado, decide separar su parte.
Divide los cocos en cinco montones, toma su parte y, como sobra un coco, se lo da al mono.
Poco después, un segundo náufrago se despierta y hace lo mismo.
Al dividir los cocos en cinco montones, vuelve a sobrar un coco y también se lo da al mono.
Uno tras otro, el tercero, cuarto y quinto náufragos hacen lo mismo.
Al día siguiente por la mañana, dividen los cocos en cinco montones sin que sobre ninguno.
¿Cuántos cocos se habían recolectado inicialmente?

\subsection*{Planteamiento del problema}

A lo largo de este problema se da repetidas veces la misma situación: Un náufrago toma un número de cocos y lo divide en cinco montones iguales y sobra 1.
Para la primera iteración, vamos a definir $x$ como el número total de cocos recolectados (la solución del ejercicio) e $y_1$ como el tamaño de los montones en los que el primer náufrago ha dividido los cocos.
De esta forma, tenemos la siguiente ecuación diofántica:

\[x = 5y_1 + 1\]

Podemos llamar $y_2$ al tamaño de los montones en lo que divide los cocos el segundo náufrago y conseguir la siguiente ecuación:

\[x - y_1 - 1 = 5y_2 + 1\]

De la misma forma, vamos a plantear el problema completo llamando $y_i$ al tamaño de los montones de cada iteración $i$ e $y$ a la cantidad de cocos que se lleva finalmente cada náufrago:

\[
\begin{array}{rcl}
	x                                   & = & 5y_1 + 1 \\
	x - y_1 - 1                         & = & 5y_2 + 1 \\
	x - y_1 - y_2 - 2                   & = & 5y_3 + 1 \\
	x - y_1 - y_2 - y_3 - 3             & = & 5y_4 + 1 \\
	x - y_1 - y_2 - y_3 - y_4 - 4       & = & 5y_5 + 1 \\
	x - y_1 - y_2 - y_3 - y_4 - y_5 - 5 & = & 5y
\end{array}
\]

\subsection*{Resolución del problema}

Como nuestro objetivo es calcular el número de cocos recolectados inicialmente en función del número de cocos obtenidos por cada uno de los náufragos, vamos a ir expresando cada $y_i$ en función de $x$ en cada una de las expresiones hasta llegar a tener únicamente $x$ e $y$.
Antes de empezar, vamos a simplificar las expresiones sabiendo que de cada $5y_i$ montones creador por cada náufrago quedan $4y_i$ montones tras tomar su parte, de forma que cada una incluya el mínimo número de variables posibles:

\[
\left\{
\begin{array}{rcl}
	x                                   & = & 5y_1 + 1 \\
	x - y_1 - 1                         & = & 5y_2 + 1 \\
	x - y_1 - y_2 - 2                   & = & 5y_3 + 1 \\
	x - y_1 - y_2 - y_3 - 3             & = & 5y_4 + 1 \\
	x - y_1 - y_2 - y_3 - y_4 - 4       & = & 5y_5 + 1 \\
	x - y_1 - y_2 - y_3 - y_4 - y_5 - 5 & = & 5y
\end{array}
\right.
\iff%
\left\{
\begin{array}{rcl}
	x    & = & 5y_1 + 1 \\
	4y_1 & = & 5y_2 + 1 \\
	4y_2 & = & 5y_3 + 1 \\
	4y_3 & = & 5y_4 + 1 \\
	4y_4 & = & 5y_5 + 1 \\
	4y_5 & = & 5y
\end{array}
\right.
\]

Comenzamos por la primera expresión:

\[x = 5y_1 + 1 \iff y_1 = \frac{x-1}{5}\]

Para no ir arrastrando fracciones, vamos a multiplicar toda la siguiente expresión por el denominador de la fracción en cuyo numerador se encuentra la $x$ (en este caso 5).
De esta forma, aunque trabajemos con números mucho más grandes, lo haremos sin ir acumulando fracciones que, inevitablemente, tendríamos que simplificar al final.
Sustituimos en la segunda expresión:

\[
\begin{array}{rcll}
	4y_1                     & = & 5y_2 + 1        & \iff \\
	4\big(\frac{x-1}{5}\big) & = & 5y_2 + 1        & \iff \\
	4x - 4                   & = & 25y_2 + 5       & \iff \\
	y_2                      & = & \frac{4x-9}{25} &
\end{array}
\]

Sustituimos en la tercera expresión:

\[
\begin{array}{rcll}
	4y_2                       & = & 5y_3 + 1           & \iff \\
	4\big(\frac{4x-9}{25}\big) & = & 5y_3 + 1           & \iff \\
	16x - 36                   & = & 125y_3 + 25        & \iff \\
	y_3                        & = & \frac{16x-61}{125} &
\end{array}
\]

Sustituimos en la cuarta expresión:

\[
\begin{array}{rcll}
	4y_3                          & = & 5y_4 + 1            & \iff \\
	4\big(\frac{16x-61}{125}\big) & = & 5y_4 + 1            & \iff \\
	64x - 244                     & = & 625y_4 + 125        & \iff \\
	y_4                           & = & \frac{64x-369}{625} &
\end{array}
\]

Sustituimos en la quinta expresión:

\[
\begin{array}{rcll}
	4y_4                           & = & 5y_5 + 1               & \iff \\
	4\big(\frac{64x-369}{625}\big) & = & 5y_5 + 1               & \iff \\
	256x - 976                     & = & 3125y_5 + 625          & \iff \\
	y_5                            & = & \frac{256x-1601}{3125} &
\end{array}
\]

Sustituimos en la sexta expresión:

\[
\begin{array}{rcll}
	4y_5                              & = & 5y + 1                   & \iff \\
	4\big(\frac{256x-1601}{3125}\big) & = & 5y + 1                   & \iff \\
	1024x - 4244                      & = & 15625y + 3125            & \iff \\
	y                                 & = & \frac{1024x-5547}{15625} &
\end{array}
\]

Con este último resultado, expresamos la ecuación diofántica a resolver:

\[y = \frac{1024x-5547}{15625} \iff 1024x - 15625y = 5547 \iff 1024x \equiv 5547 \mod 15625\]

Como 1024 es potencia de 2 y 15625 es potencia de 5 (es la forma en la que hemos obtenido estos números en los pasos anteriores), es trivial que $mcd(1024,15625)|5547$, aunque lo veremos igualmente en el algoritmo extendido de Euclides para encontrar el ivnerso de $x$ con el que resolver la congruencia:

\begin{center}
\begin{tabular}{r r r}
	$\boldsymbol{r}$ & $\boldsymbol{c}$ & $\boldsymbol{v}$     \\
	\toprule
	15625            &                  & 0                    \\
	 1024            &                  & 1                    \\
	 265             & 15               & $-15$                \\
	 229             & 3                & 46                   \\
	 36              & 1                & $-61$                \\
	 13              & 6                & 412                  \\
	 10              & 2                & $-885$               \\
	 3               & 1                & 1297                 \\
	 1               & 3                & $\boldsymbol{-4776}$ \\
\end{tabular}
\end{center}

Tenemos que $-4776 = 10849 \in\mathbb{Z}_{15625}$ y comprobamos que $1024 \cdot 10849 = 1110937 = 1 \in\mathbb{Z}_{15625}$, con lo que finalizamos la resolución de la congruencia:

\[
\left\{
\begin{array}{rcrl}
	1024x & \equiv & 5547 & \mod 15625 \\
	x     & \equiv & 4773 & \mod 15625 \\
         &        &      & \boldsymbol{x = 15267y + 4773}
\end{array}
\right.
\]

\subsection*{Solución}

Aunque no sepamos el número exacto de cocos, sabemos que los náufragos recogieron $15267y + 4773$ cocos para $y$ cocos que se repartieron finalmente cada uno.
Sinceramente, no entiendo cómo no aprovecharon la maquinaria necesaria para recoger tantísimos cocos \textit{el primer día} para salir de la isla.
