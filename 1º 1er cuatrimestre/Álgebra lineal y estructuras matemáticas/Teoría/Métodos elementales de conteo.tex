\section{Métodos elementales de conteo}\label{metodos-elementales-de-conteo}

A lo largo de este tema vamos a ver cómo computar las diferentes formas en las que se puede dispone de varios elementos concretos y abstractos.
Antes de nada, vamos a redefinir las operaciones suma y producto para trabajar de forma más eficiente con ellas.

\subsection{Principio de inclusión-exclusión}\label{principio-de-inclusion-exclusion}

\subsubsection{Principio de la suma}

\[A \cap B = \emptyset \Rightarrow |A \cup B| = |A| + |B|\]

Si tenemos dos conjuntos disjuntos, es decir, que $a \neq b, \forall a \in A, b \in B$, podemos contar los elementos de ambos conjuntos empezando a contar todos los de $A$ y seguir con los de $B$ hasta llegar al total.
Sin embargo, si conocemos la cardinalidad de los conjuntos, podemos tomar el atajo de simplemente sumarlas.
Vamos a aplicar este princpio a la combinatoria para encontrar el número de formas de las que se pueden realizar varias tareas siempre que dichas tareas sean incompatibles.
Por ejemplo, podemos usar este principio para contar las diferentes formas de elegir manzanas de dos cestos, ya que no podemos coger manzanas de dos cestos a la vez con una mano.

Si los sucesos no son incompatibles, usamos la forma general del princpio de inclusión-exclusión:

\[|A \cup B| = |A| + |B| - |A \cap B|\]

Por ejemplo, seguimos este principio para calcular el número de animales que o bien viven en la sabana ($A$) o son herbívoros ($B$) pero no ambos ($A \cap B$).
Este princpio es escalable a más conjuntos.
En el caso de tres, por ejemplo, tenemos la siguiente expresión:

\[|A \cup B \cup C| = |A| + |B| + |C| - |A \cap B| - |A \cap C| - |B \cap C| + |A \cap B \cap C|\]

Vemos que tenemos que sumar las uniones de números impares de conjuntos y restar las pares, por lo que podemos dar una forma general para este principio:

\[|A_1 \cup A_2 \cup \cdots \cup A_n| = \sum_{k=1}^{n} {(-1)}^{k+1} \cdot \sum_{1 \leq i_1 < \cdots < i_k \leq n} |A_{i_1} \cap A_{i_2} \cap \cdots \cap A_{i_n}|\]

\subsection{Principio del producto}\label{principio-del-producto}

\[A_1 \times A_2 \times \cdots \times A_3| = |A_1| \cdot |A_2| \cdots |A_n|\]

Volviendo a la analogía de las tareas, utilizamos el principio del producto para contar las diferentes formas en las que podemos resolver varias tareas consecutivas, de forma que, si cada tarea $A_i$ se puede de $|A_i|$ formas, dicha tarea y la siguiente ($A_{i+1}$) se pueden resolver de $|A_i| \cdot |A_{i+1}|$ formas.
Este principio es escalable a cualquier número de tareas.

\subsection{Principio el palomar}\label{principio-del-palomar}

Dado un palomar con $n$ agujeros para $p$ palomas en el que $p > n$, necesariamente tendrá que haber un agujero con más de una paloma.
De la misma forma, cualquier elemento $p$ que se desee repartir en $n$ partes, siempre habrá dos elementos repartidos en la misma parte si $p > n$.
