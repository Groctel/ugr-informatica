\section{Definición}\label{cuerpos-finitos-definicion}

\subsection{Anillos conmutativos}\label{anillos-conmutativos}

Para dar la definición de un cuerpo debemos dar primero la definición de un anillo conmutativo, ya que la primera es un caso particular de la segunda.
Decimos que un conjunto $A$ tiene estructura de anillo conmunitativo si define las siguientes operaciones:

\[
	\begin{array}{rl}
		\text{Operación suma: }    & A \times A \xrightarrow{+} A     \\
                                 & (a,b) \mapsto a + b              \\
		\text{Operación producto } & A \times A \xrightarrow{\cdot} A \\
                                 & (a,b) \mapsto a \cdot b          \\
\end{array}
\]

En esta estructura, la suma y el producto satisface las siguientes propiedades:

\subsubsection{Propiedades de la suma}

\begin{center}
\begin{tabular}{l l}
	\textbf{Propiedad}     & \textbf{Expresión}                           \\
	\toprule
	Asociativa             & $(a+b) + c = a + (b+a), \forall a,b,c \in A$ \\
	Conmutativa            & $a+b = b+a, \forall a,b \in A$               \\
	Elemento neutro (0)    & $a+0 = a, \forall a \in A$                   \\
	Elemento opuesto ($b$) & $\exists b : a + b = 0, \forall a \in A$     \\
\end{tabular}
\end{center}

\subsubsection{Propiedades del producto}

\begin{center}
\begin{tabular}{l l}
	\textbf{Propiedad}  & \textbf{Expresión}                                               \\
	\toprule
	Asociativa          & $(a \cdot b) \cdot c = a \cdot (b \cdot c), \forall a,b,c \in A$ \\
	Conmutativa         & $a \cdot b = b \cdot a, \forall a,b \in A$                       \\
	Elemento neutro (1) & $a \cdot 1 = a, \forall a \in A$                                 \\
\end{tabular}
\end{center}

La suma también debe ser distributiva respecto al producto:

\[a \cdot (b+c) = a \cdot b + a \cdot c, \forall a,b,c \in A\]

\subsection{Existencia de un inverso}

Para que un anillo conmutativo tenga estructura de cuerpo debe cumplirse una última propiedad:

\[\exists b : a \cdot b = 1, \forall a \in A, a \neq 0\]

Esta propiedad indica que todo elemento $a \in A$ tiene inverso, como es el caso de $\mathbb{C}$, $\mathbb{Q}$, $\mathbb{R}$ y $\mathbb{Z}_p$ para números primos $p$.
Como vimos en \S\ref{clases-residuales-modulo-m}, llamaremos unidades a los elementos que tienen inverso.
