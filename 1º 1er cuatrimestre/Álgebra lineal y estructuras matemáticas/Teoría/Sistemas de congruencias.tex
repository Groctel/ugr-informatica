\section{Sistemas de congruencias}\label{sistemas-de-congruencias}

\subsection{Ecuaciones en congruencias}

Como ya hemos visto, al realizar cálculos en módulo $m$, nos encontramos constantemente valores que son equivalentes.
Por ejemplo, en $\mathbb{Z}_3$ tenemos que $\{\ldots, -8, -5 ,-2, 1, 4, 7, \ldots\} = 1$ y, además, tenemos que todos estos valores valen exactamente $3x + 1, \forall x \in\mathbb{Z}$.
En esta sección vamos a trabajar con esta propiedad de la aritmética modular, a la que llamaremos congruencia.
Diremos que, para un caso general, $ax$ es congruente con $b$ módulo $m$ y lo expresaremos de la siguiente forma:

\[ax \equiv b \mod m\]

Las congruencias satisfacen las siguientes propiedades:

\begin{itemize}
	\item $a \equiv a' \mod m \land b \equiv b' \mod m \Rightarrow a'x \equiv b' \mod m$
	\item $d|a \land d|b \land d|c \Rightarrow \frac{a}{d}x \equiv \frac{b}{d} \mod \frac{m}{d}$
	\item $mcd(c,m) = 1 \Rightarrow cax \equiv cb \mod m$
\end{itemize}

La primera propiedad nos sirve para simplificar (y dificultar, aunque no es buena idea) las congruencias.
Por ejemplo, para la expresión $25x \equiv 7 \mod 3$, tenemos que $25 \equiv 1 \mod 3$ y que $7 \equiv 2 \mod 3$, por lo que podemos simplificar la expresión como $x \equiv 2 \mod 3$.
Cabe destacar que $n \equiv n \mod m, \forall n \in\mathbb{Z}$, por lo que no es necesario reducir todos ambos términos de la congruencia para poder reducir uno de ellos.

La segunda propiedad nos permite simplificar la congruencia dividiendo todos los términos entre un divisor común, preferiblemente su $mcd$.
De esta forma, podemos simplificar enormemente los cálculos y resolver más fácilmente la congruencia.

Por último, la tercera propiedad es la que nos permite encontrar la solución.
Si podemos encontrar un número $c$ que nos permita aislar la $x$ en su miembro, podremos resolver la congruencia.
Para que esto ocurra, se debe dar que $c=a^{-1}$, por lo que las congruencias sólo tienen solución si $a$ es unidad ($mcd(a,m)=1$) y si $mcd(a,m)|b$.
La congruencia quedaría como $aa^{-1}x \equiv ba^{-1} \mod m \Leftrightarrow x \equiv ba^{-1} \mod m$.

Como dijimos al principio de esta sección, las congruencias tienen soluciones infinitas.
Por ejemplo, $x \equiv 3 \mod 7$ tiene como soluciones $\{\ldots, -18, -11, -4, 3, 10, 17, \ldots\}$, es decir, cualquier valor $x = 7k + 3, \forall k \in\mathbb{Z}$.

Como ejemplo completo, resolvamos la siguiente congruencia:

\[14x \equiv 7 \mod 5\]

Para empezar, vamos a simplificar la congruencia.
Tenemos que $14 \equiv 4 \mod 5$ y que $7 \equiv 2 \mod 5$, por lo que $4x \equiv 2 \mod 5$.
Es fácil ver que $mcd(4,5)|2$, ya que el 1 divide a todos los números, por lo que sabemos que la congruencia tiene solución.
Nos falta encontrar $4^{-1} \mod 5$ para poder resolverla.
Sin entrar en los cálculos, tenemos que este número es 4, ya que $4 \cdot 4 = 16 \equiv 1 \mod 5$.
Sustituimos y tenemos que $x \equiv 8 \mod 5$.
Reducimos de nuevo para obtener que $x \equiv 3 \mod 5$ y ya tenemos nuestra solución, que es que $x = 5k + 3$.

\subsection{Sistemas de ecuaciones en congruencias}

Al igual que con las ecuaciones con las que trabajamos normalmente, las ecuaciones en congruencias se pueden agrupar para formar sistemas.
Veamos algunos ejemplos de resolución de estos sistemas sin entrar en cálculos complejos:

\[
\left\{
\begin{array}{ll}
	3x \equiv 0 \mod 5 & \iff x = 5k = \{\ldots, 0, 5, 10, 15, 20, 25, 30, \ldots\} \\
	4x \equiv 0 \mod 6 & \iff x = 3k = \{\ldots, 0, 3, 6, 9, 12, 15, 18, 21, 24, 27, 30, \ldots\}
\end{array}
\right.
\]

Las soluciones de este sistema son las soluciones comunes a ambas congruencias.
En este caso, las soluciones son 15, 30 y cualquier otro $x = 15k$.

\[
\left\{
\begin{array}{ll}
	3x \equiv 2 \mod 7 & \iff x = 7k + 3 \\
	5x \equiv 1 \mod 9 & \iff x = 9k + 2
\end{array}
\right.
\]

Ahora tenemos soluciones como $38 = 7\cdot 5 + 3 = 9 \cdot 4 + 2$ ó $101 = 7 \cdot 14 + 3 = 9 \cdot 11 + 2$.
Podríamos buscar una expresión que satisfaciera estas condiciones, pero no vamos a hacerlo para un caso particular sabiendo que existe una forma más sencilla que veremos a continuación.

\[
\left\{
\begin{array}{ll}
	3x \equiv 2 \mod 10 & \iff x = 10k + 4 \\
	5x \equiv 3 \mod 8  & \iff x = 8k + 5
\end{array}
\right.
\]

Podemos ver que este sistema no tiene solución, ya que las soluciones de la primera congruencia son todas pares y las de la segunda, impares.
Visto esto, vamos a dar un método para resolver sistemas de congruencias.
Usemos como ejemplo el siguiente sistema:

\[
\left\{
\begin{array}{l}
	2x \equiv 4 \mod 5 \\
	8x \equiv 7 \mod 13 \\
	5x \equiv 3 \mod 7
\end{array}
\right.
\]

Para resolver este sistema vamos a empezar por resolver $x$ para la primera congruencia e introducir el resultado en la segunda.
Luego resolveremos la segunda e introduciremos el resultado en la tercera y así en cadena hasta llegar al final.
Si en algún momento encontrásemos que una de las congruencias no tiene solución, determinamos el sistema no la tiene.
Vamos a resolver $x$ para la primera congruencia:

\[2x \equiv 4 \mod 5\]

Tenemos que $mcd(2,5)$ y que $2^{-1} = 3$, por lo que $6 \equiv 12 \mod 5$, que reducimos a $1 \equiv 2 \mod 5$.
Tenemos, por tanto, que $x = 5k_0 + 1$.
Dado este valor de $x$, lo introducimos en la segunda congruencia:

\[
\begin{array}{ll}
	8x          & \equiv 7  \mod 13 \iff \\
	8(5k_0 + 1) & \equiv 7  \mod 13 \iff \\
	40k_0 + 8   & \equiv 7  \mod 13 \iff \\
	k_0         & \equiv 12 \mod 13
\end{array}
\]

Tenemos, por tanto, que $k_0 = 13k_1 + 12$.
Introducimos este valor en la expresión de $x$, sustituyendo por $k_0$:

\[
	x = 5k_0 + 1 = 5(13k_1 + 12) + 1 = 65k_1 + 61
\]

Introducimos este valor en la tercera congruencia y resolvemos:

\[
\begin{array}{ll}
	5x            & \equiv 3 \mod 7 \iff \\
	5(65k_1 + 61) & \equiv 3 \mod 7 \iff \\
	325k_1 + 305  & \equiv 3 \mod 7 \iff \\
	3k_1          & \equiv 6 \mod 7
\end{array}
\]

Resolvemos la congruencia $3k_1 \equiv 6 \mod 7$.
Comprobamos que $mcd(3,7) = 1$ y que $3^{-1} = 5$, por lo que $k_1 \equiv 2 \mod 7$ y que $k_1 = 7k + 2$.
Introducimos esta última $k$ en la expresión de $x$ y la desarrollamos para calcular la solución del sistema:

\[x = 65k_1 + 61 = 65(7k + 2) + 61 = 455k + 191\]
