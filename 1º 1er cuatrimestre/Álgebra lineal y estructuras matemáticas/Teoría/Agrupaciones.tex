\section{Agrupaciones}\label{agrupaciones}

En este tema vamos a ver las diferentes formas de agrupar los elementos de un conjunto.
Para diferenciarlas, tendremos en cuenta el número de elementos de dicho conjunto que agrupamos y si nos importa o no su orden.
Como regla general podemos fijarnos en el siguiente diagrama para elegir el tipo de agrupación que queremos utilizar a la hora de resolver el ejercicio:

\[
	\text{¿Todos los elementos?}
	\begin{cases}
		\text{\textbf{SÍ\@:} Permutación. ¿Repeticiones?}
		\begin{cases}
			\text{\textbf{SÍ\@: }} PR_{n}^{k} = \frac{n!}{\prod_{1}^{k}(n_k!)} \\
			\text{\textbf{NO\@: }} P_n = n! \\
		\end{cases}
		\\
		\\
		\text{\textbf{NO\@:} ¿Ordenados?}
		\begin{cases}
			\text{\textbf{Sí\@:} Variación. ¿Repeticiones?}
			\begin{cases}
				\text{\textbf{SÍ\@: }} VR_{m}^{n} = m^n \\
				\text{\textbf{NO\@: }} V_{m}^{n} = \frac{m!}{(m-n)!} \\
			\end{cases}
			\\
			\\
			\text{\textbf{NO\@:} Combinación. ¿Repeticiones?}
			\begin{cases}
				\text{\textbf{SÍ\@: }} CR_{m}^{n} = \frac{(m+n-1)!}{n!(m-1)!} \\
				\text{\textbf{NO\@: }} C_{m}^{n} = \frac{m!}{n!(m-n)!}
			\end{cases}
		\end{cases}
	\end{cases}
\]

Supongamos que, como todo conocedor de la ciencia probabilística que se precie, somos los orgullosos poseedores de la típica caja con pelotas de colores\textsuperscript{TM}.
Las tres agrupaciones que podemos hacer con las pelotas de la caja son sacarlas una a una, ordenarlas en fila y meterlas de vuelta a la caja.
Todos los problemas de combinatoria se pueden reducir a operaciones con las pelotas de nuestra caja, por lo que vamos a utilizar esta analogía, junto con otras, para definir las diferentes agrupaciones.

\subsection{Variaciones}\label{variaciones}

Hablamos de variaciones cuando sacamos pelotas de la caja.
Cada vez que saquemos una pelota apuntaremos su color y el orden en el que la hemos sacado.
Por ejemplo, podríamos obtener la siguiente lista:

\begin{center}
\begin{tabular}{c l}
	\textbf{Orden} & \textbf{Color} \\
	\toprule
	1              & Amarillo       \\
	2              & Celeste        \\
	3              & Burdeos        \\
	4              & Dorado
\end{tabular}
\end{center}

Podría darse el caso de que, cada vez que tomásemos una pelota, la devolviésemos a la caja.
En este caso, podríamos obtener la siguiente lista:

\begin{center}
\begin{tabular}{c l}
	\textbf{Orden} & \textbf{Color} \\
	\toprule
	1              & Celeste        \\
	2              & Amarillo       \\
	3              & Celeste        \\
	4              & Burdeos
\end{tabular}
\end{center}

Vamos a dar una definición formal a esta idea intuitiva.

\subsubsection{Variaciones con repetición}

Llamamos variaciones con repetición de $n$ elementos tomados de $m$ en $m$ y lo expresamos formalmente como $VR_{n}^{m}$ a cada una de las formas en las que podemos elegir $m$ elementos de un conjunto de cardinal $n$ pudiendo haber repetidos.
Diferenciamos las elecciones según el elemento que hemos escogido y el orden en el que lo hemos hecho.
Por ejemplo, éstas son las diferentes formas en las que podemos escoger, con repeticiones, entre Atanasio, Bizancio, Clotilde y Delfina:

\[VR_{m}^{n} = m^n\]

\[
\begin{array}{cccc}
	aa & ba & ca & da \\
	ab & bb & cb & db \\
	ac & bc & cc & dc \\
	ad & bd & cd & dd
\end{array}
\]

Debemos tener en cuenta que la primera vez que elegimos a Atanasio o a Clotilde no tiene nada que ver con la segunda vez que lo hacemos, ya que unas representan las tuplas $\{1,a\}$ y $\{1,c\}$ mientras que las otras, las tuplas $\{2,a\}$ y $\{2,b\}$.

\subsubsection{Variaciones sin repetición}

Llamamos variaciones sin repetición de $n$ elementos tomados de $m$ en $m$ y lo expresamos formalmente como $V_{n}^{m}$ a cada una de las formas en las que podemos elegir $m$ elementos de un conjunto de $n$ no pudiendo haber repetidos.
Diferenciamos las elecciones según el elemento que hemos escogido y el orden en el que lo hemos hecho.
Vamos a utilizar el mismo ejemplo que antes pero sin repetir nombres:

\[V_{m}^{n} = \frac{m!}{(m-n)!}\]

\[
\begin{array}{cccc}
	   & ba & ca & da \\
	ab &    & cb & db \\
	ac & bc &    & dc \\
	ad & bd & cd &
\end{array}
\]

Dado que no podemos repetir elementos, no podemos elegir nunca $aa$, $bb$, $cc$ o $dd$.
Cada vez que elegimos elementos, el primero lo podemos elegir de $n$ formas, el segundo de $n - 1$, el tercero de $n - 2$ y así hasta $n - m + 1$ formas.
Simplificamos la expresión en un cociente de factoriales para obtener el total de variaciones sin repetición sobre un conjunto.

\subsection{Permutaciones}\label{permutaciones}

Hablamos de permutaciones cuando tomamos todas las pelotas de nuestra caja y las colocamos en orden sobre la mesa.
Al hacer esto, anotamos el orden en el que aparecen los colores, siendo cada una de las formas en las que podemos agrupar las pelotas una permutación.
Por ejemplo, podríamos obtener la siguiente lista:

\begin{itemize}
	\item
		Celeste, Burdeos, Dorado, Amarillo.
	\item
		Amarillo, Celeste, Dorado, Burdeos.
	\item
		Burdeos, Amarillo, Celeste, Dorado.
	\item
		Celeste, Dorado, Amarillo, Burdeos.
\end{itemize}

Para ordenar las pelotas hemos tenido que sacarlas todas de la caja y una forma de ordenarlas sería ir colocándolas según las sacásemos.
Esto funciona porque, realmente, las permutaciones son variaciones con o sin repetición donde $n = m$.
Al igual que tenemos variaciones con repetición, tenemos también permutaciones con repetición.
Éstas se presentan cuando tenemos varias pelotas del mismo color, de forma que la lista podría ser de la siguiente forma si tuviéramos dos pelotas celestes:

\begin{itemize}
	\item
		Celeste, Burdeos, Dorado, Amarillo, Celeste.
	\item
		Celeste, Burdeos, Dorado, Amarillo, Celeste.
	\item
		Amarillo, Celeste, Celeste, Dorado, Burdeos.
	\item
		Burdeos, Celeste, Amarillo, Celeste, Dorado.
	\item
		Celeste, Dorado, Amarillo, Celeste, Burdeos.
\end{itemize}

Aunque para el primer y segundo elemento las pelotas celestes se intercambian de sitio, ambas permutaciones son iguales porque estamos teniendo en cuenta el color de las pelotas, no su identidad original.
De la misma forma, al ordenar las letras de \texttt{algebra}, la primera y la última \texttt{a} deben considerarse iguales.

Vamos a dar una definición formal a esta idea intuitiva.

\subsubsection{Permutaciones sin repetición}

Llamamos permutaciones sin repetición de $n$ elementos y lo expresamos formalmente como $P_n$ a cada una de las formas en las que podemos ordenar los $n$ elementos de un conjunto no pudiendo haber repetidos.
Continuamos con Atanasio, Bizancio, Clotilde y Delfina para ver las diferentes formas en las que podemos arreglar el conjunto:

\[P_n = n!\]

\[
\begin{array}{cccc}
	abcd & bacd & cabd & dabc \\
	abdc & badc & cadb & dacb \\
	acbd & bcad & cbad & dbac \\
	acdb & bcda & cbda & dbca \\
	adbc & bdac & cdab & dcab \\
	adcb & bdca & cdba & dcba
\end{array}
\]

\subsubsection{Permutaciones con repetición}

Llamamos permutaciones con repetición de $n$ elementos con $k$ tipos de elementos y lo expresamos formalmente como $PR_{n}^{n_k}$ a cada una de las formas en las que podemos ordenar los $n$ elementos de un conjunto en el que podemos diferenciar $k$ tipos de elementos.
Vamos a ver de qué formas podemos ordenar a dos Atanasios, Bizancio y Clotilde:

\[PR_{n}^{k} = \frac{n!}{\prod_{1}^{k}(n_k!)}\]

\[
\begin{array}{cccc}
	abca & baca & caba & aabc \\
	abac & baac & caab & aacb \\
	acba & bcaa & cbaa & abac \\
	acab & bcaa & cbaa & abca \\
	aabc & baac & caab & acab \\
	aacb & baca & caba & acba
\end{array}
\]

Es fácil ver que tenemos varias respuestas iguales como dos $cbaa$, por ejemplo.
Para calcular el total de permutaciones debemos tener en cuenta el número total de elementos ($n = 4$) y el número de elementos de cada tipo.
El número de elementos de cada tipo $i$ se representa como $n_i$, por lo que lo introducimos en la fórmula general mostrada anteriormente.
Vamos a desarrollarla para este caso:

\[PR_{4}^{3} = \frac{4!}{2! \cdot 1! \cdot 1!} = \frac{24}{2} = 12\]

\subsection{Combinaciones}\label{combinaciones}

Hablamos de combinaciones cuando metemos pelotas en la caja.
No nos importa el orden en que las vamos metiendo, sino el resultado final, que son las pelotas que hay en la caja.
Por ello, una caja con las pelotas de color Amarillo y Celeste es lo mismo que otra con las pelotas de color Celeste y Amarillo.
Por tanto, cuando trabajemos con combinaciones lo haremos igual que si trabajásemos con conjuntos (o multiconjuntos en caso de que haya repeticiones).

Vamos a dar una definición formal a esta idea intuitiva.

\subsubsection{Combinaciones sin repetición}

Llamamos combinaciones sin repetición de $n$ elementos tomados de $m$ en $m$ y lo expresamos formalmente como $C_{m}^{n}$ a cada una de las formas en las que podemos seleccionar $m$ elementos de un conjunto de cardinal $n$ no pudiendo haber repetidos.
Diferenciamos las elecciones únicamente por el elemento escogido, nunca por el orden en el que lo hemos hecho.
Por ejemplo, Atanasio, Bizancio, Clotilde y Delfina están a la espera del resultado de un sorteo para un viaje a la Luna que escogerá a dos de ellos para embarcarse.
Vamos a ver de qué formas formas pueden salir elegidos:

\[C_{m}^{n} = \frac{m!}{n!(m-n)!}\]

\[
\begin{array}{cccc}
	ab &    &    \\
	ac & bc &    \\
	ad & bd & cd \\
\end{array}
\]

Podemos ver esta tabla como los elementos de una matriz $4 \times 4$ inferiores a la diagonal principal.
Los elementos de la diagonal principal serían $aa$, $bb$, $cc$ y $dd$ y, sobre ella, estarían los mismos elementos que vemos pero invertidos (el elemento $(2,1) = ab$ sería el $(1,2) = ba$).
Como no nos importa el orden y no tenemos repeticiones, nos quedamos con sólo una de las dos mitades\footnote{%
	Este razonamiento es el mismo que aplicamos al programar ejercicios de cálculo de rutas con una matriz de adyacencia:
	La distancia de una ciudad $i$ a $i$ es 0 y no la contamos y la distancia de $i$ a $j$ es la misma que de $j$ a $i$, por lo que sólo necesitamos reservar media matriz.
}.

\subsubsection{Combinaciones sin repetición}

Llamamos combinaciones sin repetición de $n$ elementos tomados de $m$ en $m$ y lo expresamos formalmente como $CR_{m}^{n}$ a cada una de las formas en las que podemos seleccionar $m$ elementos de un conjunto de cardinal $n$ pudiendo haber repetidos.
Diferenciamos las elecciones únicamente por el elemento escogido, nunca por el orden en el que lo hemos hecho.
Volvamos al ejemplo viaje espacial.
Los organizadores del sorteo le han hecho la jugada a los participantes y van a tener que competir por el puesto recolectando \textit{me gustas}.
Como sólo tienen dos amigos aparte de ellos mismos, veamos cómo podrían repartirse esos dos pulgares arriba:

\[CR_{m}^{n} = \frac{(m+n-1)!}{n!(m-1)!}\]

\[
\begin{array}{cccc}
	aa &    &    &    \\
	ab & bb &    &    \\
	ac & bc & cc &    \\
	ad & bd & cd & dd \\
\end{array}
\]

En este caso sí contamos con la diagonal principal de la matriz, pues admitimos valores repetidos.
También observamos que hemos metido la pelota de color Amarillo en la caja dos veces pero sólo tenemos una pelota de ese color.
¡Es lo que pasa por estudiar matemáticas y no física!
