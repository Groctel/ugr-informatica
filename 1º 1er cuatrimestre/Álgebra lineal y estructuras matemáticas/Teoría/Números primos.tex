\section{Números primos, teorema fundamental de la aritmética}\label{numeros-primos-teorema-fundamental-de-la-aritmetica}

\subsection{Definición}

Decimos que un número $a \in\mathbb{Z}\backslash\{-1,0,1\}$ es irreducible si sus divisores son $\pm 1$ y $\pm a$.
Puesto que los números primos son irreducibles, ésta es la definición que se da de ellos en el instituto.
Sin embargo, ésta es una consecuencia de la propiedad con la que los vamos a identificar un número primo $p$:

\[p|ab \Rightarrow p|a \lor p|b\]

Por ejemplo, tomemos el número $p=2$ y los números $a=6$ y $b=3$.
Tenemos que $6 \cdot 3 = 18$ y que $2|18$ y a 6 (aunque no a 3), por lo que sabemos que 2 es un número primo.
Lo mismo ocurriría con $p=3$, $a=18$ y $b=9$.
Sin embargo, probemos con $p=4$, $a=6$ y $b=2$.
Tenemos que $6 \cdot 2 = 12$ y que $4|12$.
Sin embargo, no se da en ningún momento que $4|6$ o que $4|2$, por lo que 4 no es un número primo.
Esto ocurre porque 4 no es irreducible, ya que sus divisores son $\{1, -1, 2, -2, 4, -4\}$.

\subsection{Teorema fundamental de la aritmética}

Sea $a \geq 2 \in\mathbb{N}$, $a$ es primo o $a$ se expresa de forma única (salvo el orden y el signo) como producto de números primos.
Por ejemplo, para $a = 17$, $a$ es primo, pero para $a = 120$, $a = 120 = 2^3 \cdot 3 \cdot 5$ y los números 2, 3 y 5 son primos.

De esta forma, podemos definir el conjunto divisores $D$ de un número $a$, descrito como $D(a)$ como el conjunto resultante de todas las combinaciones posibles de los factores de dicho número $a$.
Por ejemplo, definimos así los divisores de 120:

\[D(120) = \{2^a \cdot 3^b \cdot 3^c : 0 \leq a \leq 3,\ 0 \leq b \leq 1,\ 0 \leq c \leq 1\}\]

Podemos determinar el tamaño de este conjunto (el número de divisores de $a$) a partir de la factorización multiplicando los exponentes de todos los factores de $a$ sumando previamente 1 a cada uno de ellos.
