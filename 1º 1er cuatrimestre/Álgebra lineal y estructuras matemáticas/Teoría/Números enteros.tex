\section{Números enteros}\label{numeros-enteros}

\subsection{$\mathbb{Z}$, el conjunto de los numeros enteros}

Al igual que los números naturales, los números enteros se pueden sumar y multiplicar con las mismas propiedades.
Sin embargo, este conjunto presenta propiedades que no tienen los números naturales.
A pesar de las repeticiones, las enumeramos todas para facilitar la lectura.

\subsubsection{Los números enteros se pueden sumar}

\[a,b \in\mathbb{Z} \Rightarrow a+b \in\mathbb{Z}, \forall a,b\]

La suma satisface las siguientes propiedades:

\begin{center}
\begin{tabular}{l l}
	\textbf{Propiedad}      & \textbf{Expresión}                                       \\
	\toprule
	Asociativa              & $(a+b) + c = a + (b+a), \forall a,b,c \in\mathbb{Z}$     \\
	Conmutativa             & $a+b = b+a, \forall a,b \in\mathbb{Z}$                   \\
	Elemento neutro (0)     & $a+0 = a, \forall a \in\mathbb{Z}$                       \\
	Elemento opuesto ($-a$) & $a + (-a) = 0, \forall a \in\mathbb{Z}$                  \\
	Cancelativa             & $a+b = a+c \Rightarrow b=c, \forall a,b,c \in\mathbb{Z}$ \\
\end{tabular}
\end{center}

La existencia del elemento opuesto nos indica que, al contrario que en los números naturales, la resta cumple todas las propiedades de la suma en los números enteros.

\subsubsection{Los números enteros se pueden multiplicar}

\[a,b \in\mathbb{Z} \Rightarrow a \cdot b \in\mathbb{Z}, \forall a,b\]

Al igual que en los números naturales, la división no cumple siempre estas propiedades en los números enteros.
La multiplicación satisface las siguientes propiedades:

\begin{center}
\begin{tabular}{l l}
	\textbf{Propiedad}  & \textbf{Expresión}                                                             \\
	\toprule
	Asociativa          & $(a \cdot b) \cdot c = a \cdot (b \cdot c), \forall a,b,c \in\mathbb{Z}$       \\
	Conmutativa         & $a \cdot b = b \cdot a, \forall a,b \in\mathbb{Z}$                             \\
	Elemento neutro (1) & $a \cdot 1 = a, \forall a \in\mathbb{Z}$                                       \\
	Cancelativa         & $a \cdot b = a \cdot c \Rightarrow b=c, \forall a,b,c \in\mathbb{Z}, a \neq 0$ \\
\end{tabular}
\end{center}

También tenemos que la suma es distributiva respecto al producto:

\[a \cdot (b+c) = a \cdot b + a \cdot c, \forall a,b,c \in\mathbb{Z}\]

\subsubsection{Los números enteros se pueden ordenar}

\[\exists c \in\mathbb{Z} : a+c = b \Rightarrow a \leq b, \forall a,b,c \in\mathbb{Z}\]

Esta relación de orden satisface las siguientes tres propiedades con nombre:

\begin{center}
\begin{tabular}{l l}
	\textbf{Propiedad} & \textbf{Expresión}                                                          \\
	\toprule
	Reflexiva          & $a \leq a, \forall a \in\mathbb{Z}$                                         \\
	Asimétrica         & $a \leq b \land b \leq a \Rightarrow a=b, \forall a,b \in\mathbb{Z}$        \\
	Transitiva         & $a \leq b \land b \leq c \Rightarrow a \leq c, \forall a,b,c \in\mathbb{Z}$ \\
	Orden total        & $a \leq b \lor b \leq a, \forall a,b \in\mathbb{Z}$                         \\
\end{tabular}
\end{center}

Esta relación también cumple las siguientes propiedades

\begin{itemize}
	\item $a \leq b \Rightarrow a+c \leq b+c, \forall a,b,c \in\mathbb{Z}$
	\item $a+c \leq n+p \Rightarrow a \leq b, \forall a,b,c \in\mathbb{Z}$
	\item $a \leq b \Rightarrow a \cdot c \leq b \cdot c, \forall a,b,c \in\mathbb{Z}$
	\item $a \cdot c \leq b \cdot c \Rightarrow a \leq c, \forall a,b,c \in\mathbb{Z}$
	\item $a \leq b \Rightarrow
		\begin{cases}
			ac \leq bc & \text{si } c \geq 0 \\
			ac \geq bc & \text{si } c \leq 0
		\end{cases}
		, \forall a,b,c \in\mathbb{Z}$
\end{itemize}

\subsubsection{Los números enteros tienen un valor absoluto}

Definimos la aplicación valor absoluto $|| : \mathbb{Z} \rightarrow \mathbb{N}$ de la siguiente forma:

\[
|a| =
\begin{cases}
	a  & \text{si } a \geq 0 \\
	-a & \text{si } a < 0
\end{cases}
\]

Informalmente hablando, esta aplicación nos devuelve la \textit{distancia de $a$ al 0} y cumple las siguientes propiedades:

\begin{itemize}
	\item $|a| = 0 \Leftrightarrow a = 0, \forall a \in\mathbb{Z}$
	\item $|a \cdot b| = |a| \cdot |b|, \forall a,b \in\mathbb{Z}$
	\item $|a+b+| \leq |a| + |b|, \forall a,b \in\mathbb{Z}$
	\item $|a| \leq b \Leftrightarrow -b \leq a \leq b, \forall a,b \in\mathbb{Z}$
\end{itemize}

\subsubsection{Existe un algoritmo para dividir los números enteros}

\[a,b \in\mathbb{Z} \Rightarrow \exists! c,r \in\mathbb{Z} : a = bc + r, 0 \geq r < |b|, b \neq 0\]

\subsection{Representación en complemento}

Para restarle a un número entero $a$ otro número entero $b$ realizamos la operación $a + (-b) = a - b$ y utilizamos el algoritmo de la resta.
Sin embargo, existe una forma más sencilla de computar esta operación que sólo requiere del algoritmo de la suma.
Esta forma es la representación de un número en complemento a otro.
Vamos a construir primero una intuición al funcionamiento de esta representación:

\begin{align*}
	143 - 98  = 45  \\
	100 - 98  = 2   \\
	143 + 2   = 145 \\
	144 - 100 = 45
\end{align*}

Para realizar la operación $143 - 98$ (o cualquier otra operación aparte del ejemplo) podemos realizar la resta directamente o podemos calcular cuánto le falta al sustraendo para llegar a 100, sumarlo y restarle 100.
Utilizamos este algoritmo de forma diaria al hacer cálculos mentales sin pensar demasiado en el proceso.
Vamos a formalizarlo para un caso general:

Dado un número $a \in\mathbb{Z}$, en una base $b \geq 2$, podemos representarlo en complemento a $b$ con el siguiente algoritmo:

\begin{itemize}
	\item Añadimos un 0 a la izquierda de $a$.
	\item Si $a$ es negativo:
	\begin{itemize}
		\item Para cada cifra $a_i$ de $a$, calculamos $(b-1) - a_i$ y la sustituimos.
		\item Sumamos 1 al resultado del paso anterior\footnote{%
			Si nos saltamos el tercer paso del algoritmo, hemos calculado $a$ en complemento a $b-1$.
		}.
	\end{itemize}
\end{itemize}

Vamos a representar -98 en complemento a 10 para demostrar que este algoritmo sirve para la el ejemplo anterior:

Primero, le añadimos un 0 a la izquierda: 098.
Segundo, a cada calculamos $9 - a_i$ para cada cifra $a_i$ de 098 y la sustituimos, quedándonos con 901.
Por último, le sumamos 1 y tenemos el 902, que es -98 expresado en complemento a 10.
Ahora para realizar la resta $143 - 98$ simplemente tenemos que realizar la suma $143 + 902$:

\begin{center}
\setlength{\tabcolsep}{1ex}
\begin{tabular}{c c c c}
  & 1 & 4 & 3 \\
+ & 9 & 0 & 2 \\
\midrule
1 & 0 & 4 & 5 \\
\end{tabular}
\end{center}

Despreciamos el acarreo del 1 y tenemos que $143 - 98 = 143 + 902 = 45$.

Esta operación es extremadamente útil en la ingeniería de computadores, ya que permite a los procesadores ahorrarse la implementación de un algoritmo de resta y usar las operaciones \texttt{xor} y \texttt{add}:

\begin{lstlisting}[language={[x86masm]Assembler}]
xorq %rax, %rax
addq %rbx, %rax
\end{lstlisting}

Esto ocurre porque la representación en complemento a 2 de un número binario puede simplificarse en intercambiar los ceros y los unos y sumar 1 al resultado:

\[-9 \rightarrow -1001 \rightarrow 10111\]

Para pasar de un número en binario expresado en complemento a 2 a un número decimal, sumamos todas las cifras y restamos la más significativas:

\begin{align*}
	101101 & = 1 + 4 + 8 - 32 = -19 \\
	01011  & = 1 + 2 + 8 - 0 = 11
\end{align*}

Al igual que en los números \textit{normales} no importa la cantidad de ceros a la izquierda, en los números en complemento a $b$ no importa la cantidad de $b-1$ a la izquierda.
