\section{Números combinatorios}\label{numeros-combinatorios}

\[m,n \in\mathbb{N} : m \leq n \Rightarrow {n \choose m} = C_{m}^{n} = \frac{n!}{m!(n-m)!}\]

A la hora de trabajar con combinaciones y muchas otras herramientas matemáticas nos encontraremos con valores de la forma $n \choose m$, que en inglés se expresan como \textit{n choose m} y es español no expresamos de ninguna forma especial pero me voy a tomar la licencia de llamarlos \textit{n escoge m}, porque realmente representan la operación de escoger $m$ elementos de un conjunto de cardinal $n$.

Aunque podemos calcularlos con la fórmula general, podemos utilizar también el triángulo de Pascal (conocido en Italia como triángulo de Tartaglia) para obtener de forma rápida sus valores.
Para generarlo, vamos a introducir en cada número combinatorio el valor $n$ correspondiente a su fila y vamos a rellenar cada fila de izquierda a derecha introduciendo todos los valores $m$ desde 0 hasta $n$:

\begin{center}
\begin{tabular}{c c c c c c c c c c c}
               &               &               &               &               & $0 \choose 0$ &               &               &               &               &               \\
               &               &               &               &               &               &               &               &               &               &               \\
               &               &               &               & $1 \choose 0$ &               & $1 \choose 1$ &               &               &               &               \\
               &               &               &               &               &               &               &               &               &               &               \\
               &               &               & $2 \choose 0$ &               & $2 \choose 0$ &               & $2 \choose 2$ &               &               &               \\
               &               &               &               &               &               &               &               &               &               &               \\
               &               & $3 \choose 0$ &               & $3 \choose 1$ &               & $3 \choose 2$ &               & $3 \choose 3$ &               &               \\
               &               &               &               &               &               &               &               &               &               &               \\
               & $4 \choose 0$ &               & $4 \choose 1$ &               & $4 \choose 2$ &               & $4 \choose 3$ &               & $4 \choose 4$ &               \\
               &               &               &               &               &               &               &               &               &               &               \\
 $5 \choose 0$ &               & $5 \choose 1$ &               & $5 \choose 2$ &               & $5 \choose 3$ &               & $5 \choose 4$ &               & $5 \choose 5$ \\
\end{tabular}
\end{center}

Tiene sentido trabajar con este triángulo por la siguiente propiedad de los números combinatorios:

\[{n+1 \choose m} = {n \choose m-1} + {n \choose m}\]

Visualmente, el valor de cada nodo del triángulo es el valor de la suma de sus padres izquierdo y derecho.
Con saber que ${0 \choose 0} = 1$ y que si un nodo sólo tiene un padre su valor es el de éste, podemos desarrollar el triángulo para calcular rápidamente los valores de los números combinatorios:

\begin{center}
\begin{tabular}{c c c c c c c c c c c}
   &   &   &   &    & 1 &    &   &   &   &   \\
   &   &   &   &    &   &    &   &   &   &   \\
   &   &   &   & 1  &   & 1  &   &   &   &   \\
   &   &   &   &    &   &    &   &   &   &   \\
   &   &   & 1 &    & 2 &    & 1 &   &   &   \\
   &   &   &   &    &   &    &   &   &   &   \\
   &   & 1 &   & 3  &   & 3  &   & 1 &   &   \\
   &   &   &   &    &   &    &   &   &   &   \\
   & 1 &   & 4 &    & 6 &    & 4 &   & 1 &   \\
   &   &   &   &    &   &    &   &   &   &   \\
 1 &   & 5 &   & 10 &   & 10 &   & 5 &   & 1 \\
\end{tabular}
\end{center}

Con este método, aparte de calcular con facilidad los números combinatorios, podemos ver conocer dos resultados de antemano:

\[{n \choose 0} = {n \choose m} = 1\]

\[{n \choose \pm1} = n\]
