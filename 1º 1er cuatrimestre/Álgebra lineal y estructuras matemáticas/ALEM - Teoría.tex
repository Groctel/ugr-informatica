\documentclass[10pt, a4paper]{aqademic}

\usepackage[spanish]{babel}
	\selectlanguage{spanish}

% Document packages

\usepackage{amsmath}
\usepackage{amsfonts}
\usepackage[type=CC, modifier=by-nc-sa, version=4.0]{doclicense}
\usepackage{graphicx}
\usepackage{mathtools}
\usepackage{tikz}
\usepackage{venndiagram}

% Document settings

\author{Atanasio José Rubio Gil}
\title{Álgebra lineal y estructuras matemáticas}

\AqSetChapter{Tema}

\newcommand\Division[2]{%
	$\strut#1$\kern.25em\smash{\raise-.35ex\hbox{\rule{0.4pt}{2ex}}}$\mkern-0.72mu
	\underline{\,#2}$
}

\newcount\total
\newcount\lasttotal
\newcount\targetbase

% https://tex.stackexchange.com/questions/107637/repeated-division-converting-from-base-10-to-another-base
\def\CambioDeBase#1#2{%
	\begin{tikzpicture}[every node/.style={minimum width=1cm, minimum height=0.5cm}, x=1cm,y=0.5cm]
	%
	\total=#1%
	\targetbase=#2
	\def\newnumber{}
	%
	\pgfmathloop
	\ifnum\total<1
	\else
		%
		\ifnum\pgfmathcounter>1
			\node at (\pgfmathcounter, -\pgfmathcounter+1) (tmp) {\the\targetbase};
			\draw (tmp.north west) |- (tmp.south east);
			%
			\node at (\pgfmathcounter-1, -\pgfmathcounter) (tmp) {\pgfmathparse{int(\total*\targetbase)}\pgfmathresult};
			\draw (tmp.south west) -- (tmp.south east);
			%
			\pgfmathparse{int(\lasttotal-\total*\targetbase)}%
			\let\digit=\pgfmathresult
			\node at (\pgfmathcounter-1, -\pgfmathcounter-1) [text=red] {\digit};
			\edef\newnumber{\digit\newnumber}
		\fi
		%
		\ifnum\total<\targetbase
			\edef\newnumber{\the\total\newnumber}
			\node at (\pgfmathcounter, -\pgfmathcounter) [text=red]  {\the\total};
		\else
			\node at (\pgfmathcounter, -\pgfmathcounter) {\the\total};
		\fi
		\lasttotal=\total
		\divide\total by\targetbase
	\repeatpgfmathloop
	\node [anchor=west] at (1, -\pgfmathcounter-1) {$#1=\newnumber_{\the\targetbase}$};
	\end{tikzpicture}
}

\definecolor{light-gray}{gray}{0.75}

% Document composition

\begin{document}

\AqMaketitle[%
	cover    = logo-ugr.png,
	org      = Grado en Ingeniería Informática,
	subtitle = Teoría,
	url      = https://github.com/Groctel/ugr-informatica
]

\begin{titlepage}

\newgeometry{%
	left=5cm,
	right=5cm
}

\thispagestyle{empty}

\topskip0pt
\vspace*{\fill}

\begin{center}
	\textbf{\Huge{¡Gracias por apostar por el conocimiento libre!}}
\end{center}

\vspace{1cm}

Estos apuntes están compartidos libremente para que puedas estudiar de forma gratuita y sin bloques de publicidad molesta en tus páginas.
Su publicación y uso son un paso adelante en la idea de que el conocimiento debería ser compartido de forma libre y gratuita para todos sin ningún tipo de impedimentos o distracciones generadas por muros de pago.

Como autor, no recibo ningún tipo de ingreso por la confección y publicación de este material.
Si quieres colaborar y ayudarme a seguir ofreciéndote facilidades para estudiar, puedes unirte al proyecto a través del enlace de la portada o enviarme una propina mediante el siguiente botón o su enlace:

\vspace{0.5cm}

\begin{center}
	\href{https://ko-fi.com/groctel}{\includegraphics[scale=0.15]{RecursosTeX/BuyMeACoffee_blue-2x.pdf}}
	\url{https://ko-fi.com/groctel}
\end{center}

\vspace{0.5cm}

Dado que este documento está compartido bajo una licencia \textbf{CC-by-nc-sa}, tienes la libertad de distribuirlo y adaptarlo bajo las siguientes condiciones:

\begin{itemize}
	\item\textbf{BY:} Debes darme crédito de forma adecuada con un enlace a la licencia e indicar si has hecho cambios.
	\item\textbf{NC:} No puedes hacer uso del mismo con propósitos comerciales.
	\item\textbf{SA:} Debes distribuir tus modificaciones bajo la misma licencia que el documento original.
\end{itemize}

Puedes leer la licencia completa pinchando en el enlace de la portada.

\vspace*{\fill}

\end{titlepage}

\tableofcontents

\chapter{Números naturales y números enteros}\label{numeros-naturales-y-numeros-enteros}
	\input{Teoría/Representacion de los números naturales.tex} \pagebreak
	\section{Números enteros}\label{numeros-enteros}

\subsection{$\mathbb{Z}$, el conjunto de los numeros enteros}

Al igual que los números naturales, los números enteros se pueden sumar y multiplicar con las mismas propiedades.
Sin embargo, este conjunto presenta propiedades que no tienen los números naturales.
A pesar de las repeticiones, las enumeramos todas para facilitar la lectura.

\subsubsection{Los números enteros se pueden sumar}

\[a,b \in\mathbb{Z} \Rightarrow a+b \in\mathbb{Z}, \forall a,b\]

La suma satisface las siguientes propiedades:

\begin{center}
\begin{tabular}{l l}
	\textbf{Propiedad}      & \textbf{Expresión}                                       \\
	\toprule
	Asociativa              & $(a+b) + c = a + (b+a), \forall a,b,c \in\mathbb{Z}$     \\
	Conmutativa             & $a+b = b+a, \forall a,b \in\mathbb{Z}$                   \\
	Elemento neutro (0)     & $a+0 = a, \forall a \in\mathbb{Z}$                       \\
	Elemento opuesto ($-a$) & $a + (-a) = 0, \forall a \in\mathbb{Z}$                  \\
	Cancelativa             & $a+b = a+c \Rightarrow b=c, \forall a,b,c \in\mathbb{Z}$ \\
\end{tabular}
\end{center}

La existencia del elemento opuesto nos indica que, al contrario que en los números naturales, la resta cumple todas las propiedades de la suma en los números enteros.

\subsubsection{Los números enteros se pueden multiplicar}

\[a,b \in\mathbb{Z} \Rightarrow a \cdot b \in\mathbb{Z}, \forall a,b\]

Al igual que en los números naturales, la división no cumple siempre estas propiedades en los números enteros.
La multiplicación satisface las siguientes propiedades:

\begin{center}
\begin{tabular}{l l}
	\textbf{Propiedad}  & \textbf{Expresión}                                                             \\
	\toprule
	Asociativa          & $(a \cdot b) \cdot c = a \cdot (b \cdot c), \forall a,b,c \in\mathbb{Z}$       \\
	Conmutativa         & $a \cdot b = b \cdot a, \forall a,b \in\mathbb{Z}$                             \\
	Elemento neutro (1) & $a \cdot 1 = a, \forall a \in\mathbb{Z}$                                       \\
	Cancelativa         & $a \cdot b = a \cdot c \Rightarrow b=c, \forall a,b,c \in\mathbb{Z}, a \neq 0$ \\
\end{tabular}
\end{center}

También tenemos que la suma es distributiva respecto al producto:

\[a \cdot (b+c) = a \cdot b + a \cdot c, \forall a,b,c \in\mathbb{Z}\]

\subsubsection{Los números enteros se pueden ordenar}

\[\exists c \in\mathbb{Z} : a+c = b \Rightarrow a \leq b, \forall a,b,c \in\mathbb{Z}\]

Esta relación de orden satisface las siguientes tres propiedades con nombre:

\begin{center}
\begin{tabular}{l l}
	\textbf{Propiedad} & \textbf{Expresión}                                                          \\
	\toprule
	Reflexiva          & $a \leq a, \forall a \in\mathbb{Z}$                                         \\
	Asimétrica         & $a \leq b \land b \leq a \Rightarrow a=b, \forall a,b \in\mathbb{Z}$        \\
	Transitiva         & $a \leq b \land b \leq c \Rightarrow a \leq c, \forall a,b,c \in\mathbb{Z}$ \\
	Orden total        & $a \leq b \lor b \leq a, \forall a,b \in\mathbb{Z}$                         \\
\end{tabular}
\end{center}

Esta relación también cumple las siguientes propiedades

\begin{itemize}
	\item $a \leq b \Rightarrow a+c \leq b+c, \forall a,b,c \in\mathbb{Z}$
	\item $a+c \leq n+p \Rightarrow a \leq b, \forall a,b,c \in\mathbb{Z}$
	\item $a \leq b \Rightarrow a \cdot c \leq b \cdot c, \forall a,b,c \in\mathbb{Z}$
	\item $a \cdot c \leq b \cdot c \Rightarrow a \leq c, \forall a,b,c \in\mathbb{Z}$
	\item $a \leq b \Rightarrow
		\begin{cases}
			ac \leq bc & \text{si } c \geq 0 \\
			ac \geq bc & \text{si } c \leq 0
		\end{cases}
		, \forall a,b,c \in\mathbb{Z}$
\end{itemize}

\subsubsection{Los números enteros tienen un valor absoluto}

Definimos la aplicación valor absoluto $|| : \mathbb{Z} \rightarrow \mathbb{N}$ de la siguiente forma:

\[
|a| =
\begin{cases}
	a  & \text{si } a \geq 0 \\
	-a & \text{si } a < 0
\end{cases}
\]

Informalmente hablando, esta aplicación nos devuelve la \textit{distancia de $a$ al 0} y cumple las siguientes propiedades:

\begin{itemize}
	\item $|a| = 0 \Leftrightarrow a = 0, \forall a \in\mathbb{Z}$
	\item $|a \cdot b| = |a| \cdot |b|, \forall a,b \in\mathbb{Z}$
	\item $|a+b+| \leq |a| + |b|, \forall a,b \in\mathbb{Z}$
	\item $|a| \leq b \Leftrightarrow -b \leq a \leq b, \forall a,b \in\mathbb{Z}$
\end{itemize}

\subsubsection{Existe un algoritmo para dividir los números enteros}

\[a,b \in\mathbb{Z} \Rightarrow \exists! c,r \in\mathbb{Z} : a = bc + r, 0 \geq r < |b|, b \neq 0\]

\subsection{Representación en complemento}

Para restarle a un número entero $a$ otro número entero $b$ realizamos la operación $a + (-b) = a - b$ y utilizamos el algoritmo de la resta.
Sin embargo, existe una forma más sencilla de computar esta operación que sólo requiere del algoritmo de la suma.
Esta forma es la representación de un número en complemento a otro.
Vamos a construir primero una intuición al funcionamiento de esta representación:

\begin{align*}
	143 - 98  = 45  \\
	100 - 98  = 2   \\
	143 + 2   = 145 \\
	144 - 100 = 45
\end{align*}

Para realizar la operación $143 - 98$ (o cualquier otra operación aparte del ejemplo) podemos realizar la resta directamente o podemos calcular cuánto le falta al sustraendo para llegar a 100, sumarlo y restarle 100.
Utilizamos este algoritmo de forma diaria al hacer cálculos mentales sin pensar demasiado en el proceso.
Vamos a formalizarlo para un caso general:

Dado un número $a \in\mathbb{Z}$, en una base $b \geq 2$, podemos representarlo en complemento a $b$ con el siguiente algoritmo:

\begin{itemize}
	\item Añadimos un 0 a la izquierda de $a$.
	\item Si $a$ es negativo:
	\begin{itemize}
		\item Para cada cifra $a_i$ de $a$, calculamos $(b-1) - a_i$ y la sustituimos.
		\item Sumamos 1 al resultado del paso anterior\footnote{%
			Si nos saltamos el tercer paso del algoritmo, hemos calculado $a$ en complemento a $b-1$.
		}.
	\end{itemize}
\end{itemize}

Vamos a representar -98 en complemento a 10 para demostrar que este algoritmo sirve para la el ejemplo anterior:

Primero, le añadimos un 0 a la izquierda: 098.
Segundo, a cada calculamos $9 - a_i$ para cada cifra $a_i$ de 098 y la sustituimos, quedándonos con 901.
Por último, le sumamos 1 y tenemos el 902, que es -98 expresado en complemento a 10.
Ahora para realizar la resta $143 - 98$ simplemente tenemos que realizar la suma $143 + 902$:

\begin{center}
\setlength{\tabcolsep}{1ex}
\begin{tabular}{c c c c}
  & 1 & 4 & 3 \\
+ & 9 & 0 & 2 \\
\midrule
1 & 0 & 4 & 5 \\
\end{tabular}
\end{center}

Despreciamos el acarreo del 1 y tenemos que $143 - 98 = 143 + 902 = 45$.

Esta operación es extremadamente útil en la ingeniería de computadores, ya que permite a los procesadores ahorrarse la implementación de un algoritmo de resta y usar las operaciones \texttt{xor} y \texttt{add}:

\begin{lstlisting}[language={[x86masm]Assembler}]
xorq %rax, %rax
addq %rbx, %rax
\end{lstlisting}

Esto ocurre porque la representación en complemento a 2 de un número binario puede simplificarse en intercambiar los ceros y los unos y sumar 1 al resultado:

\[-9 \rightarrow -1001 \rightarrow 10111\]

Para pasar de un número en binario expresado en complemento a 2 a un número decimal, sumamos todas las cifras y restamos la más significativas:

\begin{align*}
	101101 & = 1 + 4 + 8 - 32 = -19 \\
	01011  & = 1 + 2 + 8 - 0 = 11
\end{align*}

Al igual que en los números \textit{normales} no importa la cantidad de ceros a la izquierda, en los números en complemento a $b$ no importa la cantidad de $b-1$ a la izquierda.

	\section{Divisibilidad}\label{divisibilidad}

\subsection{Definición}

En \S\ref{representacion-de-los-numeros-naturales} vimos un algoritmo para dividir los números naturalesm que ampliamos para el caso de los números enteros en \S\ref{numeros-enteros}.
Esta división tiene en cuenta que el resto puede ser cualquier número natura o entero.
Sin embargo, no es lo mismo dividir un número entre otro y que un número divida a otro.
Cuando decimos que un número $a$ divide a un número $b$ lo expresamos como $a|b$ y lo definimos de la siguiente forma:

\[a|b \iff \exists c \in\mathbb{Z} : a = bc\]

También podemos interpretar esta relación como que $b$ es múltiplo de $a$ o que $a$ es un divisor de $b$.
Aunque esta última forma de expresarlo es muy similar a la primera que hemos dado, hace énfasis en que cdada número puede tener más de uno que lo divida.
Llamaremos a este conjunto de números sus divisores.

Esta relación satisface las siguientes propiedades:

\begin{itemize}
	\item $a|a$
	\item $a|0 \forall a \in\mathbb{Z}$
	\item $1|a \forall a \in\mathbb{Z}$
	\item $a|b \land a|c \Rightarrow a|b \pm c$
	\item $a|b \land b|c \Rightarrow a|c$
	\item $a|b \land b|a \Rightarrow a = \pm b$
	\item $a|b \Rightarrow a|bc$
	\item $a|b \iff b \mod a = 0 \forall a \neq 0$
\end{itemize}

\subsection{Máximo común divisor y mínimo común múltiplo}

Decimos que un número $d$ es un máximo común divisor de $a$ y $b$ si se satisfacen las siguientes condiciones:

\begin{align*}
	d|a \land d|b & \\
	c|a \land c|b & \Rightarrow c|d
\end{align*}

De forma contraria a la definición usual que se da en el instituto, hablamos de \textit{un} máximo común divisor en lugar de \textit{el} máximo común divisor.
Lo expresamos así porque los máximos divisores comunes son aquellos cuyo valor absoluto es el máximo, es decir, $-d$ también es un máximo común divisor de $a$ y $b$ porque $|-d| = |d|$.
Por ejemplo, los máximos comunes divisores de 10 y 6 son 2 y -2, aunque nos quedaremos con el positivo por conveniencia.

Esta relación satisface las siguientes propiedades:

\begin{itemize}
	\item $mcd(a,b) = mcd(a,-b) = mcd(-a,b) = mcd(-a,-b) = mcd(|a|,|b|)$
	\item $mcd(a,0) = |a|$
	\item $mcd(a,1) = 1$
	\item $a|b \Rightarrow mcd(a,b) = |a|$
	\item $mcd(a, mcd(b,c)) = mcd(mcd(a,b), c) = mcd(a,b,c)$
	\item $mcd(ac,bc) = mcd(a,b) \cdot |c|$
	\item $d|a \land d|b \Rightarrow mcd\big(\frac{a}{d},\frac{b}{d}\big) = \frac{mcd(a,b)}{|d|}$
\end{itemize}

Decimos que un número $m$ es un mínimo común múltiplo de $a$ y $b$ (nótese de nuevo que existe más de uno) si se satisfacen las siguientes condiciones:

\begin{align*}
	a|m \land b|m & \\
	a|n \land b|n & \Rightarrow m|n
\end{align*}

Una forma sencilla de encontrar el mínimo común múltiplo de dos números es calculando previamente su máximo común divisor:

\[mcm(a,b) = \frac{ab}{mcd(a,b)}\]

\subsection{Algoritmo de Euclides}

Para encontrar el máximo común divisor de dos números tenemos dos formas de proceder.
La primera es factorizar ambos números y tomar los factores comunes elevados al menor exponente.
Sin embargo, este proceso es extremadamente lento para números grandes (del orden de milenios) y aún no hemos definido cómo factorizar un número o los elementos con los que operamos para ello.
En su lugar, vamos a definr un algoritmo llamado \textit{algoritmo de Euclides} que nos servirá para encontrar el máximo común divisor de un número en un tiempo extremadamente bajo.
Lo expresamos en pseudocódigo de la siguiente forma:

\begin{lstlisting}[language=Python]
EUCLIDES(a,b):
	(a,b) = (|a|,|b|)
	while b != 0:
		(a,b) = (b, a mod b)
	return a
\end{lstlisting}

Por ejemplo, calculemos el $mcd(58,42)$:

\begin{align*}
	(a,b) & = (58, 42) \\
	(a,b) & = (42, 16) \\
	(a,b) & = (16, 10) \\
	(a,b) & = (10, 6) \\
	(a,b) & = (6, 4) \\
	(a,b) & = (4, 2) \\
	(a,b) & = (2, 0) \\
\end{align*}

En unas pocas operaciones hemos obtenido que $mcd(58,42) = 2$.
A la hora de trabajar con este algoritmo lo hacemos en una tabla para facilitarlos la interpretación de los resultados.
Organizamos esta tabla de forma que los restos de las operaciones estén a la izquierda y los cocientes a la derecha:

\begin{center}
\begin{tabular}{r r}
	$\boldsymbol{r}$ & $\boldsymbol{c}$ \\
	\toprule
	58               &                  \\
	42               &                  \\
	16               & 1                \\
	10               & 2                \\
	6                & 1                \\
	4                & 1                \\
	2                & 1                \\
	0                & 1                \\
\end{tabular}
\end{center}

Con esta organización podemos definir una forma sistemática de computar cada fila $i$:

\[r_{i-2} = r_{i-1} \cdot c_i + r_i\]

\subsection{Algoritmo extendido de Euclides: Identidad de Bezout}\label{algoritmo-extendido-de-euclides}

Vamos a extender el algoritmo de Euclides con la siguiente identidad conocida como la \textit{identidad de Bezout}:

\[d = mcd(a,b) \Rightarrow \exists u,v \in\mathbb{Z} : d = au + bv\]

La extensión de este algoritmo la haremos añadiendo dos columnas $u$ y $v$ que nos ayuden a calcular esta identidad para cada $r_i$ de la siguiente forma:

\[r_i = ua + vb\]

Para calcular estos números no tenemos que probar todas las combinaciones posibles, sino que tenemos una forma sistemática de obtenerlos de la misma forma que con los restos:

\begin{align*}
	u_i = u_{i-2} - c_i \cdot u_{i-1} \\
	v_i = v_{i-2} - c_i \cdot v_{i-1} \\
\end{align*}

Por ejemplo, veamos la extensión del algoritmo de Euclides para calcular $mcd(58,42)$:

\begin{center}
\begin{tabular}{r r r r}
	$\boldsymbol{r}$ & $\boldsymbol{c}$ & $\boldsymbol{u}$ & $\boldsymbol{v}$ \\
	\toprule
	58               &                  & 1                & 0                \\
	42               &                  & 0                & 1                \\
	16               & 1                & 1                & -1               \\
	10               & 2                & -2               & 3                \\
	6                & 1                & 3                & -4               \\
	4                & 1                & -5               & 7                \\
	2                & 1                & 8                & -11              \\
	0                & 2                & -21              & 29               \\
\end{tabular}
\end{center}

Para preparar el algoritmo debemos rellenar las dos primeras filas con $u=0, v=1$ y $u=1, v=0$ respectivamente.
Podemos verificar cómo para cada fila se cumple la identidad de Bezout:

\begin{align*}
	58 & = 58 \cdot 1 + 42 \cdot 0    \\
	42 & = 58 \cdot 0 + 42 \cdot 1    \\
	16 & = 58 \cdot 1 + 42 \cdot -1   \\
	10 & = 58 \cdot -2 + 42 \cdot 3   \\
	6  & = 58 \cdot 3 + 42 \cdot -4   \\
	4  & = 58 \cdot -5 + 42 \cdot 7   \\
	2  & = 58 \cdot 8 + 42 \cdot -11  \\
	0  & = 58 \cdot -21 + 42 \cdot 29 \\
\end{align*}

	\section{Números primos, teorema fundamental de la aritmética}\label{numeros-primos-teorema-fundamental-de-la-aritmetica}

\subsection{Definición}

Decimos que un número $a \in\mathbb{Z}\backslash\{-1,0,1\}$ es irreducible si sus divisores son $\pm 1$ y $\pm a$.
Puesto que los números primos son irreducibles, ésta es la definición que se da de ellos en el instituto.
Sin embargo, ésta es una consecuencia de la propiedad con la que los vamos a identificar un número primo $p$:

\[p|ab \Rightarrow p|a \lor p|b\]

Por ejemplo, tomemos el número $p=2$ y los números $a=6$ y $b=3$.
Tenemos que $6 \cdot 3 = 18$ y que $2|18$ y a 6 (aunque no a 3), por lo que sabemos que 2 es un número primo.
Lo mismo ocurriría con $p=3$, $a=18$ y $b=9$.
Sin embargo, probemos con $p=4$, $a=6$ y $b=2$.
Tenemos que $6 \cdot 2 = 12$ y que $4|12$.
Sin embargo, no se da en ningún momento que $4|6$ o que $4|2$, por lo que 4 no es un número primo.
Esto ocurre porque 4 no es irreducible, ya que sus divisores son $\{1, -1, 2, -2, 4, -4\}$.

\subsection{Teorema fundamental de la aritmética}

Sea $a \geq 2 \in\mathbb{N}$, $a$ es primo o $a$ se expresa de forma única (salvo el orden y el signo) como producto de números primos.
Por ejemplo, para $a = 17$, $a$ es primo, pero para $a = 120$, $a = 120 = 2^3 \cdot 3 \cdot 5$ y los números 2, 3 y 5 son primos.

De esta forma, podemos definir el conjunto divisores $D$ de un número $a$, descrito como $D(a)$ como el conjunto resultante de todas las combinaciones posibles de los factores de dicho número $a$.
Por ejemplo, definimos así los divisores de 120:

\[D(120) = \{2^a \cdot 3^b \cdot 3^c : 0 \leq a \leq 3,\ 0 \leq b \leq 1,\ 0 \leq c \leq 1\}\]

Podemos determinar el tamaño de este conjunto (el número de divisores de $a$) a partir de la factorización multiplicando los exponentes de todos los factores de $a$ sumando previamente 1 a cada uno de ellos.

	\section{Clases residuales módulo $\boldsymbol{m}$}\label{clases-residuales-modulo-m}

\subsection{Definición}

Sea $m \in \mathbb{N} \geq 2$, vamos definir el conjunto cociente $\mathbb{Z}_m$ a partir de la relación de equivalencia $a \equiv b$ (expresada como \textit{a es congruente con b}) de la siguiente forma:

\[a \equiv b \mod m \iff m|(b-a)\]

Este conjunto $\mathbb{Z}:m$ tiene $m$ elementos que denotaremos $\{{[0]}_m, {[1]}_m, \ldots, {[m-1]}_m\}$.
Cada uno de estos elementos ${[i]}_m$ representa todos los números $a \equiv i \mod m$ o, lo que es lo mismo, todos los números que dan resto $i$ al dividirlos entre $m$.
Por ejemplo, en $\mathbb{Z}_5$ tenemos que $3 = 8$, ya que dividir 3 entre 5 da el mismo resto que dividir 8 entre 5: 3.
A pesar de ser los mismo elementos, trabajaremos siempre con el menor, en este caso el 3, y lo denotaremos como $[3]$ para indicar que equivale a dicho número y todos los que den el mismo resto al dividir entre el cociente del conjunto.

Dentro de estos conjuntos podemos sumar y multiplicar los números que los componen con la facilidad de que tanto la suma como la multiplicación de dos elementos congruentes con $m$ es congruente con $m$, es decir, ${[a]}_m + {[b]}_m = {[a+b]}_m$ y ${[a]}_m{[b]}_m = {[ab]}_m$ sin importar los números $a$ y $b$ que elijamos.
Sin embargo, éste no es el caso de la división, ya que no se puede asegurar que $\frac{{[a]}_m}{{[b]}_m} = {[\frac{a}{b}]}_m$.
Esto es fácilmente comprobable en $\mathbb{Z}_5$, donde tenemos que $2 = 7$ pero $\frac{2}{3}$ no es lo mismo que $\frac{7}{3}$.
Por tanto, tendremos en cuenta que la división no está bien definida en estos conjuntos.

En estos conjuntos, las operaciones suma y producto verifican las siguientes propiedades que indican que $\mathbb{Z}_m$ es un anillo conmutativo:

\subsubsection{Propiedades de la suma}

\begin{center}
\begin{tabular}{l l}
	\textbf{Propiedad}     & \textbf{Expresión}                                                 \\
	\toprule
	Asociativa             & $(a+b) + c = a + (b+a), \forall a,b,c \in\mathbb{Z}_m$             \\
	Conmutativa            & $a+b = b+a, \forall a,b \in\mathbb{Z}_m$                           \\
	Elemento neutro (0)    & $a+0 = a, \forall a \in\mathbb{Z}_m$                               \\
	Elemento opuesto ($b$) & $\exists b \in\mathbb{Z}_m : a + b = 0, \forall a \in\mathbb{Z}_m$ \\
	Cancelativa            & $a+b = a+c \Rightarrow b=c, \forall a,b,c \in\mathbb{Z}_m$         \\
\end{tabular}
\end{center}

\subsubsection{Propiedades del producto}

\begin{center}
\begin{tabular}{l l}
	\textbf{Propiedad}  & \textbf{Expresión}                                                         \\
	\toprule
	Asociativa          & $(a \cdot b) \cdot c = a \cdot (b \cdot c), \forall a,b,c \in\mathbb{Z}_m$ \\
	Conmutativa         & $a \cdot b = b \cdot a, \forall a,b \in\mathbb{Z}_m$                       \\
	Elemento neutro (1) & $a \cdot 1 = a, \forall a \in\mathbb{Z}_m$                                 \\
\end{tabular}
\end{center}

El producto no tiene la propiedad cancelativa.
También tenemos que la suma es distributiva respecto al producto:

\[a \cdot (b+c) = a \cdot b + a \cdot c, \forall a,b,c \in\mathbb{Z}_m\]

\subsection{Unidades}

\[a \in\mathbb{Z}_m \text{ es una unidad } \iff \exists! b \in\mathbb{Z}_m : ab = 1\]

Por ejemplo, 2 es unidad en $\in\mathbb{Z}_3$ porque $2 \cdot 2 = 1$, pero no en $\mathbb{Z}_4$, pues todos sus múltiplos son 0 ó 2.
1 es una unidad en $\mathbb{Z}_m, \forall m \geq 2$.
Llamaremos al número $b$ que multiplicamos por $a$ para llegar a 1 el \textit{inverso de $a$} y lo representaremos como $a^{-1}$.

\subsubsection{Cálculo de inversos}

Para calcular el inverso de $a$ en $\mathbb{Z}_m$ tendremos que verificar primero si $mcd(a,m)=1$.
Hacemos esto usando el algoritmo de Euclides, que vamos a aprovechar y extender para encontrar el inverso.
Como vimos en~\ref{algoritmo-extendido-de-euclides}, podemos usar la identidad de Bezout para encontrar un par de factores $u$ y $v$ para $a$ y $b$ respectivamente que nos permitan dar una expresión de un $r_i$ en función de $a$ y $b$.
Para este caso únicamente tenemos que calcular el valor de $v$ para $r_i = 1$.
Por ejemplo, veamos el inverso de 4 en $\mathbb{Z}_{13}$:

\begin{center}
\begin{tabular}{r r r}
	$\boldsymbol{r}$ & $\boldsymbol{c}$ & $\boldsymbol{v}$ \\
	\toprule
	13               &                  & 0                \\
	4                &                  & 1                \\
	1                & 3                & -3               \\
\end{tabular}
\end{center}

El algoritmo extendido de Euclides nos dice que el inverso de 4 en $\mathbb{Z}_{13}$ es $-3 = 10$ y es fácil comprobar que $4 \cdot 10 = 40 = 13 \cdot 3 + 1$.
Llamaremos $\mathcal{U}(\mathbb{Z}_m)$ al conjunto de unidades, que podemos definir de la siguiente forma:

\[\mathcal{U}(\mathbb{Z}_m) = \{k \in\mathbb{N}, k < m: mcd(m,k) = 1\}, \forall m \in\mathbb{N}, m \geq 2\]

Dicho de otra forma, éste es el conjunto de todos los números naturales $k$ menores que $m$ que son primos relativos con éste.

\subsection{Teorema de Euler-Fermat}

Antes de formular este teorema, vamos a introducir la función indicatriz o $\varphi$ (phi) de Euler:

\[\varphi : \mathbb{N} \backslash \{0,1\} \rightarrow \mathbb{N}, \varphi(m) = |\mathcal{U}(\mathbb{Z}_m)|, \forall m \in\mathbb{N}, m \geq 2\]

Por ejemplo, $\varphi(6) = |\{0, 1, 4, 5\}| = 4$.
Nótese que para cualquier número primo $p$ se tiene que $\varphi(p) = |\{k \in\mathbb{N} : k < p\}| = p-1$.
El cálculo de $\varphi(p)$ presenta las dos siguientes propiedades:

\begin{itemize}
	\item
		$\varphi(p^n) = p^n - p^{n-1}$
	\item
		$mcd(a,b) = 1 \Rightarrow \varphi(ab) = \varphi(a) \cdot \varphi(b)$
\end{itemize}

Un ejemplo de la aplicación de estas propiedades sería calacular $\varphi(20)$.
Tenemos que $20 = 2^2 \cdot 5$, y que $mcd(2^2, 5) = 1$, por lo que podemos calcular el producto de las indicatrices de sus factores\footnote{%
	Para cualquier par de números primos $a$ y $b$ elevados a dos exponentes $n$ y $m$ cualquiera, se da que $mcd(a^m, b^n) = 1$, por lo que una estrategia para calcular $\varphi(x)$ es factorizarlo y calcular la indicatriz de sus factores.
}.
Tenemos entonces que $\varphi(20) = \varphi(2^2) \cdot \varphi(5) = (2^2 - 2^1) \cdot (5 - 5^0) = 2 \cdot 4 = 8$.

Pasamos ahora a formular el teorema de Euler-Fermat:

\[a \in\mathbb{Z}, m \in\mathbb{N}^* : mcd(a,m) = 1 \Rightarrow a^{\varphi(m)} \equiv 1 \mod m\]

Decir que $mcd(a,m) = 1$ es lo mismo que decir que $a \in\mathcal(\mathbb{Z}_m)$.
Este teorema nos ayuda a encontrar un inverso de $a$ en $\mathbb{Z}_m$ sin tener que probar con todos los valores por los que multiplicar $a$ hasta llegar a 1.
Cabe notar que no nos dice el primer valor que encontraríamos, sino que nos dice uno de ellos.
Habrá casos en los que podamos multiplicar $a$ por un número menor para encontrar su inverso, pero nos conformamos con este valor dada su facilidad para calcularlo.

Por ejemplo, vamos a calcular un inverso de 4 en $\mathbb{Z}_7$.
Primero tenemos que $mcd(4,7) = 1$, por lo que 4 tiene inverso en $\mathbb{Z}_7$.
Una forma de llegar al inverso sería ir probando y concluir que $2 \cdot 3 = 6 \equiv 1 \mod 5$.
Usando el teorema de Euler-Fermat tenemos que $2^{\varphi{5}} \equiv 1 \mod 5$.
Como 5 es primo, tenemos que $\varphi{5} = 4$, por lo que $2^4 \equiv 1 \mod 5$.
Es fácilmente comprobable que $16 \equiv 1 \mod 5$.
Por último, ¿cuál es el inverso de 2 en $\mathbb{Z}_5$?
$2^3$.

	\section{Sistemas de congruencias}\label{sistemas-de-congruencias}

\subsection{Ecuaciones en congruencias}

Como ya hemos visto, al realizar cálculos en módulo $m$, nos encontramos constantemente valores que son equivalentes.
Por ejemplo, en $\mathbb{Z}_3$ tenemos que $\{\ldots, -8, -5 ,-2, 1, 4, 7, \ldots\} = 1$ y, además, tenemos que todos estos valores valen exactamente $3x + 1, \forall x \in\mathbb{Z}$.
En esta sección vamos a trabajar con esta propiedad de la aritmética modular, a la que llamaremos congruencia.
Diremos que, para un caso general, $ax$ es congruente con $b$ módulo $m$ y lo expresaremos de la siguiente forma:

\[ax \equiv b \mod m\]

Las congruencias satisfacen las siguientes propiedades:

\begin{itemize}
	\item $a \equiv a' \mod m \land b \equiv b' \mod m \Rightarrow a'x \equiv b' \mod m$
	\item $d|a \land d|b \land d|c \Rightarrow \frac{a}{d}x \equiv \frac{b}{d} \mod \frac{m}{d}$
	\item $mcd(c,m) = 1 \Rightarrow cax \equiv cb \mod m$
\end{itemize}

La primera propiedad nos sirve para simplificar (y dificultar, aunque no es buena idea) las congruencias.
Por ejemplo, para la expresión $25x \equiv 7 \mod 3$, tenemos que $25 \equiv 1 \mod 3$ y que $7 \equiv 2 \mod 3$, por lo que podemos simplificar la expresión como $x \equiv 2 \mod 3$.
Cabe destacar que $n \equiv n \mod m, \forall n \in\mathbb{Z}$, por lo que no es necesario reducir todos ambos términos de la congruencia para poder reducir uno de ellos.

La segunda propiedad nos permite simplificar la congruencia dividiendo todos los términos entre un divisor común, preferiblemente su $mcd$.
De esta forma, podemos simplificar enormemente los cálculos y resolver más fácilmente la congruencia.

Por último, la tercera propiedad es la que nos permite encontrar la solución.
Si podemos encontrar un número $c$ que nos permita aislar la $x$ en su miembro, podremos resolver la congruencia.
Para que esto ocurra, se debe dar que $c=a^{-1}$, por lo que las congruencias sólo tienen solución si $a$ es unidad ($mcd(a,m)=1$) y si $mcd(a,m)|b$.
La congruencia quedaría como $aa^{-1}x \equiv ba^{-1} \mod m \Leftrightarrow x \equiv ba^{-1} \mod m$.

Como dijimos al principio de esta sección, las congruencias tienen soluciones infinitas.
Por ejemplo, $x \equiv 3 \mod 7$ tiene como soluciones $\{\ldots, -18, -11, -4, 3, 10, 17, \ldots\}$, es decir, cualquier valor $x = 7k + 3, \forall k \in\mathbb{Z}$.

Como ejemplo completo, resolvamos la siguiente congruencia:

\[14x \equiv 7 \mod 5\]

Para empezar, vamos a simplificar la congruencia.
Tenemos que $14 \equiv 4 \mod 5$ y que $7 \equiv 2 \mod 5$, por lo que $4x \equiv 2 \mod 5$.
Es fácil ver que $mcd(4,5)|2$, ya que el 1 divide a todos los números, por lo que sabemos que la congruencia tiene solución.
Nos falta encontrar $4^{-1} \mod 5$ para poder resolverla.
Sin entrar en los cálculos, tenemos que este número es 4, ya que $4 \cdot 4 = 16 \equiv 1 \mod 5$.
Sustituimos y tenemos que $x \equiv 8 \mod 5$.
Reducimos de nuevo para obtener que $x \equiv 3 \mod 5$ y ya tenemos nuestra solución, que es que $x = 5k + 3$.

\subsection{Sistemas de ecuaciones en congruencias}

Al igual que con las ecuaciones con las que trabajamos normalmente, las ecuaciones en congruencias se pueden agrupar para formar sistemas.
Veamos algunos ejemplos de resolución de estos sistemas sin entrar en cálculos complejos:

\[
\left\{
\begin{array}{ll}
	3x \equiv 0 \mod 5 & \iff x = 5k = \{\ldots, 0, 5, 10, 15, 20, 25, 30, \ldots\} \\
	4x \equiv 0 \mod 6 & \iff x = 3k = \{\ldots, 0, 3, 6, 9, 12, 15, 18, 21, 24, 27, 30, \ldots\}
\end{array}
\right.
\]

Las soluciones de este sistema son las soluciones comunes a ambas congruencias.
En este caso, las soluciones son 15, 30 y cualquier otro $x = 15k$.

\[
\left\{
\begin{array}{ll}
	3x \equiv 2 \mod 7 & \iff x = 7k + 3 \\
	5x \equiv 1 \mod 9 & \iff x = 9k + 2
\end{array}
\right.
\]

Ahora tenemos soluciones como $38 = 7\cdot 5 + 3 = 9 \cdot 4 + 2$ ó $101 = 7 \cdot 14 + 3 = 9 \cdot 11 + 2$.
Podríamos buscar una expresión que satisfaciera estas condiciones, pero no vamos a hacerlo para un caso particular sabiendo que existe una forma más sencilla que veremos a continuación.

\[
\left\{
\begin{array}{ll}
	3x \equiv 2 \mod 10 & \iff x = 10k + 4 \\
	5x \equiv 3 \mod 8  & \iff x = 8k + 5
\end{array}
\right.
\]

Podemos ver que este sistema no tiene solución, ya que las soluciones de la primera congruencia son todas pares y las de la segunda, impares.
Visto esto, vamos a dar un método para resolver sistemas de congruencias.
Usemos como ejemplo el siguiente sistema:

\[
\left\{
\begin{array}{l}
	2x \equiv 4 \mod 5 \\
	8x \equiv 7 \mod 13 \\
	5x \equiv 3 \mod 7
\end{array}
\right.
\]

Para resolver este sistema vamos a empezar por resolver $x$ para la primera congruencia e introducir el resultado en la segunda.
Luego resolveremos la segunda e introduciremos el resultado en la tercera y así en cadena hasta llegar al final.
Si en algún momento encontrásemos que una de las congruencias no tiene solución, determinamos el sistema no la tiene.
Vamos a resolver $x$ para la primera congruencia:

\[2x \equiv 4 \mod 5\]

Tenemos que $mcd(2,5)$ y que $2^{-1} = 3$, por lo que $6 \equiv 12 \mod 5$, que reducimos a $1 \equiv 2 \mod 5$.
Tenemos, por tanto, que $x = 5k_0 + 1$.
Dado este valor de $x$, lo introducimos en la segunda congruencia:

\[
\begin{array}{ll}
	8x          & \equiv 7  \mod 13 \iff \\
	8(5k_0 + 1) & \equiv 7  \mod 13 \iff \\
	40k_0 + 8   & \equiv 7  \mod 13 \iff \\
	k_0         & \equiv 12 \mod 13
\end{array}
\]

Tenemos, por tanto, que $k_0 = 13k_1 + 12$.
Introducimos este valor en la expresión de $x$, sustituyendo por $k_0$:

\[
	x = 5k_0 + 1 = 5(13k_1 + 12) + 1 = 65k_1 + 61
\]

Introducimos este valor en la tercera congruencia y resolvemos:

\[
\begin{array}{ll}
	5x            & \equiv 3 \mod 7 \iff \\
	5(65k_1 + 61) & \equiv 3 \mod 7 \iff \\
	325k_1 + 305  & \equiv 3 \mod 7 \iff \\
	3k_1          & \equiv 6 \mod 7
\end{array}
\]

Resolvemos la congruencia $3k_1 \equiv 6 \mod 7$.
Comprobamos que $mcd(3,7) = 1$ y que $3^{-1} = 5$, por lo que $k_1 \equiv 2 \mod 7$ y que $k_1 = 7k + 2$.
Introducimos esta última $k$ en la expresión de $x$ y la desarrollamos para calcular la solución del sistema:

\[x = 65k_1 + 61 = 65(7k + 2) + 61 = 455k + 191\]

	\section{Ecuaciones diofánticas}\label{ecuaciones-diofanticas}

Para concluir el tema de los números naturales y enteros, vamos a dar un método para resolver en $\mathbb{Z}$ ecuaciones de la forma $ax + by = c : a,b,c \in\mathbb{Z}$.
Debido a que el número de soluciones de estas ecuaciones (cuando la tienen) es infinito, daremos la solución de estas ecuaciones en función de $x$ o $y$.
Estas ecuaciones sólo tienen solución si $mcd(a,b)|c$ y la encontraremos mediante la congruencia $ax \equiv c \mod b$ o también $by \equiv c \mod a$.

Por ejemplo, vamos a resolver la siguiente ecuación:

\[42x + 15y = 402\]

Primero comprobamos si $mcd(42,15)|402$.
Tenemos que $mcd(42,15) = 3$ y que $402 = 134 \cdot 3$, por lo que esta condición se cumple.
Por tanto, pasamos a resolver la congruencia $42x \equiv 402 \mod 15$:

\[
\begin{array}{ll}
	42x & \equiv 402 \mod 15 \iff \\
	12x & \equiv 12 \mod 15 \iff  \\
	x   & \equiv 6 \mod 15
\end{array}
\]

Esto nos dice que $x = 15k + 6$.
Vamos a sustituir $x$ en la ecuación original para obtener el valor de $y$:

\[
\begin{array}{ll}
	42x + 15y         & = 402 \iff         \\
	42(15k + 6) + 15y & = 402 \iff         \\
	630k + 252 + 15y  & = 402 \iff         \\
	15y               & = -630k + 150 \iff \\
	y                 & = \frac{-630k + 150}{15} = -42k + 10
\end{array}
\]




\chapter{Cuerpos finitos}\label{cuerpos-finitos}
	\section{Definición}\label{cuerpos-finitos-definicion}

\subsection{Anillos conmutativos}\label{anillos-conmutativos}

Para dar la definición de un cuerpo debemos dar primero la definición de un anillo conmutativo, ya que la primera es un caso particular de la segunda.
Decimos que un conjunto $A$ tiene estructura de anillo conmunitativo si define las siguientes operaciones:

\[
	\begin{array}{rl}
		\text{Operación suma: }    & A \times A \xrightarrow{+} A     \\
                                 & (a,b) \mapsto a + b              \\
		\text{Operación producto } & A \times A \xrightarrow{\cdot} A \\
                                 & (a,b) \mapsto a \cdot b          \\
\end{array}
\]

En esta estructura, la suma y el producto satisface las siguientes propiedades:

\subsubsection{Propiedades de la suma}

\begin{center}
\begin{tabular}{l l}
	\textbf{Propiedad}     & \textbf{Expresión}                           \\
	\toprule
	Asociativa             & $(a+b) + c = a + (b+a), \forall a,b,c \in A$ \\
	Conmutativa            & $a+b = b+a, \forall a,b \in A$               \\
	Elemento neutro (0)    & $a+0 = a, \forall a \in A$                   \\
	Elemento opuesto ($b$) & $\exists b : a + b = 0, \forall a \in A$     \\
\end{tabular}
\end{center}

\subsubsection{Propiedades del producto}

\begin{center}
\begin{tabular}{l l}
	\textbf{Propiedad}  & \textbf{Expresión}                                               \\
	\toprule
	Asociativa          & $(a \cdot b) \cdot c = a \cdot (b \cdot c), \forall a,b,c \in A$ \\
	Conmutativa         & $a \cdot b = b \cdot a, \forall a,b \in A$                       \\
	Elemento neutro (1) & $a \cdot 1 = a, \forall a \in A$                                 \\
\end{tabular}
\end{center}

La suma también debe ser distributiva respecto al producto:

\[a \cdot (b+c) = a \cdot b + a \cdot c, \forall a,b,c \in A\]

\subsection{Existencia de un inverso}

Para que un anillo conmutativo tenga estructura de cuerpo debe cumplirse una última propiedad:

\[\exists b : a \cdot b = 1, \forall a \in A, a \neq 0\]

Esta propiedad indica que todo elemento $a \in A$ tiene inverso, como es el caso de $\mathbb{C}$, $\mathbb{Q}$, $\mathbb{R}$ y $\mathbb{Z}_p$ para números primos $p$.
Como vimos en \S\ref{clases-residuales-modulo-m}, llamaremos unidades a los elementos que tienen inverso.

	\section{Generalidades sobre polinomios}\label{generalidades-sobre-polinomios}

En un anillo conmutativo $A$, definimos un polinomio con coeficientes en éste como una expresión de la siguiente forma:

\[a_{n}x^{n} + a_{n-1}x^{n-1} + \cdots + a_{1}x^{1} + a_{0}x^{0} : a_i \in A, x \notin A, n \in\mathbb{N}\]

Por ejemplo, $2x^2 + 3x + 1$ es un polinomio con coeficientes en $\mathbb{N}$.
No son polinomios expresiones como $x^{-3}$, $\cos(x)$ o $\sqrt{x}$.
Según esta definición, $3 + 5x^2$ tampoco lo sería, pero ignoraremos esto y trataremos las expresiones que tengan una equivalencia polinómica como tal ($5x^2 + 0x + 3$ en este caso).

Cada uno de los términos de un polinomio tiene un grado único, que es el valor $n$ al que está elevado su correspondiente $x$.
El valor que acompaña a dicho término es su coeficiente.
Por ejemplo, en $8x^3 + 2x + 3$, su coeficiente de grado 3 es 8 y su término independiente, 3.
Llamaremos coeficiente líder al coeficiente de mayor grado distinto de 0 y grado del polinomio al grado del coeficiente líder.
Por ejemplo, $3x^2 + 5x + 6$ es un polinomio de grado 2, pues éste es el grado de su coeficientes líder.
Si su grado es 0, diremos que es constante.
Si el coeficiente líder es 1 diremos que es mónico.
En el caso particular del polinomio 0, éste no tiene grado por la definición del coeficiente líder.
Para hablar del grado de un polinomio $p(x)$ usaremos la expresión $gr(p(x))$.

Llamaremos $A[x]$ al conjunto de todos los polinomios con coeficientes en $A$.

\subsection{Operaciones con polinomios}

Consideremos los dos siguientes polinomios en $A[x]$:

\[
\begin{array}{ll}
	p(x) & = a_{m}x^{m} + a_{m-1}x^{m-1} + \cdots + a_{1}x^{1} + a_{0}x^{0} \\
	q(x) & = b_{n}x^{n} + b_{n-1}x^{n-1} + \cdots + b_{1}x^{1} + b_{0}x^{0}
\end{array}
\]

Suponiendo que $m \leq n$, definimos las siguientes operaciones:

\subsubsection{Los polinomios se pueden sumar}

\[p(x) + q(x) = b_{n}x^{n} + \cdots + b_{m+1}x^{m+1} + (a_m + b_m)x^m + \cdots + (a_1 + b_1)x + (a_0 + b_0)\]

La suma satisface las siguientes propiedades:

\begin{center}
\begin{tabular}{l l}
	\textbf{Propiedad}         & \textbf{Expresión}                                                             \\
	\toprule
	Asociativa                 & $(p(x) + q(x)) + r(x) = p(x) + (q(x) + r(x)), \forall p(x),q(x),r(x) \in A[x]$ \\
	Conmutativa                & $p(x) + q(x) = q(x) + p(x), \forall p(x),q(x) \in A[x]$                        \\
	Elemento neutro (0)        & $p(x)+0 = p(x), \forall p(x) \in A[x] $                                        \\
	Elemento opuesto ($-p(x)$) & $p(x) + (-p(x)) = 0, \forall p(x) \in A[x]$                                    \\
\end{tabular}
\end{center}

\subsubsection{Los polinomios se pueden multiplicar}

\[
\begin{array}{ll}
	p(x) \cdot c_{k}x^{k} & = a_{n}c_{k}x^{k+n} + a_{n_1}c_{k}x^{k+n-1} + \cdots + a_{1}c_{k}x^{k+1} + a_{0}c_{k}x^{k} \\
	p(x) \cdot q(x)       & = p(x) \cdot q_n(x) + p(x) \cdot q_{n-1}(x) + \cdots + p(x) \cdot q_1(x) + p(x) \cdot q_0(x)
\end{array}
\]

El producto satisface las siguientes propiedades:

\begin{center}
\begin{tabular}{l l}
	\textbf{Propiedad}  & \textbf{Expresión}                                                                             \\
	\toprule
	Asociativa          & $(p(x) \cdot q(x)) \cdot r(x) = p(x) \cdot (q(x) \cdot r(x)), \forall p(x),q(x),r(x) \in A[x]$ \\
	Conmutativa         & $p(x) \cdot q(x) = q(x) \cdot r(x), \forall p(x),q(x) \in A[x]$                                \\
	Elemento neutro (1) & $p(x) \cdot 1 = p(x), \forall p(x) \in a[x]$                                                   \\
\end{tabular}
\end{center}

También tenemos que la suma es distributiva respecto al producto:

\[p(x) \cdot (q(x) + r(x)) = p(x) \cdot q(x) + p(x) \cdot r(x), \forall p(x),q(x),r(x) \in A[x]\]

Podemos calcular el producto de dos polinomios mediante el algoritmo de la multiplicación que aprendimos en primaria:

\[
\begin{array}{ll}
	p(x) & = 5x^3 + 3x^2 - x  + 8 \\
	q(x) & = 2x^3        - 5x + 3
\end{array}
\]

\begin{center}
\setlength{\tabcolsep}{1ex}
\begin{tabular}{c c c c c c c}
	   &   &          & 5   & 3  & -1  & 8  \\
	   &   & $\times$ & 2   & 0  & -5  & 3  \\
	\midrule
	   &   &          & 15  & 9  & -3  & 24 \\
	   &   & -25      & -15 & 5  & -40 &    \\
	10 & 6 & -2       & 16  &    &     &    \\
	\midrule
	10 & 6 & -27      & 16  & 14 & -43 & 24 \\
\end{tabular}
\end{center}

\[p(x) \cdot q(x) = 10x^6 + 6x^5 - 27x^4 + 16x^3 + 14x^2 - 43x + 24\]

Por supuesto, también podemos multiplicar un polinomio por un término constante $k$:

\[p(x) \cdot k = a_{m}k^{m} + a_{m-1}k^{m-1} + \cdots + a_{1}k^{1} + a_{0}k^{0} \]

\subsubsection{Los polinomios se pueden dividir}

\[p(x),q(x) \in A[x] \Rightarrow \exists! c(x),r(x) \in A[x] : p(x) = q(x) \cdot c(x) + r(x), 0 \leq gr(r(x)) \leq gr(q(x)), q(x) \neq 0\]

Podemos utilizar el algoritmo de la división \textit{con caja} que aprendimos en primaria, pero vamos a utilizar un algoritmo muchísimo más eficiente que es una generalización del ya conocido algoritmo de Ruffini.

\subsection{Algoritmo de Horner}

\subsubsection{División entre polinomios mónicos de grado 1: regla de Ruffini}

Vamos a repasar la regla de Ruffini.
Para dividir un polinomio $p(x)$ entre un polinomio $q(x)$ mónico, es decir, $q(x) = x + a$, utilizamos una tabla en la que introducimos los coeficientes de $p(x)$ en la fila superior y el término independiente de $q(x)$ cambiado de signo a la izquierda de la segunda fila:

\begin{center}
\setlength{\tabcolsep}{1ex}
\begin{tabular}{r | r r r r r}
	     & $p_m(x)$ & $p_{m-1}(x)$ & $\cdots$ & $p_1(x)$ & $p_0(x)$ \\
	$-a$ &          &              &          &          &          \\
	\hline
	     &          &              &          &          &
\end{tabular}
\end{center}

Luego, vamos rellenando la última fila de izquierda a derecha \textit{bajando} cada uno de los coeficientes sumado con el elemento de la fila intermedia si hubiese, multiplicando por $-a$ y almacenando el resultado en la fila intermedia:

\begin{center}
\setlength{\tabcolsep}{1ex}
\begin{tabular}{r | r r r r r}
        & $p_m(x)$ & $p_{m-1}(x)$              & $\cdots$ & $p_1(x)$ & $p_0(x)$                                            \\
	$-a$ &          & $p_m(x)(-a)$              & $\cdots$ & $\ddots$ & $((p_m(x)(-a) + p_{m-1}(x))(-a) \cdots p_1(x))(-a)$ \\
	\hline
        & $p_m(x)$ & $p_m(x)(-a) + p_{m-1}(x)$ & $\cdots$ & $\cdots$ & $((p_m(x)(-a) + p_{m-1}(x))(-a) \cdots p_1(x))(-a) + p_0(x)$
\end{tabular}
\end{center}

Por ejemplo, vamos a dividir los dos siguientes polinomios:

\[
\begin{array}{ll}
	p(x) & = 4x^3 + 2x^2 - 5x + 3 \\
	q(x) & =                x - 2
\end{array}
\]

\begin{center}
\setlength{\tabcolsep}{1ex}
\begin{tabular}{r | r r r r}
	  & 4 & 2  & $-5$ & 3  \\
	2 &   & 8  & 20   & 30 \\
	\hline
	  & 4 & 10 & 15   & 33
\end{tabular}
\end{center}

\[
\begin{array}{ll}
	c(x) & = 4x^2 + 10x - 15 \\
	r(x) & =              33
\end{array}
\]

\subsubsection{División entre polinomios mónicos de grado mayor que 1}

Para dividir entre polinomios mónicos de grado mayor que 1 vamos a usar el algoritmo de Horner, que tiene una forma de proceder similar a la regla de Ruffini, pero con las siguientes modificaciones:

\begin{itemize}
	\item
		Añadiremos los coeficientes del polinomio divisor cambiados de signo y ordenados de arriba a abajo en orden descendiende de su grado.
	\item
		Cada vez que operemos con el valor de la fila inferior y lo multipliquemos por los términos de la columna de la izquierda lo haremos en orden descendente y cada valor lo introduciremos en la columna siguiente a la anterior.
	\item
		Frenaremos cuando añadamos el producto de la penúltima con la fila inferior a la última columna, tras lo cual \textit{bajaremos} el resto de números y finalizaremos el algoritmo.
\end{itemize}

Por ejemplo, vamos a dividir los dos siguientes polinomios:

\[
\begin{array}{ll}
	p(x) & = 3x^3 + 5x^2 - 2x + 1 \\
	q(x) & =         x^2 + 2x - 3
\end{array}
\]

\begin{center}
\setlength{\tabcolsep}{1ex}
\begin{tabular}{r | r r r r}
	     & 3 & 5    & $-2$ & 1    \\
	$-2$ &   & $-6$ & 2    &      \\
	3    &   &      & 9    & $-3$ \\
	\hline
	     & 3 & $-1$ &  9   & $-2$
\end{tabular}
\end{center}

\[
\begin{array}{ll}
	c(x) & = 3x - 1 \\
	r(x) & = 9x - 2
\end{array}
\]

Como hemos frenado el algoritmo al multiplicar $-1 \cdot 3 = -3$, los elementos de la fila inferior (3 y $-1$) son los coeficientes del cociente y el resto de elementos que hemos \textit{bajado} (9 y $-2$), son los coeficientes del resto.

\subsubsection{Algoritmo de Horner para divisores no mónicos}

Si el grado del divisor fuese distinto de 1, simplemente tendríamos que multiplicar todos los números de la columna de la izquierda por el inverso del coeficiente líder y, al final, el cociente por el mismo valor.
Por ejemplo, vamos a dividir los dos siguientes polinomios:

\[
\begin{array}{ll}
	p(x) & = 2x^3 + 4x^2 - 3x + 5 \\
	q(x) & =        3x^2 + 4x - 3
\end{array}
\]

\begin{center}
\setlength{\tabcolsep}{1ex}
\begin{tabular}{r r | r r r r}
	                         &                & 2 & 4              & $-3$            & 5             \\
	$-4 \cdot \frac{1}{3} =$ & $-\frac{4}{3}$ &   & $-\frac{8}{3}$ & $-\frac{16}{9}$ &               \\
	$3 \cdot \frac{1}{3} =$  & 1              &   &                & 2               & $\frac{4}{3}$ \\
	\hline
	                         &                & 2 & $\frac{4}{3}$  & $-\frac{43}{9}$ & $\frac{19}{3}$
\end{tabular}
\end{center}

\[
\begin{array}{ll}
	c(x) & = \frac{1}{3}(2x + \frac{4}{3}) = \frac{2}{3}x + \frac{4}{9} \\
	r(x) & = -\frac{43}{9}x + \frac{19}{3}
\end{array}
\]


\chapter{Combinatoria}\label{combinatoria}
	\section{Métodos elementales de conteo}\label{metodos-elementales-de-conteo}

A lo largo de este tema vamos a ver cómo computar las diferentes formas en las que se puede dispone de varios elementos concretos y abstractos.
Antes de nada, vamos a redefinir las operaciones suma y producto para trabajar de forma más eficiente con ellas.

\subsection{Principio de la suma}\label{principio-de-la-suma}

\[A \cap B = \emptyset \Rightarrow |A \cup B| = |A| + |B|\]

Dados dos conjuntos $A$ y $B$ disjuntos, es decir, que $a \neq b, \forall a \in A, b \in B$, podemos contar los elementos de ambos conjuntos empezando a contar todos los de $A$ y seguir con los de $B$ hasta llegar al total.
Sin embargo, si conocemos la cardinalidad de los conjuntos, podemos tomar el atajo de simplemente sumarlas.
Vamos a aplicar este princpio a la combinatoria para encontrar el número de formas de las que se pueden realizar varias tareas siempre que dichas tareas sean incompatibles.
Por ejemplo, podemos usar este principio para contar las diferentes formas de elegir manzanas de dos cestos, ya que no podemos coger manzanas de dos cestos a la vez con una mano.
De esta forma, si en el cesto $A$ tenemos 4 manzanas y en el $B$, 3, podemos coger las manzanas de 7 formas diferentes.

\subsection{Principio de inclusión-exclusión}\label{principio-de-inclusion-exclusion}

Si los conjuntos sobre cuyas cardinalidades estamos aplicando el principio de la suma no son incompatibles, debemos usar la forma general del princpio de inclusión-exclusión para calcular la suma de las mismas:

\[|A \cup B| = |A| + |B| - |A \cap B|\]

\begin{figure}[h!]
\begin{center}
\begin{tikzpicture}
	\begin{scope}
		\clip             (180:1cm) circle (2cm);
		\fill[light-gray] (0:1cm)   circle (2cm);
	\end{scope}

	\draw (180:1cm) circle (2cm) node [text=black, label={[label distance=0.5cm]180:$A$}] {};
	\draw (0:1cm)   circle (2cm) node [text=black, label={[label distance=0.5cm]000:$B$}] {};
	\draw                        node [text=black]                                        {$A \cap B$};
\end{tikzpicture}
\ \ \ \ \ \ \ \ \ \ %
\begin{tikzpicture}
	\begin{scope}
		\clip             (90:1.5cm)  circle (2cm);
		\fill[light-gray] (180:1cm)   circle (2cm);
	\end{scope}
	\begin{scope}
		\clip             (180:1cm) circle (2cm);
		\fill[light-gray] (0:1cm)   circle (2cm);
	\end{scope}
	\begin{scope}
		\clip             (0:1cm)    circle (2cm);
		\fill[light-gray] (90:1.5cm) circle (2cm);
	\end{scope}
	\begin{scope}
		\clip             (90:1.5cm) circle (2cm);
		\clip             (180:1cm)  circle (2cm);
		\fill[white]      (0:1cm)    circle (2cm);
	\end{scope}

	\draw (90:1.5cm) circle (2cm) node [text=black, label={[label distance=0.75cm]090:$A$}]               {};
	\draw (180:1cm)  circle (2cm) node [text=black, label={[label distance=0.75cm]210:$B$}]               {};
	\draw (0:1cm)    circle (2cm) node [text=black, label={[label distance=0.75cm]330:$C$}]               {};
	\draw                         node [text=black, label={[label distance=1.10cm]060:$A \cap C$}]        {};
	\draw                         node [text=black, label={[label distance=1.10cm]120:$A \cap B$}]        {};
	\draw                         node [text=black, label={[label distance=0.55cm]270:$B \cap C$}]        {};
	\draw                         node [text=black, label={[label distance=0.20cm]090:$A \cap B \cap C$}] {};
\end{tikzpicture}
\end{center}
\caption{Suma de dos y tres conjuntos. Los sectores blancos se suman y los grises se restan.}
\end{figure}

Por ejemplo, seguimos este principio para calcular el número de animales que o bien viven en la sabana ($A$) o son herbívoros ($B$) pero no ambos ($A \cap B$).
Este princpio es escalable a más conjuntos.
En el caso de tres, por ejemplo, tenemos la siguiente expresión:

\[|A \cup B \cup C| = |A| + |B| + |C| - |A \cap B| - |A \cap C| - |B \cap C| + |A \cap B \cap C|\]

Vemos que tenemos que sumar las uniones de números impares de conjuntos y restar las pares, por lo que podemos dar una forma general para este principio:

\[|A_1 \cup A_2 \cup \cdots \cup A_n| = \sum_{k=1}^{n} {(-1)}^{k+1} \cdot \sum_{1 \leq i_1 < \cdots < i_k \leq n} |A_{i_1} \cap A_{i_2} \cap \cdots \cap A_{i_n}|\]

\subsection{Principio del producto}\label{principio-del-producto}

\[A_1 \times A_2 \times \cdots \times A_3| = |A_1| \cdot |A_2| \cdots |A_n|\]

Además de dar una forma clara de calcular la cardinalidad del producto vectorial de varios conjuntos, este principio nos sirve para resolver problemas dividiendo la tarea en tareas consecutivas más sencillas.
Si cada tarea $A$ se puede de $i$ formas, cada tarea $A_i$ y su siguiente tarea $A_{i+1}$ se pueden resolver de $|A_i| \cdot |A_{i+1}|$ formas.
Este principio es escalable a cualquier número de conjuntos o tareas que queramos calcular.

\begin{figure}[h!]
\begin{center}
\begin{tikzpicture}
	\draw[thin, black] (0,0) grid (5,5);
	\foreach \x in {1,2,3,4,5}
		\foreach \y in {1,2,3,4,5}
			\pgfmathsetmacro\ycalc{int(6-\y)}
			\node at (\x-0.5,\y-0.5) {\x:\ycalc};
\end{tikzpicture}
\end{center}
\caption{Este tablero de 5 filas y 5 columnas tiene 25 casillas.}
\end{figure}

\subsection{Principio del palomar}\label{principio-del-palomar}

Supongamos que queremos repartir $p$ palomas en un palomar con $n$ huecos.
Lo lógico es asignar una paloma a cada uno de los huecos.
Sin embargo, si tenemos más palomas que huecos ($p >n$), necesariamente tendremos que asignar más de una paloma a alguno de los huecos.
Este principio puede servirnos para resolver problemas en los que queramos dividir $p$ elementos en $n$ partes si $p > n$.

\subsubsection{Principio del palomar generalizado}

Dado un palomar con $n$ huecos en el que queremos repartir $p$ palomas, cada hueco deberá contener necesariamente al menos $\frac{p}{n}$ palomas.
Lógicamente, tomaremos el número entero inmediatamente superior a $\frac{p}{n}$.

	\section{Agrupaciones}\label{agrupaciones}

En este tema vamos a ver las diferentes formas de agrupar los elementos de un conjunto.
Para diferenciarlas, tendremos en cuenta el número de elementos de dicho conjunto que agrupamos y si nos importa o no su orden.
Como regla general podemos fijarnos en el siguiente diagrama para elegir el tipo de agrupación que queremos utilizar a la hora de resolver el ejercicio:

\[
	\text{¿Todos los elementos?}
	\begin{cases}
		\text{\textbf{SÍ\@:} Permutación. ¿Repeticiones?}
		\begin{cases}
			\text{\textbf{SÍ\@: }} PR_{n}^{k} = \frac{n!}{\prod_{i=1}^{k}(n_k!)} \\
			\text{\textbf{NO\@: }} P_n = n! \\
		\end{cases}
		\\
		\\
		\text{\textbf{NO\@:} ¿Ordenados?}
		\begin{cases}
			\text{\textbf{Sí\@:} Variación. ¿Repeticiones?}
			\begin{cases}
				\text{\textbf{SÍ\@: }} VR_{m}^{n} = m^n \\
				\text{\textbf{NO\@: }} V_{m}^{n} = \frac{m!}{(m-n)!} \\
			\end{cases}
			\\
			\\
			\text{\textbf{NO\@:} Combinación. ¿Repeticiones?}
			\begin{cases}
				\text{\textbf{SÍ\@: }} CR_{m}^{n} = \frac{(m+n-1)!}{n!(m-1)!} \\
				\text{\textbf{NO\@: }} C_{m}^{n} = \frac{m!}{n!(m-n)!}
			\end{cases}
		\end{cases}
	\end{cases}
\]

Supongamos que, como todo conocedor de la ciencia probabilística que se precie, somos los orgullosos poseedores de la típica caja con pelotas de colores\textsuperscript{TM}.
Las tres agrupaciones que podemos hacer con las pelotas de la caja son sacarlas una a una, ordenarlas en fila y meterlas de vuelta a la caja.
Todos los problemas de combinatoria se pueden reducir a operaciones con las pelotas de nuestra caja, por lo que vamos a utilizar esta analogía, junto con otras, para definir las diferentes agrupaciones.

\subsection{Variaciones}\label{variaciones}

Hablamos de variaciones cuando sacamos pelotas de la caja.
Cada vez que saquemos una pelota apuntaremos su color y el orden en el que la hemos sacado.
Por ejemplo, podríamos obtener la siguiente lista:

\begin{center}
\begin{tabular}{c l}
	\textbf{Orden} & \textbf{Color} \\
	\toprule
	1              & Amarillo       \\
	2              & Celeste        \\
	3              & Burdeos        \\
	4              & Dorado
\end{tabular}
\end{center}

Podría darse el caso de que, cada vez que tomásemos una pelota, la devolviésemos a la caja.
En este caso, podríamos obtener la siguiente lista:

\begin{center}
\begin{tabular}{c l}
	\textbf{Orden} & \textbf{Color} \\
	\toprule
	1              & Celeste        \\
	2              & Amarillo       \\
	3              & Celeste        \\
	4              & Burdeos
\end{tabular}
\end{center}

Vamos a dar una definición formal a esta idea intuitiva.

\subsubsection{Variaciones con repetición}

Llamamos variaciones con repetición de $n$ elementos tomados de $m$ en $m$ y lo expresamos formalmente como $VR_{n}^{m}$ a cada una de las formas en las que podemos elegir $m$ elementos de un conjunto de cardinal $n$ pudiendo haber repetidos.
Diferenciamos las elecciones según el elemento que hemos escogido y el orden en el que lo hemos hecho.
Por ejemplo, éstas son las diferentes formas en las que podemos escoger, con repeticiones, entre Atanasio, Bizancio, Clotilde y Delfina:

\[VR_{m}^{n} = m^n\]

\[
\begin{array}{cccc}
	aa & ba & ca & da \\
	ab & bb & cb & db \\
	ac & bc & cc & dc \\
	ad & bd & cd & dd
\end{array}
\]

Debemos tener en cuenta que la primera vez que elegimos a Atanasio o a Clotilde no tiene nada que ver con la segunda vez que lo hacemos, ya que unas representan las tuplas $\{1,a\}$ y $\{1,c\}$ mientras que las otras, las tuplas $\{2,a\}$ y $\{2,b\}$.

\subsubsection{Variaciones sin repetición}

Llamamos variaciones sin repetición de $n$ elementos tomados de $m$ en $m$ y lo expresamos formalmente como $V_{n}^{m}$ a cada una de las formas en las que podemos elegir $m$ elementos de un conjunto de $n$ no pudiendo haber repetidos.
Diferenciamos las elecciones según el elemento que hemos escogido y el orden en el que lo hemos hecho.
Vamos a utilizar el mismo ejemplo que antes pero sin repetir nombres:

\[V_{m}^{n} = \frac{m!}{(m-n)!}\]

\[
\begin{array}{cccc}
	   & ba & ca & da \\
	ab &    & cb & db \\
	ac & bc &    & dc \\
	ad & bd & cd &
\end{array}
\]

Dado que no podemos repetir elementos, no podemos elegir nunca $aa$, $bb$, $cc$ o $dd$.
Cada vez que elegimos elementos, el primero lo podemos elegir de $n$ formas, el segundo de $n - 1$, el tercero de $n - 2$ y así hasta $n - m + 1$ formas.
Simplificamos la expresión en un cociente de factoriales para obtener el total de variaciones sin repetición sobre un conjunto.

\subsection{Permutaciones}\label{permutaciones}

Hablamos de permutaciones cuando tomamos todas las pelotas de nuestra caja y las colocamos en orden sobre la mesa.
Al hacer esto, anotamos el orden en el que aparecen los colores, siendo cada una de las formas en las que podemos agrupar las pelotas una permutación.
Por ejemplo, podríamos obtener la siguiente lista:

\begin{itemize}
	\item
		Celeste, Burdeos, Dorado, Amarillo.
	\item
		Amarillo, Celeste, Dorado, Burdeos.
	\item
		Burdeos, Amarillo, Celeste, Dorado.
	\item
		Celeste, Dorado, Amarillo, Burdeos.
\end{itemize}

Para ordenar las pelotas hemos tenido que sacarlas todas de la caja y una forma de ordenarlas sería ir colocándolas según las sacásemos.
Esto funciona porque, realmente, las permutaciones son variaciones con o sin repetición donde $n = m$.
Al igual que tenemos variaciones con repetición, tenemos también permutaciones con repetición.
Éstas se presentan cuando tenemos varias pelotas del mismo color, de forma que la lista podría ser de la siguiente forma si tuviéramos dos pelotas celestes:

\begin{itemize}
	\item
		Celeste, Burdeos, Dorado, Amarillo, Celeste.
	\item
		Celeste, Burdeos, Dorado, Amarillo, Celeste.
	\item
		Amarillo, Celeste, Celeste, Dorado, Burdeos.
	\item
		Burdeos, Celeste, Amarillo, Celeste, Dorado.
	\item
		Celeste, Dorado, Amarillo, Celeste, Burdeos.
\end{itemize}

Aunque para el primer y segundo elemento las pelotas celestes se intercambian de sitio, ambas permutaciones son iguales porque estamos teniendo en cuenta el color de las pelotas, no su identidad original.
De la misma forma, al ordenar las letras de \texttt{algebra}, la primera y la última \texttt{a} deben considerarse iguales.

Vamos a dar una definición formal a esta idea intuitiva.

\subsubsection{Permutaciones sin repetición}

Llamamos permutaciones sin repetición de $n$ elementos y lo expresamos formalmente como $P_n$ a cada una de las formas en las que podemos ordenar los $n$ elementos de un conjunto no pudiendo haber repetidos.
Continuamos con Atanasio, Bizancio, Clotilde y Delfina para ver las diferentes formas en las que podemos arreglar el conjunto:

\[P_n = n!\]

\[
\begin{array}{cccc}
	abcd & bacd & cabd & dabc \\
	abdc & badc & cadb & dacb \\
	acbd & bcad & cbad & dbac \\
	acdb & bcda & cbda & dbca \\
	adbc & bdac & cdab & dcab \\
	adcb & bdca & cdba & dcba
\end{array}
\]

\subsubsection{Permutaciones con repetición}

Llamamos permutaciones con repetición de $n$ elementos con $k$ tipos de elementos y lo expresamos formalmente como $PR_{n}^{n_k}$ a cada una de las formas en las que podemos ordenar los $n$ elementos de un conjunto en el que podemos diferenciar $k$ tipos de elementos.
Vamos a ver de qué formas podemos ordenar a dos Atanasios, Bizancio y Clotilde:

\[PR_{n}^{k} = \frac{n!}{\prod_{1}^{k}(n_k!)}\]

\[
\begin{array}{cccc}
	abca & baca & caba & aabc \\
	abac & baac & caab & aacb \\
	acba & bcaa & cbaa & abac \\
	acab & bcaa & cbaa & abca \\
	aabc & baac & caab & acab \\
	aacb & baca & caba & acba
\end{array}
\]

Es fácil ver que tenemos varias respuestas iguales como dos $cbaa$, por ejemplo.
Para calcular el total de permutaciones debemos tener en cuenta el número total de elementos ($n = 4$) y el número de elementos de cada tipo.
El número de elementos de cada tipo $i$ se representa como $n_i$, por lo que lo introducimos en la fórmula general mostrada anteriormente.
Vamos a desarrollarla para este caso:

\[PR_{4}^{3} = \frac{4!}{2! \cdot 1! \cdot 1!} = \frac{24}{2} = 12\]

\subsection{Combinaciones}\label{combinaciones}

Hablamos de combinaciones cuando metemos pelotas en la caja.
No nos importa el orden en que las vamos metiendo, sino el resultado final, que son las pelotas que hay en la caja.
Por ello, una caja con las pelotas de color Amarillo y Celeste es lo mismo que otra con las pelotas de color Celeste y Amarillo.
Por tanto, cuando trabajemos con combinaciones lo haremos igual que si trabajásemos con conjuntos (o multiconjuntos en caso de que haya repeticiones).

Vamos a dar una definición formal a esta idea intuitiva.

\subsubsection{Combinaciones sin repetición}

Llamamos combinaciones sin repetición de $n$ elementos tomados de $m$ en $m$ y lo expresamos formalmente como $C_{m}^{n}$ a cada una de las formas en las que podemos seleccionar $m$ elementos de un conjunto de cardinal $n$ no pudiendo haber repetidos.
Diferenciamos las elecciones únicamente por el elemento escogido, nunca por el orden en el que lo hemos hecho.
Por ejemplo, Atanasio, Bizancio, Clotilde y Delfina están a la espera del resultado de un sorteo para un viaje a la Luna que escogerá a dos de ellos para embarcarse.
Vamos a ver de qué formas formas pueden salir elegidos:

\[C_{m}^{n} = \frac{m!}{n!(m-n)!}\]

\[
\begin{array}{cccc}
	ab &    &    \\
	ac & bc &    \\
	ad & bd & cd \\
\end{array}
\]

Podemos ver esta tabla como los elementos de una matriz $4 \times 4$ inferiores a la diagonal principal.
Los elementos de la diagonal principal serían $aa$, $bb$, $cc$ y $dd$ y, sobre ella, estarían los mismos elementos que vemos pero invertidos (el elemento $(2,1) = ab$ sería el $(1,2) = ba$).
Como no nos importa el orden y no tenemos repeticiones, nos quedamos con sólo una de las dos mitades\footnote{%
	Este razonamiento es el mismo que aplicamos al programar ejercicios de cálculo de rutas con una matriz de adyacencia:
	La distancia de una ciudad $i$ a $i$ es 0 y no la contamos y la distancia de $i$ a $j$ es la misma que de $j$ a $i$, por lo que sólo necesitamos reservar media matriz.
}.

\subsubsection{Combinaciones sin repetición}

Llamamos combinaciones sin repetición de $n$ elementos tomados de $m$ en $m$ y lo expresamos formalmente como $CR_{m}^{n}$ a cada una de las formas en las que podemos seleccionar $m$ elementos de un conjunto de cardinal $n$ pudiendo haber repetidos.
Diferenciamos las elecciones únicamente por el elemento escogido, nunca por el orden en el que lo hemos hecho.
Volvamos al ejemplo viaje espacial.
Los organizadores del sorteo le han hecho la jugada a los participantes y van a tener que competir por el puesto recolectando \textit{me gustas}.
Como sólo tienen dos amigos aparte de ellos mismos, veamos cómo podrían repartirse esos dos pulgares arriba:

\[CR_{m}^{n} = \frac{(m+n-1)!}{n!(m-1)!}\]

\[
\begin{array}{cccc}
	aa &    &    &    \\
	ab & bb &    &    \\
	ac & bc & cc &    \\
	ad & bd & cd & dd \\
\end{array}
\]

En este caso sí contamos con la diagonal principal de la matriz, pues admitimos valores repetidos.
También observamos que hemos metido la pelota de color Amarillo en la caja dos veces pero sólo tenemos una pelota de ese color.
¡Es lo que pasa por estudiar matemáticas y no física!

	\section{Números combinatorios}\label{numeros-combinatorios}

\[m,n \in\mathbb{N} : m \leq n \Rightarrow {n \choose m} = C_{m}^{n} = \frac{n!}{m!(n-m)!}\]

A la hora de trabajar con combinaciones y muchas otras herramientas matemáticas nos encontraremos con valores de la forma $n \choose m$, que en inglés se expresan como \textit{n choose m} y es español no expresamos de ninguna forma especial pero me voy a tomar la licencia de llamarlos \textit{n escoge m}, porque realmente representan la operación de escoger $m$ elementos de un conjunto de cardinal $n$.

Aunque podemos calcularlos con la fórmula general, podemos utilizar también el triángulo de Pascal (conocido en Italia como triángulo de Tartaglia) para obtener de forma rápida sus valores.
Para generarlo, vamos a introducir en cada número combinatorio el valor $n$ correspondiente a su fila y vamos a rellenar cada fila de izquierda a derecha introduciendo todos los valores $m$ desde 0 hasta $n$:

\begin{center}
\begin{tabular}{c c c c c c c c c c c}
               &               &               &               &               & $0 \choose 0$ &               &               &               &               &               \\
               &               &               &               &               &               &               &               &               &               &               \\
               &               &               &               & $1 \choose 0$ &               & $1 \choose 1$ &               &               &               &               \\
               &               &               &               &               &               &               &               &               &               &               \\
               &               &               & $2 \choose 0$ &               & $2 \choose 0$ &               & $2 \choose 2$ &               &               &               \\
               &               &               &               &               &               &               &               &               &               &               \\
               &               & $3 \choose 0$ &               & $3 \choose 1$ &               & $3 \choose 2$ &               & $3 \choose 3$ &               &               \\
               &               &               &               &               &               &               &               &               &               &               \\
               & $4 \choose 0$ &               & $4 \choose 1$ &               & $4 \choose 2$ &               & $4 \choose 3$ &               & $4 \choose 4$ &               \\
               &               &               &               &               &               &               &               &               &               &               \\
 $5 \choose 0$ &               & $5 \choose 1$ &               & $5 \choose 2$ &               & $5 \choose 3$ &               & $5 \choose 4$ &               & $5 \choose 5$ \\
\end{tabular}
\end{center}

Tiene sentido trabajar con este triángulo por la siguiente propiedad de los números combinatorios:

\[{n+1 \choose m} = {n \choose m-1} + {n \choose m}\]

Visualmente, el valor de cada nodo del triángulo es el valor de la suma de sus padres izquierdo y derecho.
Con saber que ${0 \choose 0} = 1$ y que si un nodo sólo tiene un padre su valor es el de éste, podemos desarrollar el triángulo para calcular rápidamente los valores de los números combinatorios:

\begin{center}
\begin{tabular}{c c c c c c c c c c c}
   &   &   &   &    & 1 &    &   &   &   &   \\
   &   &   &   &    &   &    &   &   &   &   \\
   &   &   &   & 1  &   & 1  &   &   &   &   \\
   &   &   &   &    &   &    &   &   &   &   \\
   &   &   & 1 &    & 2 &    & 1 &   &   &   \\
   &   &   &   &    &   &    &   &   &   &   \\
   &   & 1 &   & 3  &   & 3  &   & 1 &   &   \\
   &   &   &   &    &   &    &   &   &   &   \\
   & 1 &   & 4 &    & 6 &    & 4 &   & 1 &   \\
   &   &   &   &    &   &    &   &   &   &   \\
 1 &   & 5 &   & 10 &   & 10 &   & 5 &   & 1 \\
\end{tabular}
\end{center}

Con este método, aparte de calcular con facilidad los números combinatorios, podemos ver conocer dos resultados de antemano:

\[{n \choose 0} = {n \choose m} = 1\]

\[{n \choose \pm1} = n\]


\end{document}
