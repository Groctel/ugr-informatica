# COMPILACIÓN DE PROGRAMAS

## 0 - UTILIDADES

| **ar** | **g++** |
| :---: | :---: |
| **gcc** | **makefile** |

`ar` → Crea, modifica y extrae datos de archivos.

- `ar [lib]` → Crea la biblioteca [lib].
- `ar -r` → Inserta los archivos indicados en la biblioteca reemplazando los ya existentes del mismo origen en caso de que los hubiera.
- `ar -s` → Añade un índice al archivo o actualiza el ya existente en caso de que lo hubiera.
- `ar -v` → Verboso.

`g++` → Compila programas escrito en Fortran y C y derivados.
- `g++ [file]` → Compila el archivo [file].
- `g++ -c [file]` → Crea el módulo objeto `[file].o`.
- `g++ -I[folder]` → Busca bibliotecas para enlazarlas con módulos objeto en el directorio [folder].
- `g++ -L./` → Busca bibliotecas para enlazarlas con módulos objeto.
- `g++ -l[lib]` → Utiliza la bibioteca [lib] precedida de lib (`-ltrig` sería `libtrig.a`) para el enlazador del programa.
- `g++ -o [string] [file]` → Compila el archio [file] con el nombre [string].

`make` → Crea un archivo ejecutable a partir de un archivo `makefile`.
- `make -f [name]` → Crea un archivo ejecutable a partir del archivo `makefile` [name].

## 1 - INTRODUCCIÓN A LA COMPILACIÓN DE PROGRAMAS CON GCC Y G++

Los compiladores son programas capaces de generar un programa ejecutable (escrito en lenguaje máquina binario) a partir de archivos escritos en código fuente mediante un proceso de traducción e interpretación.

**GCC** es un compilador integrado del proyecto GNU para C, C++, Objective C y Fortran. La siguiente imagen muestra un proceso normal de compilación en C y C++:

# GCC.PNG

El proceso consta de varios módulos que se van ejecutando en el orden del algoritmo de la imagen:
- **Preprocesador:** Acepta archivos con extensiones de C o C++, como son `.c`, `.cpp` o `.h` y realiza las siguientes acciones sobre ellos:
  - Elimina los comentarios.
  - Interpreta las directivas de preprocesamiento. En el caso de `#include <iostream>`, sustituye dicha linea por todo el archivo `iostream.h`.
- **Compilador:** Toma el código fuente preprocesado y realiza las siguientes acciones:
  - Traduce el código fuente a código objeto.
    - No se resuelven las posibles referencias externas al archivo (variables o funciones declaradas en otros archivos).
  - Almacena el resultado en un archivo `codigo.o`, llamado **módulo objeto**.
- **Enlazador:** Resuelve las referencias externas y genera un archivo ejecutable.
  - Las referencias pueden estar en otros archivos `.o` o en bibliotecas.
  - Para poder crear un archivo ejecutable, debe existir una función `main()`.

Trabajemos con un archivo sencillo escrito en C++:

###### main.cpp
```c++
#include <iostream>
#include "functions.h"
using namespace std;

int main() {
   print_hello();

   cout << "El factorial de 5 es " << factorial(5) << "\n";

   return 0;
}
```

A partir de él podemos generar su módulo objeto mediante la opción `-c` de la orden `g++`. El módulo objeto creado tiene el siguiente contenido (traducido de hexadecimal a texto):

###### main.o
```
.ELF..............>.................................@.....@.....UH..SH..
......H.5....H.=.........H..............H.......H.5....H............H...
[].UH..H....}..u..}..u2.}.....u)H.=.........H......H.5....H......H......
....UH.................]..El factorial de 5 es ..................GCC: (G
NU) 8.2.1 20181127...............zR..x.......... ...........S....A....C.
.E...I..........@.......I....A....C...D..... ...`............A....C..P..
........................................................................
........................................................................
................................%.......................4.......S.......
I.......d...............................................................
........................................................s...............
S.......x...............................................................
........................................................................
........................................................................
........6........................main.cpp._ZStL19piecewise_construct._ZS
tL8__ioinit._Z41__static_initialization_and_destruction_0ii._GLOBAL__sub
_I_main._GLOBAL_OFFSET_TABLE_._Z11print_hellov._ZSt4cout._ZStlsISt11char
_traitsIcEERSt13basic_ostreamIcT_ES5_PKc._Z9factoriali._ZNSolsEi._ZNSt8i
os_base4InitC1Ev.__dso_handle._ZNSt8ios_base4InitD1Ev.__cxa_atexit......
........................................................................
........................*.......................4.......................
;.......................C.......................s.......................
x.......................................................................
........*...............................................................
........................D...............S.......d.......................
..symtab..strtab..shstrtab..rela.text..data..bss..rodata..rela.init_arra
y..comment..note.GNU-stack..rela.eh_frame...............................
........................................ .......................@.......
........................................@...............X.......P.......
........................&...............................................
................,.......................................................
........1...............................................................
>...............................................................9.......
@.......................................................J.......0.......
................................................S.......................
3.......................................h.......................8.......
................................c.......@.......................H.......
........................................................X...............
................................................C.......................
........................................r...............................
```

Aunque se puedan apreciar en él algunas trazas de código fuente y fragmentos de texto comprensible, está claro que este texto no está hecho para que lo lea un humano (al menos del siglo XXI).

Vamos ahora a compilarlo normalmente:

```
$ g++ main.cpp 
  /usr/bin/ld: /tmp/ccBwVxKe.o: in function `main':
  main.cpp:(.text+0xa): undefined reference to `print_hello()'
  /usr/bin/ld: main.cpp:(.text+0x2a): undefined reference to `factorial(int)'
  collect2: error: ld returned 1 exit status
```

Nos ha dado un error (`ld returned 1 exit status` es lo mismo que `$?=1`) porque `ld`, que es el enlazador de GNU, no ha podido resolver las referencias externas (`print_hello()` y `factorial()`). Por ello, no se ha generado el archivo `main.o`.

Para solucionar esto, vamos a cargar los archivos correspondientes a dichas referencias:

###### hello.cpp
```c++
#include <iostream>
#include "functions.h"
using namespace std;

void print_hello(){
   cout << "Hello World!";
}
```

###### factorial.cpp
```c++
#include "functions.h"

int factorial(int n){
   
   if(n!=1){
      return(n * factorial(n-1));
   }
   else return 1;
}
```

###### functions.h
```h
void print_hello();
int factorial(int n);
```

Ahora generamos sus módulos objeto correspondientes:

    $ g++ -c factorial.cpp 
    $ g++ -c hello.cpp 
    $ g++ -c main.cpp

Y ahora compilamos todos los módulos objeto en un sólo archivo ejecutable:

    $ g++ main.o factorial.o hello.o

El nombre que g++ le da por defecto a dicho archivo es `a.out`, pero podemos especificarlo como argumento de la opción `-o`.

    $ g++ -o programa1 main.o factorial.o hello.o

Esta es la forma completa y compleja de compilar un programa. En la práctica, podemos resumir todo el proceso anterior en una sola orden:

    $ g++ -o programa1 main.cpp factorial.cpp hello.cpp

## 2 - INTRODUCCIÓN A LAS BIBLIOTECAS

Una biblioteca es una colección de módulos objeto que contienen constantes, variables y funciones que pueden ser utilizadas por otros módulos. Para crear una biblioteca usamos la orden `ar`.

Por ejemplo, consideremos los siguientes programas:

###### sin.cpp
```c++
#include "mates.h"

using namespace std;

void print_sin(float valor){
   float local;

   local=valor*M_PI/180.0;
   cout << endl;
   cout << "sin\(" << valor << ")=" << sin(local);
   cout << endl;
}
```

###### cos.cpp
```c++
#include "mates.h"

using namespace std;

void print_cos(float valor){
   float local;

   local=valor*M_PI/180.0;
   cout << endl;
   cout << "cos\(" << valor << ")=" << cos(local);
   cout << endl;
}
```

###### tan.cpp
```c++
#include "mates.h"

using namespace std;

void print_tan(float valor){
   float local;

   local=valor*M_PI/180.0;
   cout << endl;
   cout << "tan\(" << valor << ")=" << tan(local);
   cout << endl;
}
```

###### mates.h
```h
#include <iostream>
#include <cmath>

#ifndef M_PI
#define M_PI 3.14159265358979323846
#endif

void print_sin(float valor);
void print_cos(float valor);
void print_tan(float valor);
```

Aunque sea un poco tonto porque `mates.h` ya incluye `cmath`, vamos a crear nuestra propia biblioteca a partir de estos archivos:

```
$ g++ -c sin.cpp
$ g++ -c cos.cpp
$ g++ -c tan.cpp
$ ar -rvs libtrig.a sin.o cos.o tan.o
```

Nuestra biblioteca `libtrig.a` tiene una estructura similar al módulo objeto mostrado anteriormente, por lo que no la vamos a enseñar. Vamos a compilar ahora un programa a partir de los módulos objetos de `hello.cpp` y `factorial.cpp`, que ya se definieron anteriormente, y `main2.cpp`, cuya estructura es la siguiente:

###### main2.cpp
```c++
#include <iostream>
#include "functions.h"
#include "mates.h"

using namespace std;

int main(){
    print_hello();
    cout << endl;
    cout << "The factorial of 5 is " << factorial(5) << endl;
    print_sin(90.0);
    print_cos(90.0);
    print_tan(90.0);

    return 0;
}
```

```
$ g++ -c main.cpp
$ g++ -o programa2 main2.o factorial.o hello.o
  main2.o: In function 'main':
  main2.cpp:(.text+0x58): undefined reference to 'print_sin(float)'
  main2.cpp:(.text+0x65): undefined reference to 'print_cos(float)'
  main2.cpp:(.text+0x72): undefined reference to 'print_tan(float)'
  collect2: ld returned 1 exit status
```

Debido a la estructura de trabajo del compilador, `main.o` no ha tenido problemas para crearse porque el módulo objeto no necesita de bibliotecas para funcionar. Sin embargo, a la hora de crear una correspondencia entre los módulos objeto con las bibliotecas correspondientes, el enlazador las ha encontrado y es aquí donde se produce el error. La forma correcta de compilar un programa incluyendo bibilotecas es la siguiente:

```
g++ -L./ -o programa2 main2.o factorial.o hello.o -ltrig
```

Al trabajar con varios módulos se van generando **dependencias** entre ellos. A la hora de compilar de nuevo los programas para realizar actualizaciones es muy importante tener en cuenta los cambios que se han hecho a todos los archivos con los que se trabaja para que no existan errores. Por ejemplo, un cambio en una biblioteca puede requerir de cambios en las sentencias del programa principal.

## 3 - USO DE ARCHIVOS DE TIPO MAKEFILE

Para gestionar las dependencias de forma más automática y cómoda, podemos usar la utilidad `make`, que comprueba qué archivos se han modificado desde la ultima vez que se utilizó para crear el archivo ejecutable y hace cambios sólo en los sectores en los que sea necesario. Esto se hace mediante un documento de texto de tipo `makefile` que especifica las dependencias del programa.

### 3.1 - ESTRUCTURA DE UN ARCHIVO MAKEFILE

Los archivos `makefile` contienen reglas que indican el programa que se quiere compilar (el **objetivo**), las **dependencias** de dicho objetivo (que pueden estar vacías) y las órdenes que seguir para compilarlo. Siguen la siguiente estructura:

```makefile
objetivo1: dependencias
	orden1
	orden2
...
	ordenn
objetivo2: dependencias
...
objetivon: dependencias
...
```

Es importante dejar claro que antes de las órdenes hay que introducir un carácter de tabulación, no vale escribir espacios como en otros archivos.

Vamos a construir el `makefile` para `programa1`:

###### makefileA
```makefile
programa1:
	g++ –o programa1 main.cpp hello.cpp factorial.cpp
```

```
$ make -f makefileA
  make: 'programa1' is up to date.
```

Como no hemos hecho ninguna modificación a `programa1`, simplemente nos avisa de que su versión es la más actualizada. Si cambiamos la línea `cout << "Hello World!!";` a `cout << "Dear World, Hello!";` nos da el mismo mensaje, pero porque él mismo ha actualizado el programa.

### 3.2 - USO DE VARIABLES EN ARCHIVOS MAKEFILE

`make` nos permite definir variables de la misma forma que en un script de bash. Tomemos como ejemplo el siguiente archivo:

###### makefileg
```makefile
# Variable que indica el compilador que se va a utilizar
CC=g++

# Variable que indica el directorio en donde se encuentran los archivos de cabecera
INCLUDE_DIR= ./includes

# Variable que indica el directorio en donde se encuentran las bibliotecas
LIB_DIR= ./

programa2: main2.o factorial.o hello.o libmates.a
   $(CC) -L$(LIB_DIR) -o programa2 main2.o factorial.o hello.o -lmates

main2.o: main2.cpp
   $(CC) -I$(INCLUDE_DIR) -c main2.cpp

factorial.o: factorial.cpp
   $(CC) -I$(INCLUDE_DIR) -c factorial.cpp

hello.o: hello.cpp
   $(CC) -I$(INCLUDE_DIR) -c hello.cpp

libmates.a: sin.o cos.o tan.o
   ar -rvs libmates.a sin.o cos.o tan.o

sin.o: sin.cpp
   $(CC) -I$(INCLUDE_DIR) -c sin.cpp

cos.o: cos.cpp
   $(CC) -I$(INCLUDE_DIR) -c cos.cpp

tan.o: tan.cpp
   $(CC) -I$(INCLUDE_DIR) -c tan.cpp
```

En él hemos declarado las variables `CC`, `INCLUDE_LIB` y `LIB_DIR`, a las que podemos llamar con la sintaxis `$()` habitual.

`make` también tiene una serie de variables especiales que actúan cada vez que se procesa una regla. Aquí veremos cuatro de ellas:

| VARIABLE | DESCRIPCIÓN |
| :---: | --- |
| `$@` | Representa el nombre del objetivo de la regla en la que nos encontramos |
| `$<` | Representa la primera dependencia de la regla en la que nos encontramos |
| `$?` | Representa las dependencias de la regla en la que nos encontramos que hayan sido actualizadas dentro del objetivo de la regla y separadas por espacios en blanco |
| `$^` | Representa todas las dependencias separadas por espacios en blanco |

Para mostrar el uso de estas variables vamos a suponer un archivo `makefile` que contenga las siguientes sentencias:

```makefile
CC = g++
CPPFLAGS = -Wall
SRCS = main.cpp factorial.cpp hello.cpp
OBJS = main.o factorial.o hello.o
HDRS = functions.h
```

La opción `-Wall` de la variable `CPPFLAGS` sirve para mostrar todas las advertencias (***W**arning all*) que pudieran aparecer durante la compilación.

#### 3.2.1 - Uso de `$@`

#### 3.2.2 - Uso de `$<`

#### 3.2.3 - Uso de `$?`

#### 3.2.4 - Uso de `$^`

#### 3.2.5 - Uso de otras opciones y variables

## 4 - EJERCICIOS

### EJERCICIO 1

Pruebe a comentar en el archivo fuente `main.cpp` la directiva de procesamiento `#include "functions.h"`. La línea quedaría así:

    //#include "functions.h"

Pruebe a generar ahora el módulo objeto de dicho archivo. ¿Qué ha ocurrido?

### EJERCICIO 2

Explique por qué fallan las siguientes órdenes y los tipos de errores que se encuentran en ellas:

```
$ mkdir includes
$ mv *.h includes
$ rm *.o programa2
$ g++ -L./ -o programa2 main2.cpp factorial.cpp hello.cpp -lmates
  main2.cpp:2:23: error: functions.h: No such file or directory
  main2.cpp:3:19: error: mates.h: No such file or directory
  main2.cpp: In function ‘int main()’:
  main2.cpp:8: error: ‘print_hello’ was not declared in this scope
  main2.cpp:10: error: ‘factorial’ was not declared in this scope
  main2.cpp:11: error: ‘print_sin’ was not declared in this scope
  main2.cpp:12: error: ‘print_cos’ was not declared in this scope
  main2.cpp:13: error: ‘print_tan’ was not declared in this scope
  factorial.cpp:1:23: error: functions.h: No such file or directory
  hello.cpp:2:23: error: functions.h: No such file or directory
```

### EJERCICIO 3

Copie el contenido del siguiente  makefile a un archivo llamado `makefileE` ubicado en el mismo directorio en el que están los archivos de código fuente `.cpp`. Pruebe a modificar distintos archivos `.cpp` (puede hacerlo usando la orden `touch` sobre uno o varios de esos archivos) y compruebe la secuencia de instrucciones que se muestra en el terminal al ejecutarse la orden `make`. ¿Se genera siempre la misma secuencia de órdenes cuando los archivos han sido modificados que cuando no? ¿A qué cree puede deberse tal comportamiento?

```makefile
programa2: main2.o factorial.o hello.o libmates.a
	g++ -L./ -o programa2 main2.o factorial.o hello.o -lmates
main2.o: main2.cpp
	g++ -I./includes -c main2.cpp
factorial.o: factorial.cpp
	g++ -I./includes -c factorial.cpp
hello.o: hello.cpp
	g++ -I./includes -c hello.cpp
libmates.a: sin.o cos.o tan.o
	ar -rvs libmates.a sin.o cos.o tan.o
sin.o: sin.cpp
	g++ -I./includes -c sin.cpp
cos.o: cos.cpp
	g++ -I./includes -c cos.cpp
tan.o: tan.cpp
	g++ -I./includes -c tan.cpp
```

### EJERCICIO 4

Obtener un nuevo `makefileF` a partir del makefile del ejercicio anterior que incluya además las dependencias sobre los archivos de cabecera. Pruebe a modificar cualquier archivo de cabecera (usando la orden `touch`) y compruebe la secuencia de instrucciones que se muestra en el terminal al ejecutarse la orden `make`.

### EJERCICIO 5

Usando como base el archivo `makefileG`, sustituya la línea de orden de la regla cuyo objetivo es `programa2` por otra en la que se use alguna de las variables especiales y cuya ejecución sea equivalente.

### EJERCICIO 6

Utilizando como base el archivo `makefileG` y los archivos fuente asociados, realice los cambios que considere oportunos para que, en la construcción de la biblioteca estática `libmates.a`, este archivo pase a estar en un subdirectorio denominado `libs` y se pueda enlazar correctamente con el resto de archivos objeto.

### EJERCICIO 7

Busque la variable predefinida de `make` que almacena la utilidad del sistema que permite construir bibliotecas. Recuerde que la orden para construir una biblioteca estática a partir de una serie de archivos objeto es `ar` (puede usar la orden `grep` para filtrar el contenido; no vaya a leer línea a línea toda la salida). Usando el archivo `makefileG`, sustituya la orden ar por su variable correspondiente.

### EJERCICIO 8

Dado el siguiente archivo makefile, explique las dependencias que existen y para qué sirve cada una de las líneas del mismo. Enumere las órdenes que se van a ejecutar a consecuencia de invocar la utilidad make sobre este archivo.

```makefile

CC=g++
CPPFLAGS=-Wall –I./includes
SOURCE_MODULES=main.cpp factorial.cpp hello.cpp
OBJECT_MODULES=$(SOURCE_MODULES:.cpp=.o)
EXECUTABLE=programa1

all: $(OBJECT_MODULES) $(EXECUTABLE)

$(EXECUTABLE): $(OBJECT_MODULES)
	$(CC) $^ -o $@

# Regla para obtener los archivos objeto .o que dependerán de los archivos .cpp
# Aquí, $< y $@ tomarán valores respectivamente main.cpp y main.o y así sucesivamente
.o: .cpp
	$(CC) $(CPPFLAGS) $< -o $@
```

### EJERCICIO 9

Con la siguiente especificación de módulos escriba un archivo denominado `makefilePolaca` que automatice el proceso de compilación del programa final de acuerdo a la siguiente descripción:
- Compilador: gcc o g++.
- Archivos cabecera: `calc.h` (ubicado en un subdirectorio denominado cabeceras).
- Archivos fuente: `main.c`, `stack.c`, `getop.c` y `getch.c`.
- Nombre del programa ejecutable: `calculadoraPolaca`

Además, debe incluir una regla denominada `borrar`, sin dependencias, cuya funcionalidad sea la de eliminar los archivos objeto y el programa ejecutable.

## 5 - SOLUCIONES

### EJERCICIO 1

Ejecutamos la compilación con dicha línea comentada:

```console
$ g++ -c main.cpp
  main.cpp: In function ‘int main()’:
  main.cpp:7:17: error: ‘print_hello’ was not declared in this scope
  print_hello();
              ^
  main.cpp:9:52: error: ‘factorial’ was not declared in this scope
  cout << "The factorial of 7 is " << factorial(7) << endl;
```

Esto ocurre porque, al comentar la línea, no se incluyen las bibliotecas pertinentes en el código fuente.

### EJERCICIO 2

Falla porque hemos movido las bibliotecas a la carpeta `./includes`, por lo que la orden de compilación correcta es la siguiente:

```console
$ g++ -L./ -o programa2 main2.o factorial.o hello.o -lmates
```

### EJERCICIO 3

Usamos `make` por primera vez:

```console
$ make -f makefileE
  g++ -I./includes -c main2.cpp
  g++ -I./includes -c factorial.cpp
  g++ -I./includes -c hello.cpp
  g++ -I./includes -c sin.cpp
  g++ -I./includes -c cos.cpp
  g++ -I./includes -c tan.cpp
  ar -rvs libmates.a sin.o cos.o tan.o
  r - sin.o
  r - cos.o
  r - tan.o
  g++ -L./ -o programa2 main2.o factorial.o hello.o libmates.a
```

Si hacemos `touch` en algunos algunas de las dependencias, `make` detecta que son más recientes y las actualiza.

```console
$ touch main2.cpp 
$ make -f makefileE
  g++ -I./includes -c main2.cpp
  g++ -L./ -o programa2 main2.o factorial.o hello.o libmates.a
$ touch hello.cpp sin.cpp
$ make -f makefileE
  g++ -I./includes -c hello.cpp
  g++ -I./includes -c sin.cpp
  ar -rvs libmates.a sin.o cos.o tan.o
  r - sin.o
  r - cos.o
  r - tan.o
  g++ -L./ -o programa2 main2.o factorial.o hello.o libmates.a
```

### EJERCICIO 4

###### makefileF
```makefile
LIB_DIR=./

programa2: main2.o factorial.o hello.o libmates.a
	g++ -L./ -o programa2 main.o factorial.o hello.o libmates.a

main2.o: main2.cpp ./includes/functions.h ./includes/mates.h 
	g++ -I./includes -c main2.cpp

factorial.o: factorial.cpp ./includes/functions.h
	g++ -I./includes -c factorial.cpp

hello.o: hello.cpp ./includes/functions.h
	g++ -I./includes -c hello.cpp

libmates.a: sin.o cos.o tan.o
	ar -rvs libmates.a sin.o cos.o tan.o

sin.o: sin.cpp ./includes/mates.h
	g++ -I./includes -c sin.cpp

cos.o: cos.cpp ./includes/mates.h
	g++ -I./includes -c cos.cpp

tan.o: tan.cpp ./includes/mates.h
	g++ -I./includes -c tan.cpp

cleanAll: cleanObj cleanLib
	rm programa2

cleanObj:
	rm main2.o factorial.o hello.o

cleanLib:
	rm sin.o cos.o tan.o libmates.a
```

Asumimos, por tanto, que las bibliotecas se encuentran en `./includes` como en los ejercicios anteriores. Modificamos ahora la cabecera con `touch` y sucede lo mismo que en el ejercicio anterior:

```console
$ touch includes/mates.h 
$ make -f makefileF
  g++ -I./includes -c main2.cpp
  g++ -I./includes -c sin.cpp
  g++ -I./includes -c cos.cpp
  g++ -I./includes -c tan.cpp
  ar -rvs libmates.a sin.o cos.o tan.o
  r - sin.o
  r - cos.o
  r - tan.o
  g++ -L./ -o programa2 main2.o factorial.o hello.o libmates.a
```

### EJERCICIO 5

###### makefileG
```makefile
CC=g++
INCLUDE_DIR= ./includes
LIB_DIR= ./

#Variable que indica las opciones que se le va a pasar al compilador
CPPFLAGS= -Wall

programa2: main2.o factorial.o hello.o libmates.a
	$(CC) -L$(LIB_DIR) -o $@ $^

main2.o: main2.cpp
	$(CC) -I$(INCLUDE_DIR) -c $<

factorial.o: factorial.cpp
	$(CC) -I$(INCLUDE_DIR) -c $<

hello.o: hello.cpp
	$(CC) -I$(INCLUDE_DIR) -c $<

libmates.a: sin.o cos.o tan.o
	ar -rvs libmates.a $^

sin.o: sin.cpp
	$(CC) -I$(INCLUDE_DIR) -c $<

cos.o: cos.cpp
	$(CC) -I$(INCLUDE_DIR) -c $<

tan.o: tan.cpp
	$(CC) -I$(INCLUDE_DIR) -c $<

clean:
	rm *.o programa2
```

### EJERCICIO 6

###### MakefileG
```makefile
CC=g++
INCLUDE_DIR= ./includes
LIB_DIR= ./libs
CPPFLAGS= -Wall

programa2: main2.o factorial.o hello.o $(LIB_DIR)/libmates.a
	$(CC) -L$(LIB_DIR) -o $@ $^

main2.o: main2.cpp
	$(CC) -I$(INCLUDE_DIR) -c $<

factorial.o: factorial.cpp
	$(CC) -I$(INCLUDE_DIR) -c $<

hello.o: hello.cpp
	$(CC) -I$(INCLUDE_DIR) -c $<
	$(LIB_DIR)/libmates.a: sin.o cos.o tan.o
	ar -rvs $@ $^

sin.o: sin.cpp
	$(CC) -I$(INCLUDE_DIR) -c $<

cos.o: cos.cpp
	$(CC) -I$(INCLUDE_DIR) -c $<

tan.o: tan.cpp
	$(CC) -I$(INCLUDE_DIR) -c $<

clean:
	rm *.o programa2
```

### EJERCICIO 7

###### makefileG
```makefile
CC=g++
INCLUDE_DIR= ./includes
LIB_DIR= ./
programa2: main2.o factorial.o hello.o libmates.a
$(CC) -L$(LIB_DIR) -o programa2 main2.o factorial.o hello.o -lmates
main2.o: main2.cpp
      $(CC) -I$(INCLUDE_DIR) -c main2.cpp
factorial.o: factorial.cpp
$(CC) -I$(INCLUDE_DIR) -c factorial.cpp
hello.o: hello.cpp
      $(CC) -I$(INCLUDE_DIR) -c hello.cpp
libmates.a: sin.o cos.o tan.o
      $(AR) -rvs libmates.a sin.o cos.o tan.o
sin.o: sin.cpp
      $(CC) -I$(INCLUDE_DIR) -c sin.cpp
cos.o: cos.cpp
      $(CC) -I$(INCLUDE_DIR) -c cos.cpp
tan.o: tan.cpp
      $(CC) -I$(INCLUDE_DIR) -c tan.cpp
```

### EJERCICIO 8

```makefile
CC=g++
CPPFLAGS=-Wall –I./includes
#SOURCEMODUELS nos indica todos los archivos cpp
SOURCE_MODULES=main.cpp factorial.cpp hello.cpp
#OBJECTMODULES nos indica los archivos .o
OBJECT_MODULES=$(SOURCE_MODULES:.cpp=.o)
#EXECUTABLE nos indica el nombre del archivo ejecutable que se generara
EXECUTABLE=programa1
#all 
all: $(OBJECT_MODULES) $(EXECUTABLE)
$(EXECUTABLE): $(OBJECT_MODULES)
       $(CC) $^ -o $@
# Regla para obtener los archivos objeto .o que dependerán de los archivos .cpp
# Aquí, $< y $@ tomarán valores respectivamente main.cpp y main.o y así sucesivamente
.o: .cpp
       $(CC) $(CPPFLAGS) $< -o $@
```