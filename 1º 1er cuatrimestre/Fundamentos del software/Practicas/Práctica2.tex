\chapter{PERMISOS Y REDIRECCIONES}

\section{Modificación de los permisos de acceso}

La orden \code{chmod} permite cambiar los permisos de aquellos ficheros de los que se es administrador.
Existen dos formas de declarar dicha modificación de ficheros:

\subsubsection{Representación absoluta u octal}

Cada permiso \code{rwx} se representa como un bit, de forma que \code{---} vale \code{0}, \code{rwx} vale \code{7} y, por ejemplo, \code{r-x} vale \code{5}.
Los tres grupos de bits se unen para formar el conjunto total de permisos:

\begin{lstlisting}[language=sh]
chmod 664 fichero
# -rw-rw-r-- 1 groctel groctel 0 oct 15 11:08 fichero
chmod 755 fichero
# -rwxr-xr-x 1 groctel groctel 0 oct 15 11:08 fichero
\end{lstlisting}

\subsection{Representación simbólica o textual}

Asigna a cada grupo de usuarios una modificación de los permisos sin afectar a los no determinados.
Existen cuatro grupos de usuarios que pueden recibir modificaciones:

\begin{itemize}
	\item\code{u}\textbf{:} Usuario creador del fichero.
	\item\code{g}\textbf{:} Grupo de usuarios al que pertenece el creador.
	\item\code{o}\textbf{:} Otros usuarios.
	\item\code{a}\textbf{:} Todos los usuarios.
\end{itemize}

Para declarar el tipo de cambio sobre los permisos se utiliza uno de dos modificadores:

\begin{itemize}
	\item\code{+}\textbf{:} Añade los permisos siguientes.
	\item\code{-}\textbf{:} Elimina los permisos siguientes.
\end{itemize}

Para declarar el permiso a modificar utilizamos las letras \code{rwx} vistas anteriormente.

\begin{lstlisting}[language=sh]
chmod u+x     # Dar permiso de ejecución al usuario
chmod go+r-w  # Dar permisos de lectura y quitar de escritura al grupo y a otros
chmod a+r,o-w # Dark permisos de lectura a todos y quitar permisos de escritura a otros
\end{lstlisting}

\section{Metacaracteres de redirección}

A la hora de establecer flujos de entrada y salida, Linux utiliza por defecto los flujos estándar:

\begin{itemize}
	\item\code{stdin}\textbf{:} Entrada estándar. Se lee desde la terminal.
	\item\code{stdout}\textbf{:} Salida estándar. Se imprime texto en la terminal.
	\item\code{stderr}\textbf{:} Salida de error estándar. Se imprime el texto de error en la terminal.
\end{itemize}

Los metacaracteres de redirección nos permiten determinar entradas y salidas alternativas.

\begin{itemize}
	\item\code{< fichero}\textbf{:} Redirige la entrada al contenido del fichero \code{fichero}.
	\item\code{> fichero}\textbf{:} Redirige la salida al contenido del fichero \code{fichero}. El contenido de \code{fichero} se sobreescribe.
	\item\code{>> fichero}\textbf{:} Redirige la salida al contenido del fichero \code{fichero} añadiendo el contenido de la salida sin sobreescribir.
	\item\code{2> fichero}\textbf{:} Redirige la salida de error al contenido del fichero \code{fichero}. El contenido de \code{fichero} se sobreescribe.
	\item\code{2>> fichero}\textbf{:} Redirige la salida de error al contenido del fichero \code{fichero} añadiendo el contenido de la salida sin sobreescribir.
	\item\code{&> fichero}\textbf{:} Redirige la salida y la salida de error al contenido del fichero \code{fichero}. El contenido de \code{fichero} se sobreescribe.
	\item\code{&>> fichero}\textbf{:} Redirige la salida y la salida de error al contenido del fichero \code{fichero} añadiendo el contenido de la salida sin sobreescribir.
\end{itemize}

\begin{lstlisting}[language=sh]

\end{lstlisting}

\subsection{Cauces}
	\item\code{[command1] | [command2]}\textbf{:} Utiliza la salida de la orden [command1] como entrada de la orden [command2] (creación de cauces)
	\item\code{[command1] |& [command2]}\textbf{:} Utiliza la salida junto con los mensajes de error de la orden [command1] como entrada de la orden [command2] (creación de cauces)

\section{METACARACTERES SINTÁCTICOS}

Los metacaracteres sintácticos nos permiten combinar varias órdenes en una sola. Existen cuatro metacaracteres sintácticos:
- \code{;}\textbf{:} Separa órdenes que se ejecutan secuencialmente
- \code{&&}\textbf{:} Separa órdenes ejecutando la segunda sólo si la primera tiene éxito
- \code{||}\textbf{:} Separa órdenes ejecutando la segunda sólo si la primera falla
- \code{()}\textbf{:} Trata las órdenes entre paréntesis como una sola

```
$ echo Hola ; cat Saludo
  Hola
  Comostamo
$ cat saludo && echo Saludo ejecutado correctamente
  Comostamo
  Saludo ejecutado correctamente
$ cat adios || echo No hay despedida
  cat: adios: No such file or directory
  No hay despedida
$ (cat adios || echo No hay despedida) && echo Toca hacerla
  cat: adios: No such file or directory
  No hay despedida
  Toca hacerla
```

\section{EJERCICIOS}

\subsection{EJERCICIO 1}

Se debe utilizar solamente una vez la orden \code{chmod} en cada apartado. Los cambios se harán en un fichero concreto del directorio de trabajo (salvo que se indique otra cosa). Cambiaremos uno o varios permisos en cada apartado (independientemente de que el fichero ya tenga o no dichos permisos) y comprobaremos que funciona correctamente:
- Dar permiso de ejecución al “resto de usuarios”.
- Dar permiso de escritura y ejecución al “grupo”.
- Quitar el permiso de lectura al “grupo” y al “resto de usuarios”.
- Dar permiso de ejecución al “propietario” y permiso de escritura el “resto de usuarios”.
- Dar permiso de ejecución al “grupo” de todos los ficheros cuyo nombre comience con la letra \code{e}. Nota: Si no hay más de dos ficheros que cumplan esa condición, se deberán crear ficheros que empiecen con \code{e} y/o modificar el nombre de ficheros ya existentes para que cumplan esa condición.

\subsection{EJERCICIO 2}

Utilizando solamente las órdenes de la práctica anterior y los metacaracteres de redirección de salida:
- Cree un fichero llamado \code{ej31}, que contendrá el nombre de los ficheros del directorio padre del directorio de trabajo.
- Cree un fichero llamado \code{ej32}, que contendrá las dos últimas líneas del fichero creado en el ejercicio anterior.
- Añada al final del fichero \code{ej32}, el contenido del fichero \code{ej31}.

\subsection{EJERCICIO 3}

Utilizando el metacarácter de creación de cauces y sin utilizar la orden \code{cd}:
- Muestre por pantalla el listado (en formato largo) de los últimos 6 ficheros del directorio \code{/etc}.
- La orden \code{wc}muestra por pantalla el número de líneas, palabras y bytes de un fichero (consulta la orden man para conocer más sobre ella). Utilizando dicha orden, muestre por pantalla el número de caracteres (sólo ese número) de los ficheros del directorio de trabajo que comiencen por los caracteres \code{e} o \code{f}.

\subsection{EJERCICIO 4}

Resuelva cada uno de los siguientes apartados:
- Cree un fichero llamado \code{ejercicio1}, que contenga las 17 últimas líneas del texto que proporciona la orden man para la orden chmod (se debe hacer en una única línea de órdenes y sin utilizar el metacarácter \code{;} ).
- Al final del fichero \code{ejercicio1}, añada la ruta completa del directorio de trabajo actual.
- Usando la combinación de órdenes mediante paréntesis, cree un fichero llamado \code{ejercicio3} que contendrá el listado de usuarios conectados al sistema (orden \code{who}) y la lista de ficheros del directorio actual.
- Añada, al final del fichero \code{ejercicio3}, el número de líneas, palabras y caracteres del fichero \code{ejercicio1}. Asegúrese de que, por ejemplo, si no existiera \code{ejercicio1}, los mensajes de error también se añadieran al final de \code{ejercicio3}.
- Con una sola orden \code{chmod}, cambie los permisos de los ficheros \code{ejercicio1} y \code{ejercicio3}, de forma que se quite el permiso de lectura al “grupo” y se dé permiso de ejecución a las tres categorías de usuarios.

\section{SOLUCIONES}

\subsection{EJERCICIO 1}

```
$ chmod o+x practFS.ext
$ ls -l practFS.ext
  -rw-r--r-x 1 groctel groctel 0 oct  6 21:23 practFS.ext
$ chmod g+rx practFS.ext
$ ls -l practFS.ext
  -rw-r-xr-x 1 groctel groctel 0 oct  6 21:23 practFS.ext
$ chmod go-r practFS.ext
$ ls -l practFS.ext
  -rw---x--x 1 groctel groctel 0 oct  6 21:23 practFS.ext
$ chmod u+x,o+w practFS.ext
$ ls -l practFS.ext
  -rwx--x-wx 1 groctel groctel 0 oct  6 21:23 practFS.ext
$ chmod g+x e*
$ ls -l e*
  -rw-r-xr-- 1 groctel groctel 0 oct  6 21:26 ejer3arch.txt
  -rw-r-xr-- 1 groctel groctel 0 oct  6 21:26 ejer3filetags.txt
```

Todas las órdenes se han realizado en el directorio \code{ejercicio1/Ejer3} de la práctica anterior. Los cambios de permisos podrían haberse declarado en formato octal, pero debido a que sólo se deben cambiar unos permisos específicos es más sencillo hacerlo de forma textual.

\subsection{EJERCICIO 2}

```
$ ls .. > Ej31
$ cat Ej31
  Ejer1
  Ejer2
  Ejer3
$ tail -2 Ejer31 > Ej32
$ cat Ej32
  Ejer2
  Ejer3
$ cat Ej31 >> Ej32
$ cat Ej32
  Ejer2
  Ejer3
  Ejer1
  Ejer2
  Ejer3
```

\subsection{EJERCICIO 3}

```
$ ls -l /etc | head -6
  total 1196
  drwxr-xr-x  3 root root     4096 jul 25 05:08 acpi
  -rw-r--r--  1 root root     3028 jul 25 05:04 adduser.conf
  drwxr-xr-x  2 root root    12288 sep 23 09:58 alternatives
  -rw-r--r--  1 root root      401 may 29  2017 anacrontab
  -rw-r--r--  1 root root      433 oct  2  2017 apg.conf
$ ls e* f* | wc -m
  45
```

\subsection{EJERCICIO 4}

```
$ (man chmod | tail -17) > ejercicio1
$ cat ejercicio1
  El tocho de 17 líneas que empieza por “GNU coreutils online help:”
$ pwd >> ejercicio1
$ cat ejercicio1
  El mismo tocho de ahora 18 líneas que acaba en /home/groctel
$ (who ; ls) > ejercicio3
$ cat ejercicio3
  groctel  :0           2018-10-08 15:47 (:0)
  robet201.me
$ wc ejercicio1 >> ejercicio3
$ cat ejercicio3
  groctel  :0           2018-10-08 15:47 (:0)
  robet201.me
  18  73 738 ejercicio1
$ chmod g+r,a+x ejercicio1 ejercicio3
$ ls -l
  total 20
  drwxr-xr-x 2 groctel groctel 4096 oct  6 21:30 Ejer1
  drwxr-xr-x 3 groctel groctel 4096 oct  6 21:29 Ejer2
  drwxr-xr-x 2 groctel groctel 4096 oct  8 16:50 Ejer3
  -rwxr-xr-x 1 groctel groctel  738 oct  8 16:53 ejercicio1
  -rwxr-xr-x 1 groctel groctel  169 oct  8 16:54 ejercicio3
```
