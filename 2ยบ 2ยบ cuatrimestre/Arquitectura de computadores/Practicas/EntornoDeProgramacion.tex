\chapter{Entorno de programación: \textit{atcgrid} y gestor de colas}

\section{Clúster de prácticas (\textit{atcgrid})}

El clúster de prácticas (a partir de ahora \textit{atcgrid}), es un computador formado por tres nodos de cómputo (servidores \textit{rack} o de armario) \href{https://www.supermicro.com/products/system/1U/6016/SYS-6016T-T.cfm}{SuperMicro SuperServer 6016T-T}.
Los estudiantes no tienen acceso directo a estos servidores.
Para trabajar con ellos, lo hacen a partir de un nodo \textit{front-end} que envía los procesos pertinentes a los nodos de cómputo.

Cada uno de los nodos tiene una placa base \href{www.supermicro.com/products/motherboard/QPI/5500/X8DTL-i.cfm}{SuperMicro X8DTL-i} de doble zócalo (\textit{dual socket}) para un \textit{chipset} \href{https://www.intel.com/content/www/us/en/design/products-and-solutions/processors-and-chipsets/tylersburg/technical-library.html}{Intel 5520}.
Cuenta con $18$ espacios de memoria DDR3 800/1066 divididos en $9$ para cada uno de los procesadores que componen el \textit{chipset} y cada uno con espacio para $4GB$, permitiendo un total de $72GB$ de memoria RAM\@.

Los procesadores instalados en la placa son \href{https://ark.intel.com/content/www/us/en/ark/products/48768/intel-xeon-processor-e5645-12m-cache-2-40-ghz-5-86-gt-s-intel-qpi.html}{Intel Xeon E5645}.
Estos procesadores cuentan con 6 núcleos y 12 hebras con una frecuencia de $2.40GHz$ por núcleo; una caché L3 compartida por todos los núcleros de 12MB y un bus de $5.86GT/s$ (Giga Transferencias por segundo).

El acceso al \textit{atcgrid} se hace mediante una conexión segura al directorio \texttt{\$HOME} que se ofrece a cada estudiante de la asignatura.
Esta conexión puede hacerse con \texttt{ssh} para interactuar con el sistema y con \texttt{sftp} para transferir ficheros entre la máquina local y el servidor.

\subsection{\texttt{ssh}}

La conexión por \texttt{ssh} se hace usando la orden homónima e introduciendo como argumento el \texttt{username} propio:

\begin{lstlisting}[language=sh]
ssh username@atcgrid.ugr.es
\end{lstlisting}

Al ejecutar esta orden e introducir la contraseña, podemos acceder a nuestra shell personal.
Dentro de ésta es posible cambiar los ficheros \texttt{.bashrc} y \texttt{.bash\_profile} para asignar \texttt{alias} y herramientas que nos ayuden a trabajar más cómodamente.
De la misma forma, podemos utilizar \texttt{vim} para hacer ediciones rápidas a los scripts si se encontrara algún error en éstos.

\subsection{\texttt{sftp}}

Éste es un programa de transferencia segura de ficheros al que se puede acceder de forma similar a \texttt{ssh}:

\begin{lstlisting}[language=sh]
sftp username@atcgrid.ugr.es
\end{lstlisting}

Al trabajar con \texttt{sftp} distinguimos entre los directorios remotos, que no necesitan un tratamiento especial, y los locales, para los cuales debemos anteponer el carácter \texttt{l} en las órdenes que usemos.
Por ejemplo \texttt{ls} muestra el contenido del directorio de trabajo remoto (\texttt{pwd}) mientras que \texttt{lls} muestra el contenido del directorio de trabajo local (\texttt{lpwd}).

Para transferir ficheros lo hacemos con \texttt{put fichero} y los recuperamos con \texttt{get fichero}.

\section{Sistema de colas}

Para ejecutar programas en \textit{atcgrid} usamos el gestor de colas \href{https://slurm.schedmd.com/pdfs/summary.pdf}{\texttt{slurm}}.
Las órdenes de trabajo más básicas y las que usaremos a lo largo de estas prácticas son las siguientes:

\subsection{Ejecución en primer plano}

\begin{lstlisting}[language=sh]
srun -p ac ./HelloOMP
srun -p ac lscpu
\end{lstlisting}

La orden \texttt{srun} envía a ejecutar un trabajo a la cola especificada con \texttt{-p} (\texttt{ac} en el caso de esta asignatura) y muestra el resultado por la salida estándar.

\subsection{Ejecución en segundo plano}

\begin{lstlisting}[language=sh]
sbatch -p ac script.sh
sbatch -p ac --wrap "echo \"Hello World\""
sbatch -p ac --wrap "./HelloOMP"
sbatch -p ac --wrap "echo \"Hello World\" && ./HelloOMP"
\end{lstlisting}

La orden \texttt{sbatch} envía a ejecutar un script definido en un fichero o tras \texttt{--wrap} a la cola pertinente y devuelve la salida en un fichero.
Ésta es la forma de ejecución recomendada, ya que permite enviar varios trabajos a la cola en lugar de tener que esperar a que uno acabe para enviar otro.

\subsection{Trabajos sobre la cola}

\begin{lstlisting}[language=sh]
squeue
scancel jobid
sinfo
\end{lstlisting}

Con la orden \texttt{scancel} podemos cancelar un trabajo cuyo \texttt{jobid} hayamos previamente recuperado mediante la orden \texttt{squeue}.
Por otro lado, \texttt{sinfo} lista información sobre las colas y los nodos del \textit{atcgrid}.
