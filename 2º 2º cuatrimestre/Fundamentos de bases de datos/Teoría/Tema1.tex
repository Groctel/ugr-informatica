\chapter{Introducción y definiciones iniciales}

\section{Concepto intuitivo de Base de Datos}

Hasta este punto, a lo largo de la carrera, se ha estudiado el almacenamiento de datos en estructuras abstractas que recuperan la información de ficheros de texto.
Esta estrategia de almacenamiento de datos presenta tres problemas intrínsecos:

\begin{itemize}
	\item\textbf{Redundancia:} El mismo dato puede aparecer representado varias veces en el mismo fichero.
	\item\textbf{Inconsistencia:} No existe un control sobre la integridad de los datos redundantes. Una errata puede invalidad un dato redundante.
	\item\textbf{Falta de reutilización:} Los datos redundantes han de modificarse individualmente cada vez.
\end{itemize}

\begin{lstlisting}
PELICULA:DIRECTOR:AÑO;
Ferris Bueller's Day Off:John Hughes:1896;
The Breakfast Club:Jon Huges:1985;
Back To The Future:Robert Zemeckis:1985;
The Breakfast Club:John Hughes:1985;
\end{lstlisting}

En este ejemplo, el director de \textit{Ferris Bueller's Day Off} y \textit{The Breakfast Club}\footnote{Dos peliculones, director recomendadísimo.}, John Hughes, aparece escrito erróneamente (\texttt{Jon Huges}) en la segunda línea.
Debido a esto, se realiza una actualización de la misma que, por algún fallo típico de una práctica de Estructuras de Datos, añade el dato de nuevo en la última fila pero no elimina el anterior.
Si buscamos secuencialmente el \texttt{DIRECTOR} de \texttt{The Breakfast Club}, ¿deberíamos quedarnos con el primer resultado o recorrer el fichero en su plenitud hasta encontrar otro dato? ¿Cómo sabemos cuál es el correcto?
Por otro lado, tenemos dos películas estrenadas en 1985 pero, debido a este error, un análisis del fichero podría resolver que tenemos tres películas estrenadas en este año.

Una alternativa al uso de ficheros que solventa todos estos errores es el uso de bases de datos (\textbf{BD}) y sistemas de gestión de bases de datos (\textbf{SGBD}).
Definimos una \textbf{BD} como un \textit{conjunto de datos comunes a un proyecto almacenados sin redundancia para ser útiles a diferentes aplicaciones} y un \textbf{SGBD} como un \textit{conjunto de elementos software con capacidad para definir, mantener y utilizar una base de datos}.

Los SGBD debe permitir, como mínimo, definir estructuras de almacenamiento, acceder a los datos de su BD de forma eficiente y segura, organizar la actualización de los datos y el acceso multiusuario y otras funcionalidades que se verán a lo largo de la asignatura. Las operaciones que se pueden realizar sobre los SGBD son referidos como \texttt{CRUD}:

\begin{itemize}
	\item\texttt{Create}\textbf{:} Insertar datos en la BD\@.
	\item\texttt{Read}\textbf{:} Obtener datos previamente insertados en la BD\@.
	\item\texttt{Update}\textbf{:} Modificar los datos existenten en la BD\@.
	\item\texttt{Delete}\textbf{:} Borrar datos existenten en la BD\@.
\end{itemize}

Todas estas operaciones se realizan de forma transparente al usuario, es decir, éste no tiene que programar código adicional para manejar los ficheros que la conforman.

En resumen, una BD es un fondo común de información almacenada en un computador para que cualquier persona o programa autorizado pueda acceder a ella independientemente del lugar del procedencia y del uso que se haga de la misma.

\section{Bases de Datos y Sistemas de Gestión de Bases de Datos}

En una BD trabajamos con los siguientes elementos:

\begin{itemize}
	\item\textbf{Datos:} Deben representarse sin redundancia y ser compartidos de forma útil para múltiples programas que accedan a la BD\@.
	\item\textbf{Hardware:} La BD puede encontrarse en una sola máquina o ser una BD distribuida.
	\item\textbf{Software SGBD:} Los Sistemas de Gestión de Bases de Datos (DBMS, \textit{DataBase Management System}) son programas utilizados para describir las estructuras y gestionar la información de la BD\@.
	\item\textbf{Usuarios:} Distinguimos entre el usuario final, el programador de aplicaciones que acceden a la BD y el administrador, también llamado DBA o DBM (\textit{DataBase Administrator}/\textit{Manager}).
\end{itemize}

\subsection{Datos operativos}

Los datos operativos (referidos comunmente como \textbf{campos}) son piezas de información básica que necesita un proyecto (empresarial, de software\ldots) para su funcionamiento.
Distinguimos entre tres tipos de datos operativos:

\begin{itemize}
	\item\textbf{Ítem básico:} Sustantivos. Son elementos sobre los que se puede pedir información. \textit{Usuario}, \textit{Coche}.
	\item\textbf{Relaciones:} Verbos. Conexiones lógicas entre ítems. \textit{Usuario compra Coche}.
	\item\textbf{Atributos:} Adjetivos y adverbios. Características de los ítems básicos y las relaciones. \textit{Coche con \textbf{matrícula}}, \textit{compra por \textbf{una cantidad de euros}}.
\end{itemize}

De la determinación y clasificación de estos datos operativos se obtiene el esquema lógico de la BD, sobre el cual trabajaremos.

\section{Concepto de independencia}

En una BD los datos deben organizarse siempre independientemente de los programas que se espera que la usen y de los ficheros en los que se almacenen, ya que debe reflejar la realidad de la forma más fiel posible.

\subsection{Independencia física}

El diseño de la BD debe ser siempre independiente del almacenamiento físico de los datos.
De esta forma se podrán realizar cambios futuros en la estructura física de la BD sin alterar la lógica de la misma.
De la misma forma, esta independencia libera a los programas que accedan a ella de gestionar su almacenamiento.

\subsection{Independencia lógica}

En una BD distinguimos entre dos tipos de estructuras lógicas:

\begin{itemize}
	\item\textbf{Esquema lógico general:} Visión global de la BD a la que tienen acceso los administradores y desarrolladores.
	\item\textbf{Vistas de usuario:} Porción de los datos de la BD a los que los usuarios y programas externos tienen permisos de acceso.
\end{itemize}

Decimos que una BD goza de independencia lógica cuando los cambios en el esquema lógico general no implican necesariamente la actualización de las vistas de usuario.
De esta forma aumenta la seguridad y fiabilida de la BD, su actualización implica menos problemas para los programas externos y aumentan las posibilidades de cambios en los esquemas de los desarrolladores de programas y de los administradores.
Esta independencia no es siempre posible, ya que se pueden producir cambios en la estructura de tal magnitud que necesiten de actualizaciones en las vistas de usuario.

\section{Objetivos de un SGBD}

Los SGBD buscan la independencia física y lógica de los datos de la BD de forma que estos datos estén centralizados, gestionándose independientemente de los programas externos.

Mediante sistemas de gestión de accesos concurrentes se busca eliminar la redundancia de los datos.
También se busca que sea consistente, de forma que todas las copias de la información de un objeto en la misma sean idénticas, y que sea íntegra, de forma que todos los datos almacenados se correspondan lo máximo posible con la realidad que representan.

Otras dos características importentes de una BD aportadas por los SGBD son la fiabilidad y la seguridad.
La primera garantiza que los datos estén protegidos contra fallos catastróficos mediante mecanismos de mantenimiento, recuperación y relanzamiento de transiciones.
La segunda garantiza que no todos los datos sean accesibles por todos los usuarios mediante mecanimos de gestión de usuarios y privilegios y de protección de información.

Para los usuarios, los datos y programas deben ser accesibles de forma amigable para dar soporte a un modelo de datos teórico, facilidades de definición y lenguajes de acceso y modificación.
De esta forma, el usuario final puede acceder a los datos de la BD y el programador no tiene que lidiar con problemas de diseño lógico y físico, depuración de errores y mantenimiento.

Para el sistema, los SGBD buscan centralizar el control de la BD, crear criterior de uniformización, permitir crear nuevos programas que accedan a ella y encontrar un equilibro entre requerimientos conflictivos.
