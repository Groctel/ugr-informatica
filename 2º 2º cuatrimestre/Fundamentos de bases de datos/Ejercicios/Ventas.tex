\section{\code{Ventas}}

\subsubsection{Conocer el nombre de los clientes cuyo agente es del mismo área (\code{working\_area})}

\begin{lstlisting}[language=SQL]
SELECT Cliente.name
FROM   Agente, Cliente
WHERE  Agente.working_area = Cliente.working_area;
\end{lstlisting}

\[\pi_{Cliente.name}(Agente\bowtie_{working\_area}Cliente)\]

\subsubsection{Agentes con la comisión más alta}

\begin{lstlisting}[language=SQL]
SELECT *
FROM   Agente
WHERE  commission = (
	SELECT MAX(commission)
	FROM   Agente
);
\end{lstlisting}

\begin{lstlisting}[language=SQL]
SELECT *
FROM   Agente
WHERE  commission IN (
	SELECT MAX(commission)
	FROM   Agente
);
\end{lstlisting}

\begin{lstlisting}[language=SQL]
SELECT *
FROM   Agente
WHERE  commission >= ALL (
	SELECT commission
	FROM   Agente
);
\end{lstlisting}

\[
\begin{split}
	& \rho(Agente)=a1 \\
	& \rho(Agente)=a2 \\
	& \sigma(Agente)-\pi_{a1.commission}\big(\sigma_{a1.commission<a2.commisison}(a1\times a2)\big)
\end{split}
\]

\pagebreak

\subsubsection{Nombres de clientes de agentes con la comisión más baja}

\begin{lstlisting}[language=SQL]
SELECT Cliente.name
FROM   Agente, Cliente
WHERE  Agente.code       = Cliente.agent_code
AND    Agente.commission = (
	SELECT MIN(commission)
	FROM   Agente
);
\end{lstlisting}

\begin{lstlisting}[language=SQL]
SELECT Cliente.name
FROM   Agente, Cliente
WHERE  Agente.code       = Cliente.agent_code
AND    Agente.commission IN (
	SELECT MIN(commission)
	FROM   Agente
);
\end{lstlisting}

\begin{lstlisting}[language=SQL]
SELECT Cliente.name
FROM   Agente, Cliente
WHERE  Agente.code        = Cliente.agent_code
AND    Agente.commission <= ALL (
	SELECT MIN(commission)
	FROM   Agente
);
\end{lstlisting}

\[
\begin{split}
	& \rho(Agente)=a1 \\
	& \rho(Agente)=a2 \\
	& Cliente\bowtie \Big(Agente-\pi_{a1.commission}\big(\sigma_{a1.commission>a2.commisison}(a1\times a2)\big)\Big)
\end{split}
\]

\subsubsection{Ciudad de clientes que realizaron los pedidos más recientes (fecha del pedido = \code{ped\_date})}

\begin{lstlisting}[language=SQL]
SELECT city
FROM   Cliente, Pedido
WHERE  Cliente.code = Pedido.client_code
AND    ped_date = (
	SELECT MAX(ped_date)
	FROM   Pedido
);
\end{lstlisting}

\begin{lstlisting}[language=SQL]
SELECT city
FROM   Cliente, Pedido
WHERE  Cliente.code = Pedido.client_code
AND    ped_date IN (
	SELECT MAX(ped_date)
	FROM   Pedido
);
\end{lstlisting}

\begin{lstlisting}[language=SQL]
SELECT city
FROM   Cliente, Pedido
WHERE  Cliente.code = Pedido.client_code
AND    ped_date >= ALL (
	SELECT ped_date
	FROM   Pedido
);
\end{lstlisting}

\[
\begin{split}
	& \rho(Agente)=a1 \\
	& \rho(Agente)=a2 \\
	& \pi_{city}\bigg(Cliente\bowtie\Big(\sigma\big(Pedido\big)-\pi_{p1.ped\_date}\big(\sigma_{p1.ped\_date<p2.ped\_date}(p1\times p2)\big)\Big)\bigg)
\end{split}
\]

\subsubsection{Agentes que son responsables de los pedidos de todos los clientes de su misma área}

\begin{lstlisting}[language=SQL]
SELECT *
FROM   Agente
WHERE NOT EXISTS (
	SELECT *
	FROM   Agente
	WHERE NOT EXISTS (
		SELECT *
		FROM   Agente, Cliente, Pedido
		WHERE  Agente.code         = Pedido.agent_code
		AND    Pedido.client_code  = Cliente.code
		AND    Agente.working_area = Cliente.working_area
	)
);
\end{lstlisting}

\[Agente\bowtie Pedido\div Agente\bowtie_{working\_area}Cliente\]

\subsubsection{Nombres de Clientes y de agentes, junto con el país, que son del mismo país}

\begin{lstlisting}[language=SQL]
SELECT Agente.name, Cliente.name, Agente.country
FROM   Agente, Cliente
WHERE  Agente.country = Cliente.country;
\end{lstlisting}

\[\pi_{Agente.name,Cliente.name,Agente.country}(Agente\bowtie_{country}Cliente)\]

\subsubsection{Listado que contenga nombres de clientes de máxima calificación (grade) y de mínima calificación}

\begin{lstlisting}[language=SQL]
SELECT name
FROM   Cliente
WHERE  grade = (
	SELECT MAX(grade)
	FROM   Cliente
)
OR grade = (
	SELECT MIN(grade)
	FROM   Cliente
);
\end{lstlisting}

\subsubsection{Nombre del cliente que hizo el pedido más antiguo}

\begin{lstlisting}[language=SQL]
SELECT name
FROM   Cliente, Pedido
WHERE  Cliente.code = Pedido.client_code
AND    ped_date = (
	SELECT MIN(ped_date)
	FROM Pedido
);
\end{lstlisting}

\subsubsection{Nombre de agentes que sólo tienen un único cliente}

\begin{lstlisting}[language=SQL]
SELECT Agente.name
FROM   Agente, Cliente
WHERE  Agente.code = Cliente.agent_code
GROUP BY Agente.code
HAVING COUNT(DISTINCT Cliente.code) = 1
\end{lstlisting}

\subsubsection{Nombre de agentes que no fueron responsables de ningún pedido en agosto de 2008}

\begin{lstlisting}[language=SQL]
SELECT Agente.name
FROM   Agente
WHERE NOT EXISTS (
	SELECT *
	FROM   Pedido
	WHERE  Agente.code = Pedido.agent_code
	AND    ped_date BETWEEN '08/01/2008' AND '08/31/2008'
);
\end{lstlisting}
