\chapter{Álgebra relacional}

\section{BikeStores}

\subsubsection{Seleccionar productos con precio superior a 1000}

\begin{lstlisting}[language=SQL]
SELECT *
FROM   Products
WHERE  list_price > 1000;
\end{lstlisting}

\[\sigma_{list\_price>1000}(Products)\]

\subsubsection{Seleccionar nombres y correos electrónicos de tiendas de `Santa Cruz'}

\begin{lstlisting}[language=SQL]
SELECT store_name, email
FROM   Stores
WHERE  city = 'Santa Cruz';
\end{lstlisting}

\[\pi_{store\_name,email}(\sigma_{city='\text{Santa Cruz}'}(Stores))\]

\subsubsection{Seleccionar precio$+$descuento de todos los items vendidos}

\begin{lstlisting}[language=SQL]
SELECT list_price+discount AS "discounted_price"
FROM   Order_Items;
\end{lstlisting}

\[\pi_{(price+discount)/discounted\_price}(Order\_Items)\]

\subsubsection{Seleccionar el email del personal cuyo nombre y apellidos es `Fabiola Jackson'}

\begin{lstlisting}[language=SQL]
SELECT email
FROM   Staffs
WHERE  first_name = 'Fabiola'
AND    last_name  = 'Jackson';
\end{lstlisting}

\[\pi_{email}(\sigma_{first\_name='\text{Fabiola}'\land last\_name='\text{Jackson}'}(Staffs))\]

\subsubsection{Seleccionar el personal que no tenga manager. Es obligatorio hacer uso de la condición \code{IS NULL} en la consulta}

\begin{lstlisting}[language=SQL]
SELECT *
FROM   Staffs
WHERE  manager_id IS NULL;
\end{lstlisting}

\[\sigma_{manager\_id=null}(Staffs)\]

\subsubsection{Seleccionar nombre y apellidos de personal (Staffs) que hayan procesado pedidos de clientes de la ciudad de `New York'}

\begin{lstlisting}[language=SQL]
SELECT Staffs.first_name, Staffs.last_name
FROM   Customers, Orders, Staffs
WHERE  Staffs.staff_id    = Orders.staff_id
AND    Orders.customer_id = Customers.customer_id
AND    city               = 'New York';
\end{lstlisting}

\[\pi_{Staffs.first\_name,Staffs.last\_name}(Staffs\bowtie Orders\bowtie\sigma_{city='New\ York'}(Customers))\]

\subsubsection{Seleccionar nombres de tiendas que tienen productos de la categoría `Mountain Bikes'}

\begin{lstlisting}[language=SQL]
SELECT store_name
FROM   Categories, Products, Stocks, Stores
WHERE  Stores.store_id          = Stocks.store_id
AND    Stocks.product_id        = Products.product_id
AND    Products.category_id     = Categories.category_id
AND    Categories.category_name = 'Mountain Bikes';
\end{lstlisting}

\[\pi_{store\_name}(Stores\bowtie Stocks\bowtie Products\bowtie\sigma_{category\_name='\text{Mountain Bikes}'}(Categories))\]

\subsubsection{Seleccionar nombres de productos vendidos, y cantidad, a clientes que viven en `Bronx'}

\begin{lstlisting}[language=SQL]
SELECT product_name, quantity
FROM   Customers, Products, Order_Items, Orders
WHERE  Customers.customer_id  = Orders.customer_id
AND    Orders.order_id        = Order_Items.order_id
AND    Order_Items.product_id = Products.product_id
AND    Customers.city         = 'Bronx';
\end{lstlisting}

\[\pi_{product\_name,quantity}(\sigma_{city='\text{Bronx}'}Customers\bowtie Orders\bowtie Order\_Items\bowtie Products)\]

\pagebreak

\subsubsection{Seleccionar nombre y apellidos de clientes que han comprado productos de la categoría `Mountain Bikes'}

\begin{lstlisting}[language=SQL]
SELECT DISTINCT first_name, last_name
FROM   Categories, Customers, Order_Items, Orders, Products
WHERE  Customers.customer_id    = Orders.customer_id
AND    Orders.order_id          = Order_Items.order_id
AND    Order_Items.product_id   = Products.product_id
AND    Products.category_id     = Categories.category_id
AND    Categories.category_name = 'Mountain Bikes';
\end{lstlisting}

\[
\begin{split}
\pi_{first\_name,last\_name}(Customers\bowtie Orders\bowtie Order\_Items\bowtie \\
Products\bowtie\sigma_{category\_name='\text{Mountain Bikes}'}(Category))
\end{split}
\]

\subsubsection{Seleccionar las marcas que ha vendido el personal de la tienda localizada en `Santa Cruz'}

\begin{lstlisting}[language=SQL]
SELECT DISTINCT brand_name
FROM   Brands, Order_Items, Orders, Products, Stores
WHERE  Stores.store_id        = Orders.store_id
AND    Orders.order_id        = Order_Items.order_id
AND    Order_Items.product_id = Products.product_id
AND    Products.brand_id      = Brands.brand_id
AND    city                   = 'Santa Cruz';
\end{lstlisting}

\[\pi_{brand\_name}(\sigma_{city='\text{Santa Cruz}'}Stores\bowtie Orders\bowtie Order\_Items\bowtie Products)\]

\subsubsection{Seleccionar Descuento mínimo, máximo y promedio que se hace de los items vendidos}

\begin{lstlisting}[language=SQL]
SELECT MIN(discount), MAX(discount), AVG(discount)
FROM   Order_Items;
\end{lstlisting}

\[\pi_{\min(discount),\max(discount),\text{avg}(discount)}(Order\_Items)\]

\subsubsection{Seleccionar Descuento mínimo, máximo y promedio que se hace de los items vendidos a personas de `New York'}

\begin{lstlisting}[language=SQL]
SELECT MIN(discount), MAX(discount), AVG(discount)
FROM   Customers, Order_Items, Orders
WHERE  Order_Items.order_id = Orders.order_id
AND    Orders.customer_id   = Customers.customer_id
AND    Customers.city       = 'New York';
\end{lstlisting}

\[\pi_{\min(discount),\max(discount),\text{avg}(discount)}(Order\_Items\bowtie Orders\bowtie\sigma_{city='\text{New York}'}(Customers))\]

\subsubsection{Seleccionar marcas que tengan algún producto que no esté en stock}

\begin{lstlisting}[language=SQL]
SELECT DISTINCT brand_name
FROM   Products, Brands
WHERE  Products.brand_id = Brands.brand_id
AND NOT EXISTS (
	SELECT *
	FROM   Stocks
	WHERE  Stocks.product_id = Products.product_id
);
\end{lstlisting}

\[\pi_{brand\_name}(Brands\bowtie Products)-\pi{brand\_name}(Brand\bowtie Products\bowtie Stocks)\]

\subsubsection{Seleccionar categorías que no tengan productos que no estén en stock (PISTA\@: Usar consultas anidadas haciendo uso de \code{NOT EXISTS})}

\begin{lstlisting}[language=SQL]
SELECT *
FROM   Categories
WHERE NOT EXISTS (
	SELECT *
	FROM   Products p
	WHERE  Categories.category_id = p.category_id
	AND NOT EXISTS (
		SELECT *
		FROM   Stocks
		WHERE  product_id = p.product_id
	)
);
\end{lstlisting}

\[\sigma(Categories)-((Categories\bowtie\rho_p(Products))-(p\bowtie Stocks))\]

\subsubsection{Seleccionar nombres de productos que no hayan sido vendidos}

\begin{lstlisting}[language=SQL]
SELECT product_name
FROM   Products
WHERE NOT EXISTS (
	SELECT *
	FROM   Order_Items
	WHERE  Products.product_id = Order_Items.product_id
);
\end{lstlisting}

\[\pi_{product\_name}(Products)-(Products\bowtie Order\_Items)\]

\pagebreak

\subsubsection{Seleccionar nombre de clientes que hayan comprado la máxima cantidad de algún producto, junto con el nombre del producto}

\begin{lstlisting}[language=SQL]
SELECT DISTINCT first_name, product_name
FROM   Customers, Order_Items, Orders, Products
WHERE  Customers.customer_id  = Orders.customer_id
AND    Orders.order_id        = Order_Items.order_id
AND    Order_Items.product_id = Products.product_id
AND    quantity IN (
	SELECT MAX(quantity)
	FROM   Order_Items
);
\end{lstlisting}

\[\pi_{first\_name,product\_name}(Customers\bowtie Orders\bowtie\sigma_{quantity=\max(quantity)}Order\_Items\bowtie Products)\]

\subsubsection{Seleccionar nombre y apellidos de personal que no haya vendido nada en (es decir, que no haya procesado pedidos en ese año)}

\begin{lstlisting}[language=SQL]
SELECT first_name, last_name
FROM   Staffs s
WHERE NOT EXISTS (
	SELECT *
	FROM   Orders
	WHERE  Orders.staff_id = s.staff_id
);
\end{lstlisting}

\[\pi_{first\_name,last\_name}(\rho_s(Staffs))-(s\bowtie Orders)\]

\subsubsection{Seleccionar nombres de tiendas y clientes que sean de la misma ciudad}

\begin{lstlisting}[language=SQL]
SELECT store_name, first_name
FROM   Customers, Stores
WHERE  Customers.city = Stores.city;
\end{lstlisting}

\[\pi_{Customers.first\_name,Stores.store\_name}(\sigma_{Customers.city=Stores.city}(Customers\times Stores))\]

\subsubsection{Seleccionar nombres de tiendas y clientes que sean de la misma ciudad, y donde los clientes hayan hecho pedidos a otras tiendas que no sean esas}

\begin{lstlisting}[language=SQL]
SELECT store_name, first_name
FROM   Customers c, Stores s
WHERE  c.city = s.city
AND EXISTS (
	SELECT *
	FROM   Orders
	WHERE  c.customer_id    = Orders.customer_id
	AND    Orders.store_id != s.store_id
);
\end{lstlisting}

\[\pi_{store\_name,first\_name}(\sigma_{c.city=s.city}(\rho_c(Customers)\times\rho_s(Stores)))\cap(c\bowtie Orders\triangleright s)\]

\subsubsection{Seleccionar nombres de clientes y su ciudad que hayan comprado los productos más caros.}

\begin{lstlisting}[language=SQL]
SELECT DISTINCT first_name, city
FROM   Customers, Order_Items, Orders, Products
WHERE  Customers.customer_id = Orders.customer_id
AND    Orders.order_id = Order_Items.order_id
AND    Order_Items.product_id = Products.product_id
AND    Products.list_price IN (
	SELECT MAX(list_price)
	FROM   Products
);
\end{lstlisting}

\[\pi_{first\_name,city}(Customers\bowtie Orders\bowtie Order\_Items\bowtie \sigma_{list\_price=\max(list\_price)}(Products))\]

\subsubsection{Seleccionar nombre de tiendas que tengan en stock productos de al menos dos marcas distintas}

\begin{lstlisting}[language=SQL]
SELECT DISTINCT store_name
FROM   Stocks, Stores, Products, Brands b1, Brands b2
WHERE  Stores.store_id   = Stocks.store_id
AND    Stocks.product_id = Products.product_id
AND    Products.brand_id = b1.brand_id
AND    Products.brand_id = b2.brand_id
AND    b1.brand_id != b2.brand_id;
\end{lstlisting}

\[\pi_{store\_name}(Stores\bowtie Stocks\bowtie Products\bowtie\sigma_{b1.brand\_id\neq b2.brand\_id}\rho_{b1,b2}(Brands))\]

\subsubsection{Seleccionar nombre y teléfono de clientes que sean de `New York' o de `Bronx'}

\begin{lstlisting}[language=SQL]
SELECT first_name, phone
FROM
	(
		(
			SELECT *
			FROM   Customers
			WHERE  city='New York'
		)
		UNION
		(
			SELECT *
			FROM   Customers
			WHERE  city='Bronx'
		)
	);
\end{lstlisting}

\[\pi_{first\_name,phone}((\sigma_{city='\text{New York}'}(Customers))\cup(\sigma_{city='\text{Bronx}'}(Customers)))\]

\pagebreak

\subsubsection{Encontrar el producto de precio máximo USANDO OPERADORES CONJUNTISTAS (sin usar operadores adicionales como min, max, all, etc.)}

\begin{lstlisting}[language=SQL]
SELECT product_name
FROM
	(
		(
			SELECT *
			FROM   Products
		)
		MINUS
		(
			SELECT p1.*
			FROM   Products p1, Products p2
			WHERE  p1.list_price < p2.list_price
		)
	);
\end{lstlisting}

\[\pi_{product\_name}(\sigma(Products)-\pi_{p1}(\sigma_{p1.list\_price<p2.list\_price}(\rho_{p1,p2}(Products))))\]

\subsubsection{Encontrar los nombres de productos que no estén en stock USANDO OPERADORES CONJUNTISTAS}

\begin{lstlisting}[language=SQL]
SELECT product_name
FROM
	(
		(
			SELECT *
			FROM   Products
		)
		MINUS
		(
			SELECT Products.*
			FROM   Products, Stocks
			WHERE  Products.product_id = Stocks.product_id
		)
	);
\end{lstlisting}

\[\pi_{product\_name}(\sigma(Products)-\pi_{Products}(Products\bowtie Stocks))\]

\pagebreak

\subsubsection{Encontrar los nombres de productos que hayan sido vendidos por al menos dos tiendas distintas}

\begin{lstlisting}[language=SQL]
SELECT DISTINCT product_name
FROM
	(
		(
			SELECT Products.*
			FROM   Order_Items, Orders, Products, Stores
			WHERE  Stores.store_id = Orders.store_id
			AND    Orders.order_id = Order_Items.order_id
			AND    Order_Items.product_id = Products.product_id
		)
		MINUS
		(
			SELECT Products.*
			FROM   Order_Items, Orders, Products, Stores
			WHERE  Stores.store_id = Orders.store_id
			AND    Orders.order_id = Order_Items.order_id
			AND    Order_Items.product_id = Products.product_id
			AND NOT EXISTS (
				SELECT *
				FROM   Stores s1, Stores s2
				WHERE  s1.store_id = s2.store_id
			)
		)
	);
\end{lstlisting}

\[
\begin{split}
\pi_{product\_name}((\pi_{Products}(Stores\bowtie Orders\bowtie Order\_Items\bowtie Products))- \\
(\pi_{Products}((\rho_{s1,s2}(Stores)\bowtie Orders\bowtie Order\_Items\bowtie Products))-\sigma_{s1=s1}(Stores)))
\end{split}
\]

\subsubsection{Marcas que tengan productos de todas las categorías}

\begin{lstlisting}[language=SQL]
SELECT DISTINCT *
FROM   Products
AND NOT EXISTS (
	(
		SELECT *
		FROM   Categories
	)
	MINUS
	(
		SELECT *
		FROM   Categories
		WHERE  Categories.category_id = Products.category_id
	)
);
\end{lstlisting}

\[\sigma(Products)\div\sigma(Categories)\]

\pagebreak

\subsubsection{Categorías que tengan productos de todas las marcas}

\begin{lstlisting}[language=SQL]
SELECT DISTINCT *
FROM   Categories
AND NOT EXISTS (
	(
		SELECT *
		FROM   Products, Brands
	)
	MINUS
	(
		SELECT *
		FROM   Products, Brands
		WHERE  Brands.brand_id      = Products.product_id
		AND    Products.category_id = Categories.category_id
	)
);
\end{lstlisting}

\[\sigma(Categories)\div(Products\bowtie Brands)\]

\subsubsection{Clientes que hayan realizado pedidos en todas las tiendas}

\begin{lstlisting}[language=SQL]
SELECT first_name, last_name
FROM   Customers
WHERE NOT EXISTS (
	(
		SELECT *
		FROM   Orders, Stores
	)
	MINUS
	(
		SELECT *
		FROM   Orders, Stores
		WHERE  Stores.store_id    = Orders.store_id
		AND    Orders.customer_id = Customers.customer_id
	)
);
\end{lstlisting}

\[\pi_{first\_name,last\_name}(Customers)\div(Orders\bowtie Stores)\]

\pagebreak

\subsubsection{Tiendas (id y nombre) que hayan vendido todos los productos de la marca `Strider'}

\begin{lstlisting}[language=SQL]
SELECT store_id, store_name
FROM   Stores
WHERE NOT EXISTS (
	(
		SELECT *
		FROM   Brands
		WHERE  brand_name = 'Strider'
	)
	MINUS
	(
		SELECT Brands.*
		FROM   Brands, Order_Items, Orders, Products
		WHERE  Stores.store_id        = Orders.store_id
		AND    Orders.order_id        = Order_Items.order_id
		AND    Order_Items.product_id = Products.product_id
		AND    Products.brand_id      = Brands.brand_id
	)
);
\end{lstlisting}

\[\pi_{store\_id,store\_name}(Stores)\div(Orders\bowtie Order\_Items\bowtie Products\bowtie\sigma_{brand\_name='\text{Strider}'}(Brands))\]

\subsubsection{Tiendas que hayan vendido productos de todas las categorías}

\begin{lstlisting}[language=SQL]
SELECT *
FROM   Stores
WHERE NOT EXISTS (
	(
		SELECT *
		FROM   Categories
	)
	MINUS
	(
		SELECT Categories.*
		FROM   Categories, Order_Items, Orders, Products
		WHERE  Stores.store_id        = Orders.store_id
		AND    Orders.order_id        = Order_Items.order_id
		AND    Order_Items.product_id = Products.product_id
		AND    Products.category_id   = Categories.category_id
	)
);
\end{lstlisting}

\[\sigma(Stores)\div(Orders\bowtie Order\_Items\bowtie Products\bowtie Categories)\]

\pagebreak

\subsubsection{Tiendas que hayan vendido productos de todas las marcas}

\begin{lstlisting}[language=SQL]
SELECT *
FROM   Stores
WHERE NOT EXISTS (
	(
		SELECT Brands.*
		FROM   Brands
	)
	MINUS
	(
		SELECT Brands.*
		FROM   Order_Items, Orders, Products, Brands
		WHERE  Stores.store_id        = Orders.store_id
		AND    Orders.order_id        = Order_Items.order_id
		AND    Order_Items.product_id = Products.product_id
		AND    Products.brand_id      = Brands.brand_id
	)
);
\end{lstlisting}

\[\sigma(Stores)\div(Orders\bowtie Order\_Items\bowtie Products\bowtie Brands)\]

\subsubsection{Categorías que hayan sido vendidas por todas las tiendas}

\begin{lstlisting}[language=SQL]
SELECT *
FROM   Categories
WHERE NOT EXISTS (
	(
		SELECT Stores.*
		FROM   Stores
	)
	MINUS
	(
		SELECT Stores.*
		FROM   Order_Items, Orders, Products, Stores
		WHERE  Stores.store_id        = Orders.store_id
		AND    Orders.order_id        = Order_Items.order_id
		AND    Order_Items.product_id = Products.product_id
		AND    Products.category_id   = Categories.category_id
	)
);
\end{lstlisting}

\[\sigma(Categories)\div(Stores\bowtie Orders\bowtie Order\_Items\bowtie Products\bowtie Categories)\]

\pagebreak

\section{Agentes}

\subsubsection{Conocer el nombre de los clientes cuyo agente es del mismo área (working area)}

\begin{lstlisting}[language=SQL]
SELECT Cliente.name
FROM   Agentes, Cliente
WHERE  Agentes.code         = Cliente.agent_code
AND    Agentes.working_area = Cliente.working_area;
\end{lstlisting}

\[\pi_{Cliente.name}\sigma_{Agentes.working\_area=Cliente.working\_area}(Agentes\bowtie Clientes)\]

\subsubsection{Agentes con la comisión más alta}

\begin{lstlisting}[language=SQL]
SELECT *
FROM   Agentes
WHERE  commission IN (
	SELECT MAX(commission)
	FROM   Agentes
);
\end{lstlisting}

\[\sigma_{commission=\max(commission)}(Agentes)\]

\subsubsection{Nombres de clientes de agentes con la comisión más baja}

\begin{lstlisting}[language=SQL]
SELECT Cliente.name
FROM   Agentes, Cliente
WHERE  Agentes.code = Cliente.agent_code
AND    commission IN (
	SELECT MIN(commission)
	FROM   Agentes
);
\end{lstlisting}

\[\pi_{Cliente.name}(\sigma_{commission=\max(commission)}(Agentes)\bowtie Cliente)\]

\subsubsection{Ciudad de clientes que realizaron los pedidos más recientes (fecha del pedido=ped\_date)}

\begin{lstlisting}[language=SQL]
SELECT city
FROM   Cliente, Pedidos
WHERE  Cliente.code = Pedidos.client_code
AND    Pedidos.ped_date IN (
	SELECT MAX(ped_date)
	FROM   Pedidos
);
\end{lstlisting}

\[\pi_{city}(\sigma_{ped\_date=\max(ped\_date)}(Cliente)\bowtie Pedidos)\]

\subsubsection{Agentes que son responsables de los pedidos de todos los clientes de su misma área}

\begin{lstlisting}[language=SQL]
SELECT Agentes.*
FROM   Agentes, Cliente, Pedidos
WHERE  Pedidos.agent_code   = Agentes.code
AND    Pedidos.client_code  = Cliente.code
AND    Agentes.working_area = Cliente.working_area;
\end{lstlisting}

\[\pi_{Agentes}(\sigma_{Agentes.working\_area=Cliente.working\_area}(Agentes\bowtie Pedidos\bowtie Cliente))\]

\subsubsection{Nombres de Clientes y de agentes, junto con el país, que son del mismo país}

\begin{lstlisting}[language=SQL]
SELECT Agentes.name, Cliente.name, Agentes.country
FROM   Agentes, Cliente
WHERE  Agentes.code    = Cliente.agent_code
AND    Agentes.country = Cliente.country;
\end{lstlisting}

\[\pi_{Agentes.name,Cliente.name,Agentes.country}(\sigma_{Agentes.country=Cliente.country}(Agentes\bowtie Cliente))\]

\subsubsection{Listado que contenga nombres de clientes de máxima calificación (grade) y de mínima calificación}

\begin{lstlisting}[language=SQL]
SELECT MAX(grade), MIN(grade)
FROM   Cliente
\end{lstlisting}

\[\pi_{\max(grade),\min(grade)}(Cliente)\]

\subsubsection{Nombre del cliente que hizo el pedido más antiguo}

\begin{lstlisting}[language=SQL]
SELECT name
FROM   Cliente, Pedidos
WHERE  Cliente.code = Pedidos.client_code
AND    Pedidos.ped_date IN (
	SELECT MAX(ped_date)
	FROM   Pedidos
);
\end{lstlisting}

\[\pi_{name}(Cliente\bowtie\sigma_{ped\_date=\max(ped\_date)}(Pedidos))\]

\pagebreak

\subsubsection{Nombre de agentes que sólo tienen un único cliente}

\begin{lstlisting}[language=SQL]
SELECT name
FROM
	(
		SELECT Agentes.name, COUNT(*) AS total
		FROM   Agentes, Cliente
		WHERE  Agentes.code = Cliente.agent_code
		GROUP  BY Agentes.name
	)
WHERE total = 1;
\end{lstlisting}

\[\pi_{name}\sigma_{total=1}(G_{name,count(\star)}(Agentes)\bowtie Cliente)\]

\subsubsection{Nombre de agentes que no fueron responsables de ningún pedido en agosto de 2008}

\begin{lstlisting}[language=SQL]
SELECT DISTINCT Agentes.name
FROM Agentes
WHERE NOT EXISTS (
	SELECT *
	FROM   Pedidos
	WHERE  Agentes.code  = Pedidos.agent_code
	AND    ped_date     <= '01-AUG-2008'
	AND    ped_date     >= '31-AUG-2008'
);
\end{lstlisting}

\[\pi_{name}(Agentes)-(Agentes\bowtie\sigma_{'\text{01-AUG-2008}'\leq ped\_date\land'\text{31-AUG-2008}'\geq ped\_date}(Pedidos))\]
