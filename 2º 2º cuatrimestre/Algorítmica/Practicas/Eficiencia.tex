\chapter{Cálculo de la eficiencia de algoritmos}\label{eficiencia}

\section{Eficiencia teórica}\label{eficiencia-teorica}

\subsection{Algoritmos iterativos}\label{eficiencia-teorica-iterativos}

Calcular la eficiencia teórica en el peor y el mejor caso para los siguientes algoritmos:

\subsubsection{Algoritmo iterativo 1}

\begin{lstlisting}[language=C]
int pivotar (double* v, const int ini, const int fin) {
	double pivote = v[init],
	       aux;
	int i = ini+1,
	    j = fin;

	while (i <= j) {
		while (v[i] < pivote && i <= j)
			i++;

		while (v[j] >= pivote && j >= i)
			j--;

		if (i < j) {
			aux  = v[i];
			v[i] = v[j];
			v[j] = aux;
		}
	}

	if (j > ini) {
		v[ini] = v[j];
		v[j]   = pivote;
	}

	return j;
}
\end{lstlisting}

\textbf{Tamaño del caso:}

El tamaño del caso es $j-i$, siendo $i$ la posición de inicio del vector con el que se trabaja y $j$ la posición de final, ambas inclusive.

\textbf{Eficiencia en el caso peor:}

En este caso, el algoritmo entra en el bucle \texttt{while (i <= j)}, que está compuesto de dos bucles que analizamos individualmente:

\begin{itemize}
	\item\texttt{while (v[i] < pivote \&\& i </ j)}\textbf{:} La ejecución más larga posible de este bucle se da cuando \texttt{v[i]} siempre es menor que \texttt{pivote}, de forma que se ejecuta el bucle hasta que \texttt{i} es igual que \texttt{j}, iterando ascendentemente por todo el vector menos el primer elemento, de forma que la eficiencia es $O(n-1)=O(n)$.
	\item\texttt{while (v[j] >= pivote \&\& j >= i)}\textbf{:} La ejecución más larga posible de este vector se da cuando \texttt{v[j]} siempre es mayor o igual que \texttt{pivote}, iterando descendentemente por todo el vector menos el primer elemento, de forma que la eficiencia es $O(n-1)=O(n)$.
\end{itemize}

Por la regla del máximo, la eficiencia del cuerpo del bucle es $O(n)$.
Al ejecutarse el cuerpo en el caso peor, tenemos que \texttt{i} es mayor que \texttt{j}, por lo que el bucle \texttt{while (i <= j)} se ejecuta una única vez, siendo su eficiencia $O(n)$.

De esta forma determinamos que \textbf{la eficiencia de este algoritmo en caso peor es $\boldsymbol{O(n)}$}.

\textbf{Eficiencia en el caso mejor:}

En el caso mejor, el algoritmo no llega a entrar en el primer bucle, ya que \texttt{i} es mayor que \texttt{j}, por lo que sólo se ejecutan operaciones en tiempo constante.
\textbf{La eficiencia de este algoritmoe en el caso mejor es $\boldsymbol{\Omega(1)}$}.

\subsubsection{Algoritmo iterativo 2}

\begin{lstlisting}[language=C]
int Busqueda (int* v, int n, int elem) {
	int inicio = 0,
	    fin    = n - 1,
		 centro = (inicio + fin) / 2;

	while (inicio <= fin && v[centro] != elem) {
		if (elem < v[centro])
			fin = centro - 1;
		else
			inicio = centro + 1;

		centro = (inicio + fin) / 2;
	}

	if (inicio > fin)
		return -1;

	return centro;
}
\end{lstlisting}

\textbf{Tamaño del caso:}

El tamaño del caso es \texttt{n}, ya que se toma el \texttt{inicio} del vector como \texttt{0} y el \texttt{fin}, como \texttt{n-1}.

\textbf{Eficiencia en el caso peor:}

En el caso peor el algoritmo entra en el bucle \texttt{while} y éste se ejecuta hasta que \texttt{inicio} sea mayor que \texttt{fin}, es decir, no se encuentra \texttt{elem}.
En este caso, el bucle ejecuta una búsqueda binaria a lo largo del vector, posicionando \texttt{elem} en una de las dos mitades, ignorando la otra y dividiendo la mitad en la que se encuentra \texttt{elem} en dos hasta que sólo quedan dos elementos y se descartan ambos.
Por tanto, \textbf{la eficiencia de este algoritmo en el caso peor es $\boldsymbol{O(logn)}$}.

\textbf{Eficiencia en el caso mejor:}

En el caso mejor, \texttt{v[centro]} es el valor que busca \texttt{elem}, por lo que no es necesario entrar en el bucle y el resto de operaciones se ejecutan en tiempo constante.
\textbf{La eficiencia de este algoritmo en el caso mejor es $\boldsymbol{\Omega(1)}$}.

\pagebreak

\subsubsection{Algoritmo iterativo 3}

\begin{lstlisting}[language=C]
void EliminaRepetidos (double original[], int & nOriginal) {
	for (int i=0; i<nOriginal; i++) {
	//	Buscamos el valor repetido de original[i] desde original [i+1] hasta el final.
		int j = i+1;

		do {
			if (original[j] == original[i]) {
			/*
			 *	Desplazamos todas las componentes desde j+1 hasta el final,
			 *	una componente a la izquierda.
			 */
				for (int k=j+1; k<nOriginal; k++)
					original[k-1] = original[k];
			/*
			 *	Como hemos eliminado una componente, reducimos el número de
			 *	componenetes útiles.
			 */
				nOriginal--;
			}
			else {
			//	Si el valor no está repetido, pasamos al siguiente j
				j++;
			}
		} while (j < nOriginal);
	}
}
\end{lstlisting}

\textbf{Tamaño del caso:}

El tamaño del caso es \texttt{nOriginal}, el número de elementos del vector \texttt{original}.

\textbf{Eficiencia en el caso peor:}

En este caso, todos los elementos del vector son iguales y deben eliminarse, entrando en el bloque \texttt{if (original[j] == original[i])} tantas veces como elementos haya menos uno.
Este bloque contiene un bucle \texttt{for} cuyo cuerpo, inicialización, comprobación y actualización se ejecutan en tiempo constante.
Este bucle se ejecuta de \texttt{i+2} a \texttt{nOriginal}, avanzando de uno en uno, por lo que su eficiencia es $O(n)$.

En cualquier caso, el bucle \texttt{do while} siempre se ejecuta \texttt{nOriginal-j} veces, ya sea por incremento de \texttt{j} o decremento de \texttt{nOriginal}, por lo que la eficiencia de este bucle es siempre $O(n)$.
Teniendo esto en cuenta y siguiendo la regla del producto, determinamos que la eficiencia total del buclde \texttt{do while} es $O(n^2)$ y que \textbf{la eficiencia de este algoritmo en el caso peor es $\boldsymbol{O(n^2)}$}.

\textbf{Eficiencia en el caso mejor:}

Como hemos visto en el análisis del caso peor, el bucle \texttt{do while} siempre se ejecuta con eficiencia $O(n)$.
En el caso mejor, no hay ningún elemento repetido, por lo que el bucle \texttt{for} del cuerpo no se ejecuta nunca y la eficiencia total del cuerpo es $O(1)$.
Por tanto, teniendo en cuenta la regla del producto, \textbf{la eficiencia de este algoritmo en el caso mejor es $\boldsymbol{\Omega(n)}$}.

\pagebreak

\subsection{Algoritmos recursivos}\label{eficiencia-recursivos}

Analizar la eficiencia y calcular la ecuación en recurrencias para los siguientes algoritmos:

\subsubsection{Algoritmo recursivo 1}

\begin{lstlisting}[language=C]
void hanoi (int M, int i, int j) {
	if (M > 0) {
		hanoi(M-1, i, 6-i-j);
		printf("%d -> %d\n", i, j);
		hanoi(M-1, 6-i-j, j);
	}
}
\end{lstlisting}

Identificamos los casos:

\[T(n)=
\begin{cases}
	1         & n=0\text{ (caso base)} \\
	1+2T(n-1) & n>1\text{ (caso general)}
\end{cases}
\]

Nos queda la siguiente ecuación lineal no homogénea:

\[T(n)=2T(n-1)+1\Leftrightarrow T(n)-2T(n-1)=1\]

Calculamos las raíces de la parte homogénea tomando su polinomio característico:

\[T(n)-2T(n-1)=0\Rightarrow x^n-2x^{n-1}=0\Leftrightarrow(x-2)x^{n-1}=0\Leftrightarrow x-2=0\Leftrightarrow x=2\]

Calculamos las raíces de la parte no homogénea:

\[1=b_1^n\cdot q_1(n)\Rightarrow b_1=1,q_1(n)=1\]

Tenemos por tanto dos raíces en nuestro polinomio característico:

\begin{itemize}
	\item $x_1=2, m_1=1$
	\item $x_2=2, m_1=1$
\end{itemize}

Queda la siguiente ecueción en recurrencias:

\[T_n=c_{1,0}2^n n^0+c_{2,0}1^n n^0=2^n c_{1,0}+c_{2,0}\]

Por tanto, \textbf{la eficiencia de este algoritmo es $\boldsymbol{O(2^n)}$}.

\pagebreak

\subsubsection{Algoritmo recursivo 2}

\begin{lstlisting}[language=C]
void HeapSort(int* v, int n) {
	double* apo = new double[n];
	int tamapo  = 0;

	for (int i=0; i<n; i++) {
		apo[tamapo] = v[i];
		tamapo++;
		insertarEnPos(apo, tamapo);
	}

	for (int i=0; i<n; i++) {
		v[i] = apo[0];
		tamapo--;
		apo[0] = apo[tamapo];
		reestructurarRaiz(apo, 0, tamapo);
	}

	delete [] apo;
}
\end{lstlisting}

\begin{lstlisting}[language=C]
void insertarEnPos (double* apo, int pos) {
	int idx = pos - 1,
	    padre;

	if (idx > 0) {
		if (idx % 2 == 0)
			padre = (idx - 2) / 2;
		else
			padre = (idx - 1) / 2;

		if (apo[padre] > apo[idx]) {
			double tmp = apo[idx];
			apo[idx]   = apo[padre];
			apo[padre] = tmp;
			insertarEnPos(apo, padre+1);
		}
	}
}
\end{lstlisting}

\begin{lstlisting}[language=C]
void reestructurarRaiz (double* apo, int pos, int tamapo) {
	int minhijo;

	if (2*pos+1 < tamapo) {
		minhijo = 2 * pos + 1;

		if (minhijo+1 < tamapo && apo[minhijo] > apo[minhijo+1])
			minhijo++;

		if (apo[pos] > apo[minhijo]) {
			double tmp   = apo[pos];
			apo[pos]     = apo[minhijo];
			apo[minhijo] = tmp;
			reestructurarRaiz(apo, minhijo, tamapo);
		}
	}
}
\end{lstlisting}

Vamos a dividir este algoritmo en tres partes, calculando primero la eficiencia de \texttt{reestructurarRaiz}, luego la de \texttt{insertarEnPos} y, finalmente, la de \texttt{HeapSort}.

\textbf{\texttt{insertarEnPos}:}

Identificamos los casos:

\[T(n)=
\begin{cases}
	1                & n\leq1\text{ (caso base)} \\
	1+T(\frac{n}{2}) & n>1\text{ (caso general)}
\end{cases}
\]

Nos queda la siguiente ecuación lineal no homogénea:

\[T(n)=T(\frac{n}{2})+1\Leftrightarrow T(n)-T(\frac{n}{2})=1\]

Hacemos el cambio de variable $n=2^m$ y calculamos las raíces de la parte homogénea tomando su polinomio característico:

\[T(2^m)-T(2^{m-1})=0\Rightarrow x^{2m}-x^{2m-1}=0\Leftrightarrow(x-1)x^{2m-1}=0\Leftrightarrow x-1=0\Leftrightarrow x=1\]

Calculamos las raíces de la parte no homogénea:

\[1=b_1^n\cdot q_1(n)\Rightarrow b_1=1,q_1(n)=1\]

Tenemos por tanto dos raíces en nuestro polinomio característico:

\begin{itemize}
	\item $x_1=2, m_1=2$
\end{itemize}

Queda la siguiente ecueción en recurrencias:

\[T_m=c_{1,0}1^m m^0+c_{1,1}1^m m^1=1^m c_{1,0}+m1^m c_{1,1}\]

Deshacemos el cambio de variable:

\[T_n=1^{\log_2n}c_{1,0}+\log_2n\cdot1^{\log_2n}c_{1,1}\]

Por tanto, \textbf{la eficiencia de este algoritmo es $\boldsymbol{O(\log n)}$}.

\textbf{\texttt{reestructurarRaiz}:}

Este algoritmo también tiene eficiencia $O(\log n)$, ya que en el caso general $T(n)=1+T(\frac{n}{2})$, desarrollándose igual que \texttt{insertarEnPos}.

\textbf{\texttt{HeapSort}}

Por último, queda analizar este algoritmo, que consideramos iterativo habiendo analizado sus bloques recursivos.
Este algoritmo está compuesto por dos bucles \texttt{for}, ambos con cuerpos de eficiencia $O(\log n)$ y con inicialización, comprobación y actualización de eficiencia $O(1)$ que se repiten $n$ veces cada uno, por lo que tienen cada uno eficiencia $O(n\cdot\log n)$.
Aplicando la regla del máximo, determinamos que \textbf{la eficiencia de \texttt{HeapSort} es $\boldsymbol{O(n\cdot\log n)}$}.

\pagebreak

\section{Eficiencia práctica}\label{eficiencia-practica}

Para medir la eficiencia práctica se han tomado para todos muestras de $10000$ a $500000$ en incrementos de $10000$.

Para el algoritmo \texttt{MergeSort} se han obtenido (entre otros\footnote{Los valores obtenidos han sido para todos los tamaños de $10000$ a $500000$ en incrementos de $10000$, se omiten algunos para facilitar la lectura}) los siguientes resultados:

\begin{center}
\begin{tabular}{r r || r r}
\textbf{Tamaño} & \textbf{Tiempo} & \textbf{Tamaño} & \textbf{Tiempo} \\
\toprule
10000           & 711             & 250000          & 24060           \\
20000           & 1516            & 300000          & 27927           \\
50000           & 3971            & 310000          & 29103           \\
100000          & 9403            & 320000          & 30087           \\
110000          & 10816           & 350000          & 33250           \\
120000          & 10900           & 400000          & 38004           \\
150000          & 15143           & 410000          & 39056           \\
200000          & 21954           & 420000          & 40506           \\
210000          & 19686           & 450000          & 44966           \\
220000          & 21438           & 500000          & 48632           \\
\end{tabular}
\end{center}

Para el algoritmo \texttt{HeapSort} se han obtenido (entre otros) los siguientes resultados:

\begin{center}
\begin{tabular}{r r || r r}
\textbf{Tamaño} & \textbf{Tiempo} & \textbf{Tamaño} & \textbf{Tiempo} \\
\toprule
10000           & 791             & 250000          & 32946           \\
20000           & 1732            & 300000          & 39077           \\
50000           & 4881            & 310000          & 40499           \\
100000          & 11541           & 320000          & 41434           \\
110000          & 13203           & 350000          & 45298           \\
120000          & 13823           & 400000          & 53345           \\
150000          & 19085           & 410000          & 53373           \\
200000          & 25911           & 420000          & 59734           \\
210000          & 27742           & 450000          & 57466           \\
220000          & 29218           & 500000          & 66525           \\
\end{tabular}
\end{center}

Se puede observar la progresión del tiempo de ejecución de ambos algoritmos en las siguientes gráficas:

\begin{figure}[h]
\begin{center}
\begin{tikzpicture}
\begin{axis}[xlabel=Número de elementos,
             ylabel=Tiempo de ejecución ($\mu s$)
            ]
\addplot [scatter,
          only marks,
          mark size=1pt
         ] table {Practicas/Eficiencia/merge-sort.log};
\addplot [no markers,
          black
         ] table [y={create col/linear regression={y=y}}] {Practicas/Eficiencia/merge-sort.log};
\end{axis}
\end{tikzpicture}
\begin{tikzpicture}
\begin{axis}[xlabel=Número de elementos,
             ylabel=Tiempo de ejecución ($\mu s$)
            ]
\addplot [scatter,
          only marks,
          mark size=1pt
         ] table {Practicas/Eficiencia/heap-sort.log};
\addplot [no markers,
          black
         ] table [y={create col/linear regression={y=y}}] {Practicas/Eficiencia/heap-sort.log};
\end{axis}
\end{tikzpicture}
\end{center}
\caption{Tiempos de ejecución de \texttt{MergeSort} (izquierda) y \texttt{HeapSort} (derecha)}
\end{figure}

\section{Cálculo de la constante oculta}\label{eficiencia-oculta}

\subsection{\texttt{MergeSort}}\label{eficiencia-oculta-mergesort}

Calculamos la constante $K$ haciendo uso de \texttt{gnuplot}:

\begin{lstlisting}[language=tcl]
gnuplot> f(x)=a*x*log(x)
gnuplot> fit f(x) 'merge-sort.log' via a
iter      chisq       delta/lim  lambda   a
   0 6.9308675510e+14   0.00e+00  3.75e+06    1.000000e+00
   1 2.6651057555e+11  -2.60e+08  3.75e+05    2.698698e-02
   2 4.1261386413e+07  -6.46e+08  3.75e+04    7.530609e-03
   3 4.1250731902e+07  -2.58e+01  3.75e+03    7.526718e-03
   4 4.1250731902e+07  -1.44e-10  3.75e+02    7.526718e-03
iter      chisq       delta/lim  lambda   a

After 4 iterations the fit converged.
final sum of squares of residuals : 4.12507e+07
rel. change during last iteration : -1.44494e-15

degrees of freedom    (FIT_NDF)                        : 49
rms of residuals      (FIT_STDFIT) = sqrt(WSSR/ndf)    : 917.525
variance of residuals (reduced chisquare) = WSSR/ndf   : 841852

Final set of parameters            Asymptotic Standard Error
=======================            ==========================
a               = 0.00752672       +/- 3.459e-05    (0.4596%)
\end{lstlisting}

Tenemos que la constante $K$ vale, de media, $0.00752672$.
Calculando el tiempo teórico y comparándolo con el tiempo de ejecución tenemos la siguiente gráfica:

\begin{figure}[h]
\begin{center}
\begin{tikzpicture}
\begin{axis}[xlabel=Número de elementos,
             ylabel=Tiempo de ejecución ($\mu s$),
             legend pos=outer north east
            ]
\addplot [color=red,
          mark=*,
          mark size=1pt
         ] table [x index=0,y index=2,col sep=space] {Practicas/Eficiencia/merge-sort.log};
\addplot [color=blue,
          mark=*,
          mark size=1pt
         ] table [x index=0,y index=1,col sep=space] {Practicas/Eficiencia/merge-sort.log};
\legend{Tiempo teórico,Tiempo de ejecución}
\end{axis}
\end{tikzpicture}
\end{center}
\caption{Comparación entre los tiempos de ejecución práctico y teórico del algoritmo \texttt{MergeSort}}
\end{figure}

\pagebreak

\subsection{\texttt{HeapSort}}\label{eficiencia-oculta-heapsort}

Calculamos la constante $K$ haciendo uso de \texttt{gnuplot}:

\begin{lstlisting}
gnuplot> f(x)=a*x*log(x)
gnuplot> fit f(x) 'heap-sort.log' via a
iter      chisq       delta/lim  lambda   a
   0 6.8915727396e+14   0.00e+00  3.75e+06    1.000000e+00
   1 2.6508616504e+11  -2.60e+08  3.75e+05    2.974923e-02
   2 1.2764186428e+08  -2.08e+08  3.75e+04    1.034809e-02
   3 1.2763127018e+08  -8.30e+00  3.75e+03    1.034421e-02
   4 1.2763127018e+08  -1.75e-10  3.75e+02    1.034421e-02
iter      chisq       delta/lim  lambda   a

After 4 iterations the fit converged.
final sum of squares of residuals : 1.27631e+08
rel. change during last iteration : -1.75127e-15

degrees of freedom    (FIT_NDF)                        : 49
rms of residuals      (FIT_STDFIT) = sqrt(WSSR/ndf)    : 1613.91
variance of residuals (reduced chisquare) = WSSR/ndf   : 2.60472e+06

Final set of parameters            Asymptotic Standard Error
=======================            ==========================
a               = 0.0103442        +/- 6.084e-05    (0.5882%)
\end{lstlisting}

Tenemos que la constante $K$ vale, de media, $0.0103442$.
Calculando el tiempo teórico y comparándolo con el tiempo de ejecución tenemos la siguiente gráfica:

\begin{figure}[h]
\begin{center}
\begin{tikzpicture}
\begin{axis}[xlabel=Número de elementos,
             ylabel=Tiempo de ejecución ($\mu s$),
             legend pos=outer north east
            ]
\addplot [color=red,
          mark=*,
          mark size=1pt
         ] table [x index=0,y index=2,col sep=space] {Practicas/Eficiencia/heap-sort.log};
\addplot [color=blue,
          mark=*,
          mark size=1pt
         ] table [x index=0,y index=1,col sep=space] {Practicas/Eficiencia/heap-sort.log};
\addplot [no markers,
          black
         ] table [z={create col/linear regression={z=z}}] {Practicas/Eficiencia/heap-sort.log};
\legend{Tiempo teórico,Tiempo de ejecución}
\end{axis}
\end{tikzpicture}
\end{center}
\caption{Comparación entre los tiempos de ejecución práctico y teórico del algoritmo \texttt{HeapSort}}
\end{figure}
