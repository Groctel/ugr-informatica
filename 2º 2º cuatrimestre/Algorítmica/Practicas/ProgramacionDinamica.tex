\chapter{Programación dinámica}\label{programacion-dinamica}

\section{Enunciado}\label{pd-enunciado}

Un excéntrico nutricionista va a un restaurante.
En la carta aparecen todos los PLATOS disponibles con el número de calorı́as.
El nutricionista conoce el número mı́nimo de calorı́as que su cuerpo necesita en esa comida.
Su objetivo es encontrar el menú que cubra exactamente esa cantidad de calorı́as o, en su defecto, que las supere de forma mı́nima sin repetir PLATOS\@.
Diseñe un algoritmo de programación dinámica que determine qué PLATOS forman parte del menú óptimo y el número de calorı́as del menú óptimo.

\section{Análisis del problema}\label{pd-analisis}

\subsection{Elementos y definiciones}\label{pd-elementos-y-definiciones}

Para la realización de este problema contamos con los siguientes elementos:

\begin{itemize}
	\item Un número $K$ de caloría a alcanzar.
	\item Un total de $n$ PLATOS, con una cantidad de calorías $k_i$ para cada plato $i$.
\end{itemize}

Debemos encontrar un subconjunto $P$ de los PLATOS del menú cuyas calorías sean iguales o, en su defecto, mayores que $K$.
Si no hubiera más remedio que pasarnos de $K$, debemos minimizar la diferencia.
En el caso ideal, la relación entre las calorías de los PLATOS y $K$ sería la siguiente:

\[K=\sum_{n=0}^{i}k_i,\forall i\in P\]

Para tomar la decisión de qué PLATOS se van a elegir, utilizaremos una variable $p_i\in\{0,1\}$ que nos indicará la probabilidad que tiene cada plato $i$ de ser escogido.
Teniendo esto en cuenta, podemos reformular la relación anterior de la siguiente forma:

\[K=\sum_{n=0}^{i}k_i p_i,\forall i\in P\]

Dado que consideramos el caso en el que superamos las calorías, aunque lo minimizaremos, definimos para todo el problema la relación entre $K$ y $P$ de la siguiente forma:

\[K\leq\sum_{n=0}^{i}k_i p_i,\forall i\in P\]

\subsection{Resolubilidad mediante programación dinámica}\label{pd-resolubilidad}

Para que el problema pueda ser resuelto mediante técnicas de programación dinámica debe cumplir con los siguientes requisitos:

\subsubsection{Debe tratarse de un problema de optimización}

Como hemos explicado en~\ref{pd-elementos-y-definiciones}, queremos que la suma de las calorías de los PLATOS consumidos sea mayor o igual que $K$ de forma que la diferencia entre ambos sea mínima, por lo que claramente es un problema de minimización.

\subsubsection{Debe ser resoluble por etapas}

En cada etapa podemos decidir si añadir un plato a $P$, de forma que se modifica $p_i$ a $1$ si se añade el plato y $0$ en caso contrario.

\subsubsection{Debe ser expresable mediante una ecuación en recurrencias}

Para resolver este problema podemos considerar $M[i][j]$ como la mínima diferencia entre las calorías consumidas ($i$) y el total de calorías buscado ($j$), de forma que podemos expresar nuestro problema como $M[n][K]$.
De esta forma a la hora de elegir cada uno de los PLATOS podemos definir el cambio sobre $M[i][j]$ de la siguiente forma:

\[
	M[i][j]=
	\begin{cases}
		k_i+M[i-1][j-k_i] & \text{si } p_i=1 \\
		M[i-1][j]         & \text{si } p_i=0
	\end{cases}
\]

Como lo que queremos es minimizar la diferencia entre el total de calorías y $K$, podemos expresarlo de la siguiente forma:

\[M[i][j]=\min(k_i+M[i-1][j-k_i],M[i-1][j])\]

Como casos base de la ecuación en recurrencias tenemos los siguientes:

\[
	M[1][j]=
	\begin{cases}
		0   & \text{si } j>k_1 \\
		k_1 & \text{si } j\leq k_1
	\end{cases}
\]

\[M[i][0]=0\]

El primero es óptimo, ya que al tener un único plato a elegir sólo lo haremos si supera el número de calorías $K$.
El segundo es óptimo trivialmente porque si queremos acercarnos lo máximo posible a $0$ calorías, lo que debemos hacer es salir corriendo del restaurante y no mirar atrás.

\pagebreak

\section{Diseño del algoritmo}

Para simplificar las operaciones y no caer en problemas de aritmética de coma flotante, trabajaremos con valores $i,j\in\mathbb{N^*}$.
Para trabajar con este algoritmo realizaremos el siguiente diseño de memoria:

\begin{itemize}
	\item\textbf{Se guardarán los datos en una tabla $\boldsymbol{M[i][j]}$:} En esta tabla las filas representarán cada uno de los PLATOS y las columnas, las calorías.
	Cada elemento $M[i][j]$ almacenará el mínimo número de calorías consumibles mediante la elección de los $i-1$ PLATOS anteriores.
	\item\textbf{Se rellenarán las filas de izquierda a derecha:} Como el problema es expresable mediante una ecuación en recurrencia, para conocer cada elemento $i$ de la solución deberemos conocer los $i-1$ anteriores, por lo que debemos empezar a rellenar desde el elemento $1$ e incrementar gradualmente.
	\item\textbf{Asumiremos inválidos los valores $\boldsymbol{i=0}$ y $\boldsymbol{j=0}$.}
\end{itemize}

Definido este comportmiento, procedemos a crear el algoritmo:

\begin{lstlisting}[language=Python]
# Contantes
const CALORIAS
const INFINITO
const vector PLATOS

# Variables
consumidas
vector solucion
matriz matriz_m;

for fila in PLATOS
	for columna in CALORIAS
		matriz_m[fila] = 0

for i in 1..CALORIAS
	matriz_m[0][i] = INFINITO

for i in 1..PLATOS
	for j in 1..CALORIAS
		actual         = matriz_m[i-1][j]
		consumible     = PLATOS[i-1]
		matriz_m[i][j] = INFINITO

		if PLATOS[i-1] <= j
			consumible = matriz_m[i-1][j-PLATOS[i-1]];

			if consumible < INFINITO
				consumible += PLATOS[i-1];

		if consumible >= j
			matriz_m[i][j] = min(actual, consumible)

i = PLATOS
j = CALORIAS

while i>0 and j>0
	if matriz_m[i][j] != matriz_m[i-1][j]
		# Añade al final del vector solucion el valor i-1
		j = j - PLATOS[i-1];

	i = i-1

# Muestra la solución
\end{lstlisting}

Con esta implementación procedemos a calcular la eficiencia del algoritmo.
Este algoritmo está formado por cuatro bucles, tres \code{for} y un \code{while}.
Tres de los bucles realizan tantas iteraciones como el tamaño de \code{PLATOS} mientras que uno de ellos (el segundo bucle \code{for}) las realiza sobre el valor de \code{CALORIAS}.
De estos bucles, el tercero tiene un segundo bucle \code{for} dentro de él, que también itera sobre el valor de \code{CALORIAS}.
De esta forma, la eficiencia del tercer bucle \code{for} en el caso peor es $O(ij)$ para una lista de $i$ platos y $k$ calorías totales, ya que el resto del bucle lo componen operaciones elmentales.
Por la regla del máximo, ya que $ij>\max(i,j)$, concluimos que \textbf{la eficiencia de este algoritmo es $\boldsymbol{O(ij)}$}.

\section{Implementación del algoritmo}

Se ha implementado el algoritmo en C++ haciendo uso de la STL, de forma que los platos y la lista de solución se han representado en un \code{std::vector} y la matriz de trabajo como un \code{std::vector<std::vector>}.
Para todos los elementos se ha utilizado el tipo de dato \code{unsigned} para asegurar que todos los elementos pertenecen a $\mathbb{N}$.
De la misma forma se ha asegurado que en ningún momento se produzca un \textit{underflow} en los bucles.

Para compilar el proyecto es suficiente con ejecutar \code{g++} sobre el único fichero \code{main.cpp} que lo compone.

\section{Ejecución del algoritmo}

Para ejecutar el algoritmo simplemente debemos editar la constante \code{CALORIAS} y el \code{std::vector} de \code{PLATOS}, compilarlo y ejecutarlo.
Aquí se muestran algunas salidas del mismo:

\subsubsection{$\boldsymbol{100}$ calorías con el vector \code{\{75, 83, 19, 64, 25, 6\}}}

\begin{lstlisting}
Plato 5: 25 calorías
Plato 1: 75 calorías
Total consumido: 100 calorías
\end{lstlisting}

\subsubsection{$\boldsymbol{100}$ calorías con el vector \code{\{75, 83, 19, 64, 21, 6\}}}

\begin{lstlisting}
Plato 6: 6 calorías
Plato 3: 19 calorías
Plato 1: 75 calorías
Total consumido: 100 calorías
\end{lstlisting}

\subsubsection{$\boldsymbol{313}$ calorías con el vector \code{\{75, 83, 19, 64, 21, 6, 33, 51, 22, 31, 51, 76, 40, 12, 45, 21, 84\}}}

\begin{lstlisting}
Plato 8: 51 calorías
Plato 5: 21 calorías
Plato 4: 64 calorías
Plato 3: 19 calorías
Plato 2: 83 calorías
Plato 1: 75 calorías
Total consumido: 313 calorías
\end{lstlisting}
