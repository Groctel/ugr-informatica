\chapter{La eficiencia de los algoritmos}

\section{El concepto de algoritmo}

Un algoritmo\footnote{La palabra \textit{algoritmo} proviene del nombre del matemático persa del siglo IX Abd Allah Muhhamad ibn Musa \textbf{al-Khwarizmi}.} es una \textit{secuencia finita ordenada de pasos exentos de ambigüedad tal que, al llevarse a cabo con fidelidad, dará como resultado la tarea para la que se ha diseñado}.
De esta definición podemos extraer algunos términos clave:

\begin{itemize}
	\item\textbf{Secuencia:} Los algoritmos se componen de instrucciones que deben realizarse una detrás de otra.
	\item\textbf{Finita:} Los algoritmos deben acabar en algún momento para poder producir un resultado.
	\item\textbf{Ordenada:} Las instrucciones que componen un algoritmo deben respetar un orden para desempeñar correctamente su tarea asociada.
	\item\textbf{Exentos de ambigüedad:} Los algoritmos no deben dejar ningún elemento a la libre interpretación del lector, ya que ésta cambiará su resultado.
	\item\textbf{Fidelidad:} Los algoritmos deben ser deterministas, es decir, deben producir el mismo resultado con la misma entrada.
\end{itemize}

Los algoritmos no son programas informáticos, aunque éstos pueden incorporar algoritmos para su correcto funcionamiento.
Los algoritmos deben cumplir necesariamente cinco propiedades:

\begin{itemize}
	\item\textbf{Ser nociones abstractas:} Los algoritmos no dependen del lenguaje en el que se implementen, sino que deben poder implementarse en varios lenguajes.
	\item\textbf{Estar bien definidos:} Cada paso de los algoritmos debe estar claramente expresado y sin ambigüedades.
	\item\textbf{Ser coherentes:} Los algoritmos deben producir el mismo resultado siempre que se le introduzcan los mismos datos iniciales.
	\item\textbf{Ser finitos:} Los algoritmos deben terminar.
	\item\textbf{Ser efectivos:} Los algoritmos deben resolver su problema asociado.
\end{itemize}

Un ejemplo de un mal algoritmo sería una receta de croquetas:

\begin{enumerate}
	\item $Echar\ aceite\ en\ la\ sartén$.
	\item $Esperar\ a\ que\ el\ aceite\ esté\ caliente$.
	\item $Echar\ croquetas\ en\ la\ sartén$.
	\item $Mientras\ las\ croquetas\ no\ se\ hayan\ dorado,\ esperar$.
	\item $Sacar\ las\ croquetas\ a\ un\ plato$:
\end{enumerate}

Aunque no cabe duda de que nos saldrían unas croquetas buenísimas siguiendo esta receta, falla como algoritmo por varios motivos.
Primero, no está \textbf{bien definido}.
¿Cuánto aceite echamos en la sartén?
¿Una garrafa?
¿Una pincelada?
¿Cuántas croquetas echamos en la sartén?
¿Qué nivel de dorado queremos que tengan las croquetas?
¿A qué potencia debemos poner el fuego?
¿En qué orden sacamos las croquetas?
Tampoco es \textbf{coherente}, ya que no siempre proporciona el mismo resultado.
En función del orden en que saquemos las croquetas algunas saldrán más hechas que otras en distintas iteraciones.
Por último, tampoco es \textbf{finito}.
Como no hemos especificado que ha de encenderse el fuego, ¡estaremos esperando eternamente a que el aceite esté caliente!

Siguiendo con analogías culinarias, esta receta de masa de pizza sí sería un buen algoritmo\footnote{Las medidas imperiales basadas en cucharas tienen equivalencias en el sistema métrico. Es una receta que sale muy rica.}:

\begin{enumerate}
	\item $Introducir\ 200g\ de\ agua\ a\ 20\ ^{\circ}C\ en\ la\ panificadora$.
	\item $Introducir\ 350g\ de\ harina\ de\ fuerza\ en\ la\ panificadora$.
	\item $Introducir\ una\ cucharada\ sopera\ de\ aceite\ en\ la\ panificadora$.
	\item $Introducir\ una\ cucharadita\ de\ sal$.
	\item $Introducir\ media\ cucharadita\ de\ azúcar\ moreno$.
	\item $Introducir\ 7\ gramos\ de\ levadura\ en\ polvo$.
	\item $Programar\ la\ panificadora\ para\ trabajar\ en\ modo\ \text{masa}\ durante\ 45\ minutos$.
	\item $Iniciar\ la\ panificadora$.
	\item $Esperar\ 45\ minutos$.
	\item $Extraer\ la\ masa\ de\ la\ panificadora$.
\end{enumerate}

\section{La eficiencia de algoritmos: Problema, tamaño e instancia, principio de invarianza}

Dado un problema, podemos encontrar diferentes algoritmos que lo resuelven.
¿Cuál de todos es el mejor?
¿En qué situaciones podemos afirmar que uno es mejor que otro?
La eficiencia estudia la viabilidad de la implementación de un algoritmo para la resolución de un problema determinado.
Podemos medirla en función de los recursos que consume mediante dos factores:

\begin{itemize}
	\item\textbf{Tiempo:} Demora del algoritmo en la resolución del problema.
	\item\textbf{Espacio:} Recursos del sistema usados por el algoritmo.
\end{itemize}

A lo largo de esta asignatura nos centraremos en la dimensión temporal de la eficiencia.
Para estudiarla, debemos definir previamente cuatro conceptos:

\begin{itemize}
	\item\textbf{Problema:} Problema general a resolver expresado sin especificar unidades. Por ejemplo, ordenar ascendentemente un vector.
	\item\textbf{Instancia del problema:} Problema concreto derivado del problema general expresado con unidades. Por ejemplo, ordenar el vector \code{[3,5,1,2]}.
	\item\textbf{Caso:} Conjunto de instancias con dificultad similar o idéntica
	\begin{itemize}
		\item\textbf{Caso peor:} Instancia en la que el algoritmo ejecuta el máximo de operaciones posible. Por ejemplo, ordenar ascendentemente por inserción un vector previamente ordenado descendentemente.
		\item\textbf{Caso promedio:} Instancia en la que tarda un tiempo medio. Se da generalmente cuando los datos iniciales son aleatorios.
		\item\textbf{Caso mejor:} Instancia en la que el algoritmo ejecuta el mínimo de posible. Por ejemplo, ordenar ascendentemente un vector previamente ordenado ascendentemente.
	\end{itemize}
	\item\textbf{Tamaño del caso:} tamaño de la instancia a resolver. Por ejemplo, el vector \code{[3,5,1,2]} es un caso de tamaño $n=4$.
\end{itemize}

Usaremos la notación $O(n)$ para indicar la eficiencia de un algoritmo en el caso peor y $\Omega(n)$ para su eficiencia en el caso mejor.
Ambos casos peor y mejor lo son únicamente para instancias del mismo tamaño.

\pagebreak

Tomemos como ejemplo el siguiente algoritmo para ordenar un vector:

\begin{lstlisting}[language=C++]
void insertion (double* v, int a, int b) {
	for (int i=a+1; i<=b; i++) {
		for (int j=i; j>a && v[j]<v[j-1]; j--) {
			double aux = v[j];

			v[j]   = v[j-1];
			v[j-1] = aux;
		}
	}
}
\end{lstlisting}

Este algoritmo ordena ascendentemente un vector \code{v} desde la posición \code{a} hasta la posición \code{b}, ambas inclusive.
Para ello, va comparando todos los valores desde la posición \code{a+1} hasta la posición \code{b} con el valor inmediatamente anterior a ellos (por ejemplo, cuando \code{i=1}, comparamos \code{a+1} con \code{a}) y lo intercambiamos con éste tantas veces como sea necesario hasta que el valor que estamos ordenando sea mayor que el anterior o esté en la primera posición a ordenar.
Éste es un caso de tamaño $n=a-b+1$ cuyo mejor caso es cuando \code{v} se introduce ordenado ascendentemente y cuyo peor caso es cuando \code{v} se introduce ordenado descendentemente.

A la hora de medir la eficiencia de un algoritmo podemos usar tres métodos:

\begin{itemize}
	\item\textbf{Método empírico:} Implementar el algoritmo y medir el tiempo de ejecución.
	\item\textbf{Método teórico:} Calcular una función que indique la evolución del algoritmo con respecto al tamaño del caso.
	\item\textbf{Método híbrido:} Una mezcla de ambos métodos.
\end{itemize}

Al igual que un algoritmo debe ser independiente de un lenguaje, su eficiencia también lo es.
Llamamos a esta propiedad \textbf{Prinicpio de Inviarianza}:

\begin{displayquote}
		  \textit{Dadas dos implementaciones $I_1$ e $I_2$ de un mismo algoritmo, el tiempo de ejecución para una misma instancia de tamaño $n$, $T_{I_1}(n)$ y $T_{I_2}(n)$ no diferirá en más de una constante multiplicativa. Es decir, existe una constante positiva $K$ que verifica $T_{I_1}(n)\leq K\cdot T_{I_2}(n)$.}
\end{displayquote}

Teóricamente, esta constante $K$ es siempre igual para distintas ejecuciones del algoritmo, aunque en la práctica puede variar ligeramente debido a la carga del procesador a la hora de ejecutar el programa.
Lo que siembre será cierto es que, aunque dos lenguajes tarden tiempos diferentes en realizar la misma tarea, sus gráficas espacio-temporales tendrán la misma forma.
Por ejemplo, dos programas que ejecuten un algoritmo de eficiencia $O(n^2)$ en el peor caso mostrarán gráficas de curvatura $k\cdot n^2$.

\section{La notación asintótica, órdenes peor, mejor y exacto}

\section{Análisis de algoritmos}

\section{Resolución de recurrencias}
