\chapter{Herramientas de programación paralela III\@: Interacción con el entorno en OpenMP y evaluación de prestaciones}

\section{Funciones y variables de entorno y control}

Las variables de entorno especifican el comportamiento de los programas paralelizados con OpenMP a partir de la \textit{shell} en la que éstos se ejecutan.
Para declarar una variable de entorno usamos \code{export} o \code{setenv}:

\begin{lstlisting}[language=sh]
export variable=valor
setenv variable=valor
\end{lstlisting}

Podemos asignar las siguiente variables de entorno:

\begin{itemize}
	\item\code{OMP\_DYNAMIC=(TRUE)|(FALSE)}\textbf{:} Determina el ajuste dinámico del número de hebras. Su valor inicial depende de la implementación de OpenMP\@.
	\item\code{OMP\_NUM\_THREADS=[0-9]+}\textbf{:} Determina el número de hebras que se ejecutarán en las secciones paralelas. Su valor inicial depende de la implementación de OpenMP\@.
	\item\code{OMP\_THREAD\_LIMIT=[0-9]+}\textbf{:} Determina el máximo número de hebras que se ejecutarán a la ves. Su valor inicial depende de la implementación de OpenMP\@.
	\item\code{OMP\_NESTED=(TRUE)|(FALSE)}\textbf{:} Determina si el programa permite paralelismo anidado. Se inicializa a \code{false} por defecto.
	\item\code{OMP\_SCHEDULE=\{kind\}(,[0-9]+)?}\textbf{:} Determina el sistema de planificación de los bucles en tiempo de ejecución mediante el tipo (\code{kind}), que se detallará en~\ref{hfdjgksfd} y, opcionalmente, el \textit{chunk}. Su valor inicial depende de la implementación de OpenMP\@.
\end{itemize}

Todas estas variables de entorno se corresponden con una variable de control que puede consultarse y modificarse:

\begin{center}
\begin{table}[!h]
\begin{tabular}{l l l l}
	\textbf{Variables de entorno} & \textbf{Variables de contro} & \textbf{Consultores}             & \textbf{Modificadores}           \\
	\toprule
	\code{OMP\_DYNAMIC}           & \code{dyn-var}               & \code{omp\_get\_dynamic()}       & \code{omp\_set\_dynamic()}       \\
	\code{OMP\_NUM\_THREADS}      & \code{nthreads-var}          & \code{omp\_get\_num\_threads()}  & \code{omp\_set\_num\_threads()}  \\
	\code{OMP\_THREAD\_LIMIT}     & \code{thread-limit-var}      & \code{omp\_get\_thread\_limit()} & \code{thread\_limit [cláusulas]} \\
	\code{OMP\_NESTED}            & \code{nest-var}              & \code{omp\_get\_nested()}        & \code{omp\_set\_nested()}        \\
	\code{OMP\_SCHEDULE}          & \code{run-sched-var}         & \code{omp\_get\_schedule()}      & \code{omp\_set\_schedule()}      \\
\end{tabular}
\caption{Correspondencia entre variales de entorno, variables de control interna y sus funciones}
\end{table}
\end{center}

Todas estas variables de entorno y control afectan a los bloques \code{parallel} a exceipción de \code{OMP\_SCHEDULE}, que afecta únicamente a los bucles \code{for}.
Junto con esta variable, podemos definir para los bucles \code{def-sched-var}, que determina la planificación por defecto de los bucles.
Esta variable no cuenta con una variable de entorno correspondiente ni con funciones de consulta o modificación.

Cabe destacar las siguientes funciones del entorno de ejecución:

\begin{itemize}
	\item\code{omp\_get\_thread\_num()}\textbf{:} Devuelve a la hebra invocante su identificador de hebra.
	\item\code{omp\_get\_num\_threads()}\textbf{:} Devuelve el número total de hebras ejecutando el bloque paralelo en el que se encuentra.
	\item\code{omp\_get\_num\_procs()}\textbf{:} Devuelve el número de procesadores disponibles para el programa en tiempo de ejecución.
	\item\code{omp\_in\_parallel()}\textbf{:} Devuelve \code{(true)|(false)} en función de si la llamada se hace desde un bloque \code{parallel}.
\end{itemize}

\section{Cláusulas de interacción con el entorno}

Junto a las cláusulas detalladas en la práctica~\ref{clausulas-openmp}, podemos definir las cláusulas \code{if()}, \code{num\_threads()} y \code{schedule()}, que cumplen con las limitaciones de aplicación mostradas en el cuadro~\ref{clausulas-omp-definiciones-aceptacion}.

\subsection{Definición del número de hebras a utilizar}

En orden descendente de prioridad, las hebras se definen por los siguientes valores:

\begin{itemize}
	\item El valor devuelto por la cláusula \code{if}.
	\item El valor fijado por la cláusula \code{num\_threads}.
	\item El valor fijado por la función \code{omp\_set\_num\_threads()}.
	\item El valor de la variable de entorno \code{OMP\_NUM\_THREADS}.
	\item El valor fijado por defecto por la implementación de OpenMP\@.
\end{itemize}

\subsection{Cláusula \code{if}}

\begin{lstlisting}[language=C]
#pragma omp [directiva] if (expr)
\end{lstlisting}

Determina en tiempo de ejecución si el bloque posterior debe ejecutarse de forma paralela en función del valor de verdad de la expresión \code{expr}.

\begin{lstlisting}[language=C]
#include <stdio.h>
#include <stdlib.h>
#include <omp.h>

int main (int argc, char ** argv) {
	int ID,
		 valor;

	valor = atoi(argv[1]);

	#pragma omp parallel private(ID) if (valor % 2 == 0)
	{
		ID = omp_get_thread_num();
		printf("Impresión ejecutada por la hebra %d\n", ID);
	}

	return 0;
}
\end{lstlisting}

\begin{lstlisting}[language=sh]
./prueba 1
# Impresión ejecutada por la hebra 0
./prueba 2
# Impresión ejecutada por la hebra 0
# Impresión ejecutada por la hebra 2
# Impresión ejecutada por la hebra 3
# Impresión ejecutada por la hebra 1
\end{lstlisting}

\subsection{Cláusula \code{schedule}}

\begin{lstlisting}[language=C]
#pragma omp [directiva] schedule (kind[,chunk])
\end{lstlisting}

\section{Clasificación de las funciones de OpenMP}

\section{Funciones para obtener el tiempo de ejecución}
