\section{Evaluación de prestaciones de una arquitectura}\label{evaluacion-prestaciones-arq}

\subsection{Medidas usuales para evaluar prestaciones}\label{medidas-usuales-evaluar-prestaciones}

\subsubsection{Tiempo de respuesta}

El tiempo de respuesta de un programa está compuesto por la suma de tres tiempos:

\begin{itemize}
	\item\textbf{Tiempo de usuario:} Tiempo en ejecución en el espacio del usuario.
	\item\textbf{Tiempo de sistema:} Tiempo de ejecución en el epsacio del kernel.
	\item\textbf{Tiempo en espera:} Tiempo que el programa está esperando a operaciones de E/S o a que finalice otro proceso.
\end{itemize}

\[Tiempo\ de\ respuesta=tCPU_{user}+tCPU_{sys}+t_{espera}\]

Tenemos varias formas de obtener el tiempo de ejecución de un programa:

\begin{center}
\begin{tabular}{l l c r}
	\textbf{Función}           & \textbf{Fuente}        & \textbf{Tipo}         & \textbf{Precisión (ms)} \\
	\toprule
	\texttt{time}              & \texttt{/usr/bin/time} & elapsed, user, system & 10000                   \\
	\texttt{clock()}           & \texttt{time.h}        & CPU                   & 10000                   \\
	\texttt{gettimeofday()}    & \texttt{sys/time.h}    & elapsed               & 1                       \\
	\texttt{clock\_gettime()}  & \texttt{time.h}        & elapsed               & 0.001                   \\
	\texttt{omg\_get\_wtime()} & \texttt{omp.h}         & elapsed               & 0.001                   \\
	\texttt{SYSTEM\_CLOCK()}   & Fortran                & elapsed               & 1                       \\
\end{tabular}
\end{center}

\subsubsection{Tiempo de CPU}

El \textbf{tiempo de CPU} ($T_{CPU}$) de un programa se calcula como el producto del número de ciclos del mismo y el tiempo que tarda el procesador en ejecutar cada ciclo.
Podemos expresarlo tambien como el cociente entre el número de ciclos y la frecuencia de reloj del procesador, que es inversamente proporcinal al tiempo de ciclo.

\[T_{CPU}=Ciclos\cdot T_{ciclo}=\frac{Ciclos\ del\ programa}{Frecuencia\ del\ reloj}\]

También podemos calcular el tiempo de CPU de un programa como el producto entre el número de instrucciones del mismo, el número de \textbf{ciclos por instrucción} ($CPI$) y el tiempo por ciclo.
Trivialmente, se tiene que el $CPI$ de un programa es el cociente entre los ciclos del mismo y el número de instrucciones por las que está compuesto.

\[T_{CPU}=NI\cdot CPI\cdot T_{ciclo}\]
\[\text{Ciclos por instrucción}\ (CPI)=\frac{Ciclos\ del\ programa}{N\acute{u}mero\ de\ instrucciones\ (NI)}\]

Dado un programa cualquiera, éste está compuesto por un número $I_i$ de instrucciones del tipo $i\forall i\in\mathbb{N}$.
Si cada instrucción del tipo $i$ consume $CPI_i$ ciclos y hay $k$ tipos de instrucciones distintos, podemos expresar el número de ciclos del programa y, por consiguiente, el CPI de la siguiente manera:

\[Ciclos\ del\ programa=\sum_{i=0}^{k}CPU_i\cdot I_k\]
\[CPI=\frac{\sum_{i=0}^{k}CPI_i\cdot I_i}{N\acute{u}mero\ de\ instrucciones}\]

Otra forma de obtener el $CPI$ es mediante el cociente entre los \textbf{ciclos por emisión} ($CPE$) e \textbf{instrucciones por emisión}, que son el número mínimo de ciclos transcurridos entre los instantes en los que el procesador puede emitir instrucciones y el número de instrucciones emitibles cada vez que se produce una emisión, respectivamente.

\[CPI=\frac{CPE}{IPE}\]

Distinguimos cinco segmentos en el cauce de un procesador\footnote{Estructura de computadores, tema 4.}.
Un procesador no segmentado ejecutará dos instrucciones de forma secuencial ocupando cada una los cinco segmentos cada vez, de forma que cada instrucción deberá emitirse una a una y tardará cinco ciclos en completarse.
Un procesaador segmentado ejecutará las instrucciones superpuestas entre sí de forma que mientras una instrucción $I_i$ esté ejecutando en el segmento $n_i$, la instrucción $I_{i-1}$ estará ejecutando en el segmento $n_{i-1}$.
Por último, un procesador superescarlar podrá ejecutar varias instrucciones en un solo segmento ($I_i$ e $I_{i-1}$ ejecutándose paralelamente en $n_i$).

Para un modelo no segmentado, uno segmentado sin riesgos y uno superescalar sin riesgos y capaz de ejecutar dos instrucciones paralelamente obtendríamos los siguientes valores:

\begin{center}
\begin{tabular}{l r r r}
	                                  & \textbf{CPE} & \textbf{IPE} & \textbf{CPI} \\
	\toprule
	\textbf{Procesador no segmentado} & 5            & 1            & 5 \\
	\textbf{Procesador segmentado}    & 1            & 1            & 1 \\
	\textbf{Procesador superescalar}  & 1            & 2            & 0.5 \\
\end{tabular}
\end{center}

El número de instrucciones es calculable como el cociente entre el \textbf{número de operaciones} ($N_{op}$) que realiza el programa y el \textbf{número de operaciones codificables en una instrucción} ($OP_{instr}$).

\[NI=\frac{N_{op}}{OP_{instr}}\]

A la hora de mejorar el tiempo de CPU, las mejoras en la tecnología y en la estructura y organización del computador afectan al $CPI$ y al tiempo por ciclo, mientras que las mejoras en el repertorio de instrucciones y del compilador afectan al $CPI$ y al número de instrucciones.

\subsubsection{Productividad: MIPS y MFLOPS}

\subsubsection{MIPS}

El número de instrucciones que un procesador puede ejecutar en un segundo se mide en $\boldsymbol{MIPS}$\footnote{\textit{Meaningless Indication of Processor Speed}.} (\textit{Millions of Instructions Per Second}).
El valor de los $MIPS$ depende del repertorio de instrucciones, por lo que es difícil comparar máquinas con repertorios distintos, y puede variar en función del programa ejecutado, por lo que no sirve para caracterizar la máquina que se evalúa.
Un mayor valor de $MIPS$ corresponde a peores prestaciones de la máquina.

\[MIPS=\frac{NI}{T_{CPU}\cdot10^6}=\frac{Frecuencia\ del\ reloj}{CPI\cdot10^6}\]

\subsubsection{MFLOPS}

Por otro lado tenemos los $\boldsymbol{MFLOPS}$ (\textit{Millions of FLoating Operations Per Second}).
Ésta no es una medida adecuada para todos los programas, ya que algunos pueden no realizar operaciones en coma flotante, ni es directamente desplazable a todas las máquinas, ya que el conjunto de operaciones en coma flotante no es constante en máquinas diferentes y la potencia de las operaciones no es igual en todas ellas.
Se hace necesaria una normalización de las instrucciones en coma flotante para poder calcular correctamente los $MFLOPS$.

\[MFLOPS=\frac{Operaciones\ en\ coma\ flotante}{T_{CPU}\cdot10^6}\]

\subsection{Conjunto de programas de prueba (\textit{Benchmark})}\label{benchmark}

Los \textbf{\textit{benchmarks}} son programas que analizan las prestaciones de los procesadores para comparar diferentes sistemas o realizaciones de un mismo sistema con un método fiable y reproducible.
Distinguimos diferentes tipos de \textit{benchmarks}:

\begin{itemize}
	\item\textbf{De bajo nivel (\textit{microbenchmark}):} Evaluación de operaciones con números enteros o de coma flotante.
	\item\textbf{Kernels:} Resolución de sistemas de ecuaciones, factorización y multiplicación de matrices\ldots
	\item\textbf{Sintéticos:} Programas como Dhrystone o Whetstone utilizados para poner a prueba las capacidades de cálculo de las arquitecturas.
	\item\textbf{Programas reales:} Puede usarse como \textit{benchmark} el rendimiento de \texttt{gcc}, \texttt{zip} u otros programas que realicen trabajos pesados con una gran cantidad de datos.
	\item\textbf{Aplicaciones diseñadas:} Programas de predicción de tiempo o simulación de terremotos, entre otros, pueden utilizarse para evaluar las prestaciones del computador debido a la inmensa cantidad de datos que manejan.
\end{itemize}

\subsection{Ganancia en prestaciones}\label{ganancia-prestaciones}

Al incrementar las prestaciones de un recurso de un procesador haciendo que su velocidad sea $n$ veces mayor (usando $n$ procesadores en lugar de uno, realizando la ALU las operaciones en un tiempo $n$ veces menor\ldots), se tiene que el incremento de velocidad alcanzable en la nueva situación con respecto a la previa (la máquina base), se expresa como el cociente entre la velocidad de la máquina mejorada $V_n$ y la velocidad de la máquina base $V_0$ o como el cociente entre el tiempo de ejecución en la máquina mejorada $T_n$ y el tiempo de ejecución en la máquina base $T_0$.
Llamamos a este incremento la \textbf{ganancia de velocidad} o \textit{speed-up} $S_n$.

\[S_n=\frac{V_n}{V_0}=\frac{T_0}{T_n}\]

\subsubsection{Ley de Amdahl}

La ley de Amdahl afirma que la mejora de velocidad $S$ que se puede obtener cuando se mejora un recurso de una máquina en un factor $n$ está limitada por la fracción del tiempo de ejecución en la máquina sin la mejora durante el tiempo que dicha mejora es inaplicable:

\[S\leq\frac{n}{1+f\cdot(n-1)}\]

Por ejemplo, si un programa pasa el $25\%$ de su tiempo de ejecución realizando instrucciones de coma flotante y se mejora la máquina de forma que estas instrucciones se realicen en la mitad de tiempo, tenemos que $n=2$, $f=0.75$.

\[S\leq\frac{2}{1+0.75\cdot(1-1)}=1.14\]
