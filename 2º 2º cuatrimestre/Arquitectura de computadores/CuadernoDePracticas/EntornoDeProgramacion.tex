\chapter{Entorno de programación}\label{entorno-de-programacion}

\section{}\label{ej1-1}

\subsubsection{Enunciado}

Ejecutar \texttt{lscpu} en el PC y en un nodo de cómputo de \textit{atcgrid}.

\begin{itemize}
	\item Mostrar el resultado de estas ejecuciones.
	\item ¿Cuántos núcleos físicos y cuántos núcleos lógicos tienen los nodos de cómputo de \textit{atcgrid} y el PC\@? Razonar las respuestas.
\end{itemize}

\subsubsection{Solución}

Ejecución de \texttt{lscpu} en el PC:

\begin{lstlisting}[language=sh]
lscpu
# Architecture:                    x86_64
# CPU op-mode(s):                  32-bit, 64-bit
# Byte Order:                      Little Endian
# Address sizes:                   39 bits physical, 48 bits virtual
# CPU(s):                          12
# On-line CPU(s) list:             0-11
# Thread(s) per core:              2
# Core(s) per socket:              6
# Socket(s):                       1
# NUMA node(s):                    1
# Vendor ID:                       GenuineIntel
# CPU family:                      6
# Model:                           158
# Model name:                      Intel(R) Core(TM) i7-8750H CPU @ 2.20GHz
# Stepping:                        10
# CPU MHz:                         900.075
# CPU max MHz:                     4100.0000
# CPU min MHz:                     800.0000
# BogoMIPS:                        4401.32
# Virtualization:                  VT-x
# L1d cache:                       192 KiB
# L1i cache:                       192 KiB
# L2 cache:                        1.5 MiB
# L3 cache:                        9 MiB
# NUMA node0 CPU(s):               0-11
# ...
\end{lstlisting}

\pagebreak

Ejecución de \texttt{lscpu} en el \textit{atcgrid}:

\begin{lstlisting}[language=sh]
lscpu
# Architecture:          x86_64
# CPU op-mode(s):        32-bit, 64-bit
# Byte Order:            Little Endian
# CPU(s):                24
# On-line CPU(s) list:   0-23
# Thread(s) per core:    2
# Core(s) per socket:    6
# Socket(s):             2
# NUMA node(s):          2
# Vendor ID:             GenuineIntel
# CPU family:            6
# Model:                 44
# Model name:            Intel(R) Xeon(R) CPU           E5645  @ 2.40GHz
# Stepping:              2
# CPU MHz:               1600.000
# CPU max MHz:           2401,0000
# CPU min MHz:           1600,0000
# BogoMIPS:              4799.93
# Virtualization:        VT-x
# L1d cache:             32K
# L1i cache:             32K
# L2 cache:              256K
# L3 cache:              12288K
# NUMA node0 CPU(s):     0-5,12-17
# NUMA node1 CPU(s):     6-11,18-23
\end{lstlisting}

El PC tiene $12$ núcleos lógicos y el \textit{atcgrid} tiene $24$ núcleos lógicos:

\[\begin{split}
           & \text{nº núcleos lógicos}=\text{sockets}\cdot\text{cores per socket}\cdot\text{threads per core}\\
PC\ \      & 1\cdot6\cdot2=12 \\
\textit{atcgrid}\ \ & 2\cdot6\cdot2=24
\end{split}\]

\section{}\label{ej1-2}

\subsubsection{Enunciado}

Compilar y ejecutar en el PC el código \texttt{HelloOMP.c} del seminario.

\begin{itemize}
	\item Mostar la compilación y ejecución en el PC\@.
	\item Justificar el número de ``Hello world'' que se imprimen en pantalla teniendo en cuenta la salida que devuelve \texttt{lscpu}.
\end{itemize}

\pagebreak

\subsubsection{Solución}

\begin{lstlisting}[language=sh]
gcc -O2 -fopenmp HelloOMP.c -o HelloOMP
./HelloOMP
# (9:Hello, World!!!)
# (7:Hello, World!!!)
# (10:Hello, World!!!)
# (2:Hello, World!!!)
# (4:Hello, World!!!)
# (5:Hello, World!!!)
# (3:Hello, World!!!)
# (8:Hello, World!!!)
# (1:Hello, World!!!)
# (0:Hello, World!!!)
# (11:Hello, World!!!)
# (6:Hello, World!!!)
\end{lstlisting}

Se imprimen $12$ ``Hello World'', uno por cada hebra (o núcleo lógico) del sistema identificado por un número en el rango $[0-11]$ mediante la llamada \texttt{omp\_get\_thread\_num()}.

\section{}\label{ej1-3}

\subsubsection{Enunciado}

Copiar el ejecutable de \texttt{HelloOMP.c} que ha generado en~\ref{ej1-2} y que se encuentra en su directorio del PC al directorio de su \texttt{\$HOME} en el \textit{front-end} de \textit{atcgrid}.
Ejecutar este código en un nodo de cómputo de \textit{atcgrid} a través de cola \texttt{ac} del gestor de colas (no use ningún script) utilizando directamente en la línea de comandos:

\begin{lstlisting}[language=sh]
srun -p ac -n1 --cpus-per-task=12 --hint=nomultithread HelloOMP
srun -p ac -n1 --cpus-per-task=24 HelloOMP
srun -p ac -n1 HelloOMP
\end{lstlisting}

Mostrar el envío a la cola de la ejecución y el resultado de esta ejecución tal y como la devuelve el gestor de colas.

¿Qué orden \texttt{srun} usaría para que \texttt{HelloOMP} utilice los $12$ núcleos físicos de un nodo de cómputo de \textit{atcgrid} (se debe imprimir un único mensaje desde cada uno de ellos, $12$ en total)?

\subsubsection{Solución}

\begin{lstlisting}[language=sh]
srun -p ac -n1 --cpus-per-task=12 --hint=nomultithread HelloOMP
# (6:Hello, World!!!)
# (0:Hello, World!!!)
# (3:Hello, World!!!)
# (8:Hello, World!!!)
# (5:Hello, World!!!)
# (11:Hello, World!!!)
# (1:Hello, World!!!)
# (9:Hello, World!!!)
# (2:Hello, World!!!)
# (10:Hello, World!!!)
# (4:Hello, World!!!)
# (7:Hello, World!!!)
\end{lstlisting}

\pagebreak

\begin{lstlisting}[language=sh]
srun -p ac -n1 --cpus-per-task=24 HelloOMP
# (0:Hello, World!!!)
# (21:Hello, World!!!)
# (17:Hello, World!!!)
# (1:Hello, World!!!)
# (9:Hello, World!!!)
# (18:Hello, World!!!)
# (13:Hello, World!!!)
# (15:Hello, World!!!)
# (22:Hello, World!!!)
# (16:Hello, World!!!)
# (5:Hello, World!!!)
# (8:Hello, World!!!)
# (23:Hello, World!!!)
# (19:Hello, World!!!)
# (14:Hello, World!!!)
# (2:Hello, World!!!)
# (12:Hello, World!!!)
# (4:Hello, World!!!)
# (10:Hello, World!!!)
# (20:Hello, World!!!)
# (11:Hello, World!!!)
# (7:Hello, World!!!)
# (6:Hello, World!!!)
# (3:Hello, World!!!)
\end{lstlisting}

\begin{lstlisting}[language=sh]
srun -p ac -n1 HelloOMP
# (1:Hello, World!!!)
# (0:Hello, World!!!)
\end{lstlisting}

Para que \texttt{HelloOMP} utilize los $12$ núcleos físicos de un nodo de cómputo de \textit{atcgrid} se debe usar la siguiente orden:

\begin{lstlisting}[language=sh]
srun -p ac -n1 --cpus-per-task=12 --hint=nomultithread HelloOMP
\end{lstlisting}

\section{}\label{ej1-4}

\subsubsection{Enunciado}

Modificar en su PC \texttt{HelloOMP.c} para que se imprima ``world'' en un \texttt{printf} distinto al usado para ``Hello''.
En ambos \texttt{printf} se debe imprimir el identificador de la hebra que escribe en pantalla.
Nombrar al código resultante \texttt{HelloOMP2.c}.
Compilar este nuevo código en el PC y ejecutarlo.
Copiar el fichero ejecutable resultante al \textit{front-end} de \textit{atcgrid}.
Ejecutar el código en un nodo de cómputo de \textit{atcgrid} usando el script \texttt{script\_helloomp.sh} del seminario.

Utilizar la siguiente orden\footnote{%
	Utilizar siempre con \texttt{sbatch} las opciones \texttt{-n1}, \texttt{--cpus-per-task}, \texttt{--exclusive} y, para usar núcleos físicos y no lógicos, no olvidar incluir \texttt{--hint=nomultithread}.
	Utilizar siempre con \texttt{srun}.
	Si lo usa fuera de un script, usar las mismas opcioens que con \texttt{sbatch} a excepción de \texttt{--exclusive}.
	Recordar que los \texttt{srun} dentro de un script heredan las opciones utilizadas en el \texttt{sbatch} que se usa para enviar el script a la cola \texttt{slurm}.
	Se recomienda usar \texttt{sbatch} en lugar de \texttt{srun} para enviar trabajos a ejecutar a través \texttt{slurm} porque éste último deja bloqueada la ventana hasta que termina la ejecución, mientras que usando \texttt{sbatch} la ejecución se realiza en segundo plano.
}:

\begin{lstlisting}[language=sh]
sbatch -p ac -n1 --cpus-per-task=12 --hint=nomultithread script_helloomp.sh
\end{lstlisting}

Mostrar el nuevo código, la compilación, el envío a la cola de la ejecución y el resultado de esta ejecución tal y como la devuelve el gestor de colas.

¿Qué nodo de cómputo de \textit{atcgrid} ha ejecutado el script?
Explicar cómo ha obtenido esta información.

\subsubsection{Solución}

\texttt{HelloOMP2.c}:
\lstinputlisting[language=C]{CuadernoDePracticas/Ej0.4/HelloOMP2.c}

Ejecución en el PC\@:
\begin{lstlisting}[language=sh]
gcc -O2 -fopenmp HelloOMP2.c -o HelloOMP2
./HelloOMP2
# (2:Hello)(0:Hello)(9:Hello)(6:Hello)(11:Hello)(5:Hello)(3:Hello)(4:Hello)(10:Hello)(1:Hello)(7:Hello)(8:Hello)(0:World)
\end{lstlisting}

Ejecución en \textit{atcgrid}:

\begin{lstlisting}[language=sh]
sbatch -p ac -n1 --cpus-per-task=12 --hint=nomultithread script_helloomp.sh
# Submitted batch job 54390
cat slurm-54392.out
# Id. usuario del trabajo: d1estudiante24
# Id. del trabajo: 54392
# Nombre del trabajo especificado por usuario: script_helloomp.sh
# Directorio de trabajo (en el que se ejecuta el script): /home/d1estudiante24/bp0/ejer3
# Cola: ac
# Nodo que ejecuta este trabajo:atcgrid.ugr.es
# Nº de nodos asignados al trabajo: 1
# Nodos asignados al trabajo: atcgrid1
# CPUs por nodo: 24
#
# 1. Ejecución helloOMP una vez sin cambiar nº de threads (valor por defecto):
# (8:Hello)(3:Hello)(5:Hello)(7:Hello)(11:Hello)(6:Hello)(1:Hello)(9:Hello)(4:Hello)(0:Hello)(2:Hello)(10:Hello)(0:World)
# 2. Ejecución helloOMP varias veces con distinto nº de threads:
#
#   - Para 12 threads:(1:Hello)(8:Hello)(2:Hello)(11:Hello)(10:Hello)(9:Hello)(0:Hello)(7:Hello)(5:Hello)(4:Hello)(6:Hello)(3:Hello)(0:World)
#   - Para 6 threads:(3:Hello)(0:Hello)(5:Hello)(4:Hello)(2:Hello)(1:Hello)(0:World)
#   - Para 3 threads:(1:Hello)(2:Hello)(0:Hello)(0:World)
#   - Para 1 threads:(0:Hello)(0:World)[AtanasioRubioGil d1estudiante24@atcgrid:~/bp0/ejer3] 2020-06-11 jueves
\end{lstlisting}

El script se ha ejecutado por el nodo \texttt{atcgrid1}, como indica la línea \texttt{Nodos asignados al trabajo}.

\pagebreak

\section{}\label{ej1-5}

\subsubsection{Enunciado}

Generar en el PC el ejecutable del código fuente C del \textit{Listado 1} para vectores locales (para ello, antes de compilar, debe descomentar la definición de \texttt{VECTOR\_LOCAL} y comentar las definiciones de \texttt{VECTOR\_GLOBAL} y \texttt{VECTOR\_DYNAMIC}).
El comentario inicial del código muestra la orden para compilar (siempre hay que usar \texttt{-O2} al compilar como se indica en las normas de prácticas).
Mostrar la compilación y la ejecución correcta del código en el PC\@.

\subsubsection{Solución}

\begin{lstlisting}[language=sh]
gcc -O2 SumaVectores.c -o SumaVectores -lrt
./SumaVectores 9
# Tamaño Vectores: 9 (4B)
# Tiempo: 0.000000268 --- Tamaño Vectores: 9
#  - V1[0]+V2[0] = V3[0] (0.900000+0.900000 = 1.800000)
#  - V1[1]+V2[1] = V3[1] (1.000000+0.800000 = 1.800000)
#  - V1[2]+V2[2] = V3[2] (1.100000+0.700000 = 1.800000)
#  - V1[3]+V2[3] = V3[3] (1.200000+0.600000 = 1.800000)
#  - V1[4]+V2[4] = V3[4] (1.300000+0.500000 = 1.800000)
#  - V1[5]+V2[5] = V3[5] (1.400000+0.400000 = 1.800000)
#  - V1[6]+V2[6] = V3[6] (1.500000+0.300000 = 1.800000)
#  - V1[7]+V2[7] = V3[7] (1.600000+0.200000 = 1.800000)
#  - V1[8]+V2[8] = V3[8] (1.700000+0.100000 = 1.800000)
./SumaVectores 100000
# Tamaño Vectores: 100000 (4B)
# Tiempo: 0.000520104 --- Tamaño Vectores: 100000
# V1[0]+V2[0] = V3[0] (10000.000000+10000.000000 = 20000.000000)
# V1[99999]+V2[99999] = V3[99999] (19999.900000+0.100000 = 20000.000000)
./SumaVectores 1000000
# Tamaño Vectores: 1000000 (4B)
# [1]    32567 segmentation fault (core dumped)  ./SumaVectores 1000000
\end{lstlisting}

\section{}\label{ej1-6}

\subsubsection{Enunciado}
En el código del \textit{Listado 1} se utiliza la función \texttt{clock\_gettime()} para obtener el tiempo de ejecución del trozo de código que calcula la suma de vectores.
El código se imprime la variable \texttt{ncgt}.
¿Qué contiene esta variable?

¿En qué estructura de datos devuelve \texttt{clock\_gettime()} la información de tiempo?
Indicar el tipo de estructura de datos, describir la estructura de datos, e indicar los tipos de datos que usa.
¿Qué información devuelve exactamente esta función en su estructura de datos?
¿Qué representan los valores numéricos que devuelve?

\subsubsection{Solución}

La variable \texttt{ncgt} contiene el tiempo que tarda el programa en realizar los cómputos con los tres vectores.

La llamada a \texttt{clock\_gettime()} devuelve el tiempo en una estructura \texttt{timespec}, que es una estructura definida en \texttt{<time.h>} y que está compuesta por un dato \texttt{time\_t} que contiene el número de segundos desde el 1 de enero de 1970 y un dato \texttt{long} que contiene el número de nanosegundos pasados desde el último segundo.

\section{}\label{ej1-7}

\subsubsection{Enunciado}

Rellenar una tabla con los tiempos de ejecución del código del \textit{Listado 1} para vectores locales, globales y dinámicos.
Obtener estos resultados usando scripts (partir del script que hay en el seminario).
Debe haber una tabla para \textit{atcgrid} y otra para su PC\@.
En la columna ``Bytes de un vector'' hay que poner el total de bytes reservado para un vector.

\subsubsection{Solución}

\begin{center}
\begin{table}[!h]
\begin{tabular}{r r r r r}
\textbf{Componentes} & \textbf{Bytes por vector} & \textbf{Tº vect. locales} & \textbf{Tº vect. globales} & \textbf{Tº vect. dinámicos} \\
\toprule
65536                   & 524288                      & 0                         & 0                          & 0                           \\
131072                  & 1048576                     & 0                         & 0                          & 0                           \\
262144                  & 2097152                     & 0                         & 0                          & 0                           \\
524288                  & 4194304                     & core dumped               & 0                          & 0                           \\
1048576                 & 8388608                     & core dumped               & 0.01                       & 0                           \\
2097152                 & 16777216                    & core dumped               & 0.01                       & 0.01                        \\
4194304                 & 33554432                    & core dumped               & 0.02                       & 0.02                        \\
8388608                 & 67108864                    & core dumped               & 0.03                       & 0.03                        \\
16777216                & 134217728                   & core dumped               & 0.07                       & 0.06                        \\
33554432                & 268435456                   & core dumped               & 0.13                       & 0.11                        \\
67108864                & 536870912                   & core dumped               & 0.13                       & 0.22                        \\
\end{tabular}
\caption{Resultados de \texttt{SumaVectores.c} en mi PC}
\end{table}

\begin{table}[!h]
\begin{tabular}{r r r r r}
\textbf{Componentes} & \textbf{Bytes por vector} & \textbf{Tº vect. locales} & \textbf{Tº vect. globales} & \textbf{Tº vect. dinámicos} \\
\toprule
65536                & 524288                    & 0                         & 0                          & 0                           \\
131072               & 1048576                   & 0                         & 0                          & 0                           \\
262144               & 2097152                   & 0                         & 0                          & 0                           \\
524288               & 4194304                   & core dumped               & 0                          & 0                           \\
1048576              & 8388608                   & core dumped               & 0.01                       & 0.01                        \\
2097152              & 16777216                  & core dumped               & 0.01                       & 0.01                        \\
4194304              & 33554432                  & core dumped               & 0.02                       & 0.02                        \\
8388608              & 67108864                  & core dumped               & 0.03                       & 0.03                        \\
16777216             & 134217728                 & core dumped               & 0.07                       & 0.06                        \\
33554432             & 268435456                 & core dumped               & 0.13                       & 0.13                        \\
67108864             & 536870912                 & core dumped               & 0.13                       & 0.25                        \\
\end{tabular}
\caption{Resultados de \texttt{SumaVectores.c} en el \textit{atcgrid}}
\end{table}
\end{center}

\section{}\label{ej1-8}

\subsubsection{Enunciado}

Representar en una misma gráfica los tiempos de ejecución obtenidos en \textit{atcgrid} y en su PC para vectores locales, globales y dinámicos (eje $y$) en función del tamaño en bytes de un vector (por tanto, los valores de la segunda columna de la tabla, que están en escala logarítmica, deben estar en el eje $x$).
Utilizar escala logarítmica en el eje de ordenadas (eje $y$).
¿Hay diferencias en los tiempos de ejecución?

\subsubsection{Solución}

TODO\@: Escribir esta solución.

\section{}\label{ej1-9}

\subsubsection{Enunciado}

Cuando se usan vectores locales, globales y dinámicos, ¿se obtiene error para alguno de los tamaños?
¿A qué cree que es debido lo que ocurre?

\subsubsection{Solución}

La violación de segmento al usar vectores locales se produce porque el programa trata de acceder a valores de la pila superiores al espacio reservado.

No se produce ningún error usando vectores globales porque el tamaño de los vectores viene indicado en tiempo de compilación, por lo que se reserva espacio suficiente para trabajar con ellos.

Con vectores dinámicos no se obtiene error en los tamaños porque la reserva de memoria se hace de forma dinámica en el \textit{heap}.
La única forma de que no se pudisese reservar memoria sería que no quedara espacio disponible en memoria principal.

\section{}\label{ej1-10}

\subsubsection{Enunciado}

¿Cuál es el máximo valor que se puede almacenar en la variable \texttt{N} teniendo en cuenta su tipo?
Razonar la respuesta.

Modificar el código fuente C (en el PC) para que el límite de los vectores cuando se declaran como variables globales sea igual al máximo número que se puede almacenar en la variable \texttt{N} y generar el ejecutable.
¿Qué ocurre?
¿A qué es debido?
Mostrar lo que ocurre.

\pagebreak

\subsubsection{Solución}

El máximo valor almacenable en \texttt{N} es $2^{32}-1 = 4294967295$, definido en la constante \texttt{UINT\_MAX} de \texttt{<limits.h>}.
Es el valor máximo almacenable en un \texttt{unsigned} en x86\_64.

\begin{lstlisting}[language=sh]
gcc -O2 SumaVectores.c -o SumaVectores -lrt
# SumaVectores.c: In function ‘main’:
# SumaVectores.c:98:18: warning: iteration 2147483647 invokes undefined behavior [-Waggressive-loop-optimizations]
#    98 |  for (i=0; i<N; i++) {
#       |                 ~^~
# SumaVectores.c:98:2: note: within this loop
#    98 |  for (i=0; i<N; i++) {
#       |  ^~~
# SumaVectores.c:107:18: warning: iteration 2147483647 invokes undefined behavior [-Waggressive-loop-optimizations]
#   107 |  for (i=0; i<N; i++)
#       |                 ~^~
# SumaVectores.c:107:2: note: within this loop
#   107 |  for (i=0; i<N; i++)
#       |  ^~~
./SumaVectores 1
# Tamaño Vectores: 4294967295 (4B)
# ^C
\end{lstlisting}

Al modificar el fuente, el tamaño de los vectores es tan grande como el valor máximo asumible por el sistema, por lo que a valores tan altos el comportamiento del mismo no está definido.
De hecho, el programa debe ser finalizado manualmente debido a la cantidad de tiempo que tarda en ejecutarse.
