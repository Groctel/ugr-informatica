\chapter{Órdenes básicas de UNIX/Linux}

\section{Los intérpretes de órdenes}

Un intérprete de órdenes (\textit{shell} en la terminología UNIX) es un programa normal de usuario capaz de ejecutar órdenes escritas en él.
Existen diferentes shells, como Bourne Again Shell (bash), TC Shell (tcsh) y Z Shell (zsh)\footnote{Al existir diferentes shells, estos apuntes siguen el estándar POSIX a menos que se especifique lo contrario. De esta forma, las instrucciones explicadas a lo largo de ellos pueden ejecutarse en cualquier shell.}.
Al estar construidas como programas de usuario normales, tienen la ventaja de que se pueden elegir según las preferencias del usuario.

Las shells instaladas en el sistema están registradas en el fichero \code{/etc/shell}.
Para cambiar de shell podemos utilizar la siguiente orden:

\begin{lstlisting}[language=sh]
chsh -s /bin/zsh # Por ejemplo, para cambiar a zsh
\end{lstlisting}

Lógicamente, la shell debe estar instalada en el sistema para que esta orden tenga efecto.
Tras ejecutarla, basta con cerrar y abrir sesión para ver los cambios.
También puede ejecutarse el nombre de la shell en otra para iniziarla.
Por ejemplo, trabajando en \code{bash} podemos ejecutar \code{zsh} para abrir una shell de \code{zsh} en esa misma terminal.
Para cerrar la shell basta con ejecutar la orden \code{exit}.

\subsection{La orden \code{man}}

Como se comentó en la práctica anterior, \code{man} abre el paginador del manual del sistema.
Al ejecutar la orden, \code{man} busca la página asociada a la consulta pasada como argumento y la muestra en caso de que exista.
Si hubiera más de una página para dicha consulta, \code{man} sólo muestra la primera.

El paginador utiliza los siguientes comandos para buscar o desplazarse por el texto:

\begin{itemize}
	\item\code{f}\textbf{:} Avanzar página.
	\item\code{b}\textbf{:} Retrasar página.
	\item\code{j}\textbf{:} Moverse una línea hacia arriba.
	\item\code{k}\textbf{:} Moverse una línea hacia abajo.
	\item\code{/string}\textbf{:} Busca la cadena \code{string} (incluyendo los espacios en blanco) desde la primera línea mostrada en la pantalla en adelante, marcando todas las ocurrencias encontradas.
	\item\code{?string}\textbf{:} Busca la cadena \code{string} (incluyendo los espacios en blanco) desde la primera línea mostrada en la pantalla hacia atrás, marcando todas las ocurrencias encontradas.
	\item\code{n}\textbf{:} Siguiente elemento en la búsqueda.
	\item\code{N}\textbf{:} Elemento previo en la búsqueda.
	\item\code{v}\textbf{:} Lanza (si es posible) el editor por defecto para editar el fichero que estamos viendo.
	\item\code{q}\textbf{:} Cierra el manual.
\end{itemize}

\subsection{La jerarquía de permisos}

Al ejecutar la orden \code{ls -l} podemos ver que a la izquierda de los ficheros se muestra una tabla de diez caracteres de ancho.
Cada una de las filas representa los permisos de lectura, escritura y ejecución que tienen los diferentes usuarios sobre su respectivo fichero.

El primer carácter es \code{-} si se trata de un fichero y \code{d} si se trata de un directorio.
Los nueve caracteres restantes están divididos en tres grupos de tres caracteres.
El primer grupo representa los permisos del usuario creador del fichero, los tres siguientes los de los usuarios pertenecientes al grupo de usuarios del creador y los tres últimos los del resto de usuarios.

Cada uno de estos grupos define los siguentes permisos:

\begin{itemize}
	\item\code{r}\textbf{:} Lectura (\textit{read}).
	\item\code{w}\textbf{:} Escritura (\textit{write}).
	\item\code{x}\textbf{:} Ejecución (\textit{execute}).
	\item\code{-}\textbf{:} Falta de permiso en la posición correspondiente.
\end{itemize}

\begin{lstlisting}[language=sh]
ls -l
# -rw-rw-r--  1 groctel groctel 84831 sep 22 18:56 'Fisica.zip'
# drwxr-xr-x  4 groctel groctel  4096 sep 19 19:47  Games
# drwxr-xr-x  7 groctel groctel  4096 oct  6 21:19  UGR
\end{lstlisting}

\section{Metacaracteres de ficheros}

A la hora de nombrar ficheros y directorios es muy útil utilizar caracteres especiales que puedan servirnos de comodín para expresar combinaciones de caracteres o caracteres desconocidos en el momento, sobre todo si nos encontramos con muchos ficheros con nombres similares.
Podemos utilizar cinco metacaracteres:

\begin{itemize}
\item\code{?}\textbf{:} Representa un sólo carácter cualquiera en la posición indicada.
\item\code{*}\textbf{:} Representa de cero a infinitos caracteres cualesquiera en la posición indicada.
\item\code{[]}\textbf{:} Designan varios caracteres individuales que pueden ocupar el espacio en el que se declaran.
\item\code{\{\}}\textbf{:} Designan cadenas de caracteres separadas por comas que pueden ocupar el espacio en el que se declaran y rangos de caracteres separados por \code{..}.
\item\code{\~{}}\textbf{:} Abrevia la ruta absoluta de \code{/home/\$USER}.
\end{itemize}

\section{Ejercicios}

\titleformat{\subsection}[block]{\normalfont\bfseries\Large}{Ejercicio \thesubsection}{0pt}{}[]

\subsection{}\label{ej1-1}

\subsubsection{Enunciado}

Cree el siguiente árbol de directorios a partir de un directorio de su cuenta de usuario:

\begin{itemize}
	\item\code{ejercicio1}
	\begin{itemize}
		\item\code{Ejer1}
		\begin{itemize}
			\item\code{Ejer21}
		\end{itemize}
		\item\code{Ejer2}
		\item\code{Ejer3}
	\end{itemize}
\end{itemize}

Indique también cómo sería posible crear toda esa estructura de directorios mediante una única orden (mire las opciones de la orden de creación de directorios mediante \code{man mkdir}).
Posteriormente realice las siguientes acciones:

\begin{itemize}
	\item En \code{Ejer1}, cree los ficheros \code{arch100.txt}, \code{filetags.txt}, \code{practFS.ext} y \code{robet201.me}.
	\item En \code{Ejer21}, cree los ficheros \code{robet202.me}, \code{ejer11sol.txt} y \code{blue.me}.
	\item En \code{Ejer2}, cree los ficheros \code{ejer2arch.txt}, \code{ejer2filetags.txt} y \code{readme2.pdf}.
	\item En \code{Ejer3}, cree los ficheros \code{ejer3arch.txt}, \code{ejer3filetags.txt} y \code{readme3.pdf}.
\end{itemize}

¿Podrían realizarse las acciones anteriores empleando una única orden?
Indique cómo.

\subsubsection{Solución}

Creación del árbol de directorios en una única orden:

\begin{lstlisting}[language=sh]
mkdir -p ejercicio1/Ejer1/Ejer21 ejercicio1/Ejer2 ejercicio1/Ejer3
\end{lstlisting}

Creación de los ficheros en una única línea:

\begin{lstlisting}[language=sh]
touch Ejer1/arch100.txt Ejer1/filetags.txt Ejer1/practFS.ext Ejer1/robet201.me \
      Ejer21/robet202.me Ejer21/ejerllsol.txt Ejer21/blue.me \
      Ejer2/ejer2arch.txt Ejer2/ejer2filetags.txt Ejer2/readme2.pdf \
      Ejer3/ejer3arch.txt Ejer3/ejer3filetags.txt Ejer3/readme3.pdf \
\end{lstlisting}

Aunque la orden \code{touch} ocupa varias líneas, se interpreta como una sola gracias a los caracteres \code{\\} que se encuentran al final de cada una.

\subsection{}\label{ej1-2}

\subsubsection{Enunciado}

Situados en el directorio \code{ejercicio1} creado anteriormente, realice las siguientes acciones:

\begin{itemize}
	\item Mueva el directorio \code{Ejer21} al directorio \code{Ejer2}.
	\item Copie los ficheros de \code{Ejer1} cuya extensión tenga una \code{x} al directorio \code{Ejer3}.
	\item Si estamos situado en el directorio \code{Ejer2} y ejecutamos la orden \code{ls -la ../Ejer3/*arch*}, ¿qué fichero/s, en su caso, debería mostrar?
\end{itemize}

\subsubsection{Solución}

\begin{lstlisting}[language=sh]
mv Ejer1/Ejer21 Ejer2
mv Ejer1/*.*x* Ejer3
cd Ejer2
ls -la ../Ejer3/*arch*
# -rw-r--r-- 1 groctel groctel 0 oct  6 21:23 ../Ejer3/arch100.txt
# -rw-r--r-- 1 groctel groctel 0 oct  6 21:26 ../Ejer3/ejer3arch.txt
\end{lstlisting}

\subsection{}\label{ej1-3}

\subsubsection{Enunciado}

Si estamos situados en el directorio \code{Ejer2}, indique la orden necesaria para listar sólo los nombres de todos los ficheros del directorio padre.

\subsubsection{Solución}

\begin{lstlisting}[language=sh]
ls ..
\end{lstlisting}

\subsection{}\label{ej1-4}

\subsubsection{Enunciado}

Liste los ficheros que estén en su directorio actual y fíjese en alguno que no disponga de la fecha y hora actualizada, es decir, la hora actual y el día de hoy.
Ejecute la orden \code{touch} sobre dicho fichero y observe qué sucede sobre la fecha del citado fichero cuando se vuelva a listar.

\subsubsection{Solución}

\begin{lstlisting}[language=sh]
ls -l
# total 4
# drwxr-xr-x 2 groctel groctel 4096 oct  6 21:25 Ejer21
# -rw-r--r-- 1 groctel groctel    0 oct  6 21:25 ejer2arch.txt
# -rw-r--r-- 1 groctel groctel    0 oct  6 21:25 ejer2filetags.txt
# -rw-r--r-- 1 groctel groctel    0 oct  6 21:25 readme2.pdf
touch readme2.pdf
ls -l readme2.pdf
# -rw-r--r-- 1 groctel groctel 0 oct  6 21:38 readme2.pdf
\end{lstlisting}

\subsection{}\label{ej1-5}

\subsubsection{Enunciado}

La organización del espacio en directorios es fundamental para poder localizar fácilmente aquello que estemos buscando.
En ese sentido, realice las siguientes acciones dentro de su directorio \code{\$HOME} (es el directorio por defecto sobre el que trabajamos al entrar en el sistema):

\begin{itemize}
	\item Obtenga en nombre de camino absoluto (absolute pathname) de su directorio \code{\$HOME}. ¿Es el mismo que el de su compañero/a?
	\item Cree un directorio para cada asignatura en la que se van a realizar prácticas de laboratorio y, dentro de cada directorio, nuevos directorios para cada una de las prácticas realizadas hasta el momento.
	\item Dentro del directorio de la asignatura fundamentos del software (llamado \code{fs}) y dentro del directorio creado para esta práctica, copie los ficheros \code{hosts} y \code{passwd} que se encuentran dentro del directorio \code{/etc}.
	\item Muestre el contenido de cada uno de los ficheros.
\end{itemize}

\subsubsection{Solución}

\begin{lstlisting}[language=sh]
pwd
# /home/groctel
mkdir -p lab/fs/p1 lab/fp/p1 lab/calculo/p1 lab/algebra
cp /etc/hosts lab/fs/p1
cp /etc/passwd lab/fs/p1
more lab/fs/p1/hosts
# (Muestra del contenido del fichero)
more lab/fs/p1/passwd
# (Muestra del contenido del fichero)
\end{lstlisting}

\subsection{}\label{ej1-6}

\subsubsection{Enunciado}

Situados en algún lugar de su directorio principal de usuario (directorio \code{\$HOME}), cree los directorios siguientes: \code{Sesion.1}, \code{Sesion.10}, \code{Sesion.2}, \code{Sesion.3}, \code{Sesion.4}, \code{Sesion.27}, \code{Prueba.1} y \code{Sintaxis.2} y realice las siguientes tareas:

\begin{itemize}
	\item Borre el directorio \code{Sesion.4}.
	\item Liste todos aquellos directorios que empiecen por \code{Sesion.} y a continuación tengan un único carácter.
	\item Liste aquellos directorios cuyos nombres terminen en \code{.1}.
	\item Liste aquellos directorios cuyos nombres terminen en \code{.1} o \code{.2}.
	\item Liste aquellos directorios cuyos nombres contengan los caracteres \code{si}.
	\item Liste aquellos directorios cuyos nombres contengan los caracteres \code{si} y terminen en \code{.2}.
\end{itemize}

\subsubsection{Solución}

\begin{lstlisting}[language=sh]
mkdir Sesion.1 Sesion.10 Sesion.2 Sesion.3 Sesion.4 Sesion. 27 Prueba.1
rm Sesion.4
ls Sesion.?
# Sesion.1  Sesion.2  Sesion.3
ls *.1 *.2
# Prueba.1  Sesion.1  Sesion.2
ls *si*.*
# Sesion.1  Sesion.10  Sesion.2  Sesion.27  Sesion.3
ls *si*.2
# Sesion.2
\end{lstlisting}

\subsection{}\label{ej1-7}

\subsubsection{Enunciado}

Desplacémonos hasta el directorio \code{/bin}.
Genere los siguientes listados de ficheros (siempre de la forma más compacta y utilizando los metacaracteres apropiados):

\begin{itemize}
	\item Todos los ficheros que contengan sólo cuatro caracteres en su nombre.
	\item Todos los ficheros que comiencen por los caracteres \code{d}, \code{f}.
	\item Todos los ficheros que comiencen por las parejas de caracteres \code{sa}, \code{se}, \code{ad}.
	\item Todos los ficheros que comiencen por \code{t} y acaben en \code{r}.
\end{itemize}

\subsubsection{Solución}

\begin{lstlisting}[language=sh]
cd /bin
ls ????
ls d* f*
ls sa* se* ad*
ls t*r
\end{lstlisting}

\subsection{}\label{ej1-8}

\subsubsection{Enunciado}

Liste todos los ficheros que comiencen por \code{tem} y terminen por \code{.gz} o \code{.zip}:

\begin{itemize}
	\item De su directorio \code{\$HOME}.
	\item Del directorio actual.
\end{itemize}

¿Hay alguna diferencia en el resultado de su ejecución? Razone la respuesta

\subsubsection{Solución}

\begin{lstlisting}[language=sh]
ls ~/tem*.gz ~/tem*.zip
# ls: cannot access '/home/tem*.gz': No such file or directory
# ls: cannot access '/home/tem*.zip': No such file or directory
ls tem*.gz tem*.zip
# ls: cannot access 'tem*.gz': No such file or directory
# ls: cannot access 'tem*.zip': No such file or directory
\end{lstlisting}

La única diferencia en este caso es la sintaxis.
Siendo un rango de caracteres y tipo de ficheros tan acotados habría sido sorprendente que ambas órdenes encontraran ficheros.
En cualquier otro caso, con otras restricciones más democráticas se conseguirían resultados similares a los del Ejercicio 2.

\subsection{}\label{ej1-9}

\subsubsection{Enunciado}

Muestre del contenido de un fichero regular que contenga texto:

\begin{itemize}
	\item Las 10 primeras líneas
	\item Las 5 últimas líneas
\end{itemize}

\subsubsection{Solución}

\begin{lstlisting}[language=sh]
head /etc/fstab
tail -5 /etc/fstab
\end{lstlisting}

\subsection{}\label{ej1-10}

\subsubsection{Enunciado}

Cree un fichero empleando para ello cualquier editor de textos y escriba en el mismo varias palabras en diferentes líneas.
A continuación trate de mostrar su contenido de manera ordenada empleando diversos criterios de ordenación.

\subsubsection{Solución}

\begin{lstlisting}[language=sh]
sort -f excelente.txt
# de
# ejemplo
# texto
sort -r excelente.txt
# texto
# ejemplo
# de
\end{lstlisting}

\subsection{}\label{ej1-11}

\subsubsection{Enunciado}

¿Cómo podría ver el contenido de todos los ficheros del directorio actual que terminen en \code{.txt} o \code{.c}?

\subsubsection{Solución}

\begin{lstlisting}[language=sh]
cat *.txt *.c
\end{lstlisting}

\titleformat{\subsection}[block]{\normalfont\bfseries\Large}{\aqademia@subsection\thesubsection:\ }{0pt}{}[]
